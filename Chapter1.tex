%!TEX root = If.tex
\documentclass[If.tex]{subfiles}

\begin{document} 
\chapter{Accessibility}\label{chap:accessibility}
\section{The context-sensitivity of conditionals} \label{sect:context}
Even holding fixed which propositions are the antecedent and consequent, there are different propositions that could be expressed by uttering a conditional on different occasions. For one thing, it is well known that the difference between ‘indicative morphology’ and ‘counterfactual morphology’ is semantically significant, as exemplified by Ernest Adams's famous minimal pair:
\begin{prop}
	\nitem	
	\begin{prop}
		\aitem \label{indoswald}
			If Oswald didn't kill Kennedy, someone else did
		\aitem \label{cfoswald}
			If Oswald hadn't killed Kennedy, someone else would have
	\end{prop}
\end{prop}
This difference seems to be driven by something about the pattern of tense markings that appear on the surface within the main and subordinate clauses; however, the way in which these morphological differences actually contribute to meaning is quite controversial. For the moment, we will take a capacity to sort “indicative” from “counterfactual” conditionals for granted, and we will try only to use examples whose classification is uncontroversial.

For another thing, the proposition expressed by a conditional can vary even when the wording remains exactly the same. In the case of counterfactuals (i.e.~sentences syntactically like \ref{cfoswald}), this variability can be illustrated by an example of Quine's \parencite[described by][]{LewisCounterfactuals}:
\begin{prop}
	\nitem
	\begin{prop}
		\aitem \label{nukes}
			If Caesar had fought in Korea, he would have used nuclear weapons
		\aitem \label{catapults}
			If Caesar had fought in Korea, he would have used catapults
	\end{prop}
\end{prop}
Given appropriate conversational background, it seems that one could speak the truth by uttering each of these sentences, though in no normal background could one speak the truth by uttering their conjunction. Similarly, each of \ref{jumpedkilled} and \ref{jumpednet} \citep{JacksonCTC} seems like something one could use to assert a truth, although their conjunction seems absurd:
\begin{prop}
	\nitem	
	\begin{prop}
		\aitem \label{jumpedkilled}
		If I had jumped out the window right now, I would have been killed
		\aitem \label{jumpednet}
		If I had jumped out the window right now, I would have done so only because there was a soft landing laid out for me
	\end{prop}
\end{prop}
In the case of indicatives (i.e.~sentences syntactically like \ref{indoswald}), the case for context-sensitivity is more controversial, but still strong. The main observations here are due to \citet{GibbardTRTC}, although the following case is due to Bennett. The setting is one where there are two channels by which water can flow from an upper reservoir to a lower reservoir; each is closed off by a gate. Speaker A discovers that the east gate is closed and asserts
\begin{prop}
	\nitem \label{west}
	If the water got to the bottom, it got there via the west channel% and not the east channel
\end{prop}
Speaker B discovers that the west gate is closed and asserts
\begin{prop}
	\nitem \label{east}
	If the water got to the bottom, it got there via the east channel% and not the west channel
\end{prop}
Intuitively, both speeches seem true.  But the conjunction of \ref{west} and \ref{east} is extremely odd. It is thus very natural to explain the acceptability of \ref{west} and \ref{east} by appeal to context-dependence.

% The structure of the “closest accessible world” theory makes available two possible accounts of such variation

We propose that all of the differences we have just been talking about are to be explained by differences in what it takes for a world to be “accessible” in the sense relevant to the context. For example, the accessibility relation operative in the context where \ref{nukes} seems true requires match with respect to what kinds of military technology is in use at a given historical period, whereas the one operative in the context where \ref{catapults} seems true does not require this, but does require match with respect to what kinds of military technology were actually available to particular people.%
\footnote{Note that if, unbeknownst to us, there are no such things as nuclear weapons and catapults are still in wide use even now, the true-seeming utterances of \ref{nukes} are in fact false; similarly, if unbeknownst to us, Martians actually provided Caesar with nuclear weapons which he only didn't use because he had no need to resort to them, the true-seeming utterances of \ref{catapults} are in fact false.}
Similarly, in the context where \ref{jumpedkilled} seems true, accessibility requires match with respect to the laws of nature and the course of history up until very shortly before the time of utterance, whereas in the context where \ref{jumpednet} seems true, accessibility requires much less than this.%
\footnote{The relevant notion of “match” here may not require \emph{perfect} match in all respects, no matter how microscopic: see \cite{DorrACM}.}
In the contexts where \ref{west} and \ref{east} seem true, meanwhile, accessibility involves something like compatibility with the knowledge of the speaker at the time of utterance, and the contextual variability is primarily due to variation in who is speaking and when.

Even holding fixed who is speaking and when, we think there is plenty of room for further contextual variation in the interpretation of the accessibility parameter for indicative conditionals (like \ref{west} and \ref{east}).  First, in some cases the knowledge state of other relevant individuals or groups may be relevant, rather than just the knowledge of the speaker.  Secondly, the epistemic relation that fixes accessibility may not always be knowledge---in some settings it may be something more demanding (e.g.\ being known for sure), while in others it may be laxer.   Thirdly, when questions are salient (or ‘under discussion’) in a conversation, they may serve to constrain accessibility, so that the worlds accessible from any given world are required to match that world with respect to the answers to those questions, in addition to being epistemically live.  (The phenomenon of constraint by questions under discussion also arises for counterfactuals.)  We will return later to all of these potential dimensions of contextual variation, and put them to work in explaining various data points.  

The structure of the “closest accessible world” analysis does not require attributing the differences we have been considering to differences in the accessibility parameter.  One could instead attribute some or all of them to differences in the operative closeness relation, in which case one could, if one wished, hold that the accessibility parameter is always wide open (i.e.\ that every world always counts as accessible from every other).  Proponents of this approach could even mimic our specific suggestions about what plays the accessibility role in the examples we have considered by saying that while the operative accessibility relations do not vary, the operative closeness relations vary in such a way that each world we would count as accessible counts in context as closer than each world we would count as inaccessible.  Much of our picture could survive this change in the division of labour between accessibility and closeness; however, the picture of the accessibility parameter as the primary driver of context-sensitivity has some advantages which will emerge in the present chapter.  


% \begin{itemize}
% 	\item
% 	or maybe go back to the lingo of ‘presupposition’ here
% 	\item
% 	be more explicit that ‘accessibility’ is a \emph{property} of worlds, not a \emph{set} of worlds
% \end{itemize}

\section{The difference between indicatives and counterfactuals} 
\label{sect:indcf}
The grammatical difference between indicative and counterfactual conditionals clearly has some systematic semantic significance: given what we claimed the previous section, this must consist in some systematic difference in the values that are allowed for the accessibility parameter.  Loosely following \citet{vonFintelPSC}, we suggest that this difference takes the form of general constraint on indicatives which does not apply to counterfactuals: namely that for an indicative conditional, \emph{accessibility must entail epistemic possibility}.  Roughly, to say that a world is epistemically possible is to say that it is a live candidate for being how things actually are, from the perspective of some contextually relevant individual or group. 

This explains why we are forced to use counterfactual morphology to get across the plausible thoughts we naturally would try to get across by uttering \ref{nukes} or \ref{catapults}.  In the contexts they naturally evoke, these sentences are non-vacuously true---that is, \ref{nukes} evokes a context where ‘If Caesar had fought in Korea, he would never have used nukes’ is false, and \ref{catapults} evokes a context where ‘If Caesar had fought in Korea, he would never have used catapults’ is false.%
\footnote{This is explained in part by the presupposition of nonvacuity, to be discussed in \autoref{sect:nonvacuity} below.}
It is very hard to get oneself into a mindset where the proposition that Caesar fought in Korea is regarded as epistemically “live”.  (And it is harder still to get oneself into a mindset where there are “live” worlds in which Caesar was present in Korea but where one is still willing to assert something that entails that Caesar didn't use nukes in Korea or something that entails that Caesar didn't use catapults in Korea.)  Similarly, conditionals beginning with ‘If I didn't exist right now\ldots{}’ are generally much more intelligible than those beginning with ‘If I don't exist right now\ldots{}’: making sense of the latter requires entering into a most unusual sense of openness to nihilistic metaphysics.

Should we also posit a correlative constraint for counterfactuals, according to which the value of the accessibility parameter for a counterfactual has to be a property of worlds that does \emph{not} entail epistemic liveness?  The generalisation that counterfactuals are in fact interpreted using such accessibility relations looks quite plausible.  Often, the antecedents of counterfactuals are clearly not regarded as epistemically live, in which case the only way for them to be nonvacuously true is for some worlds that are not epistemically live to be accessible.  And even when the speaker regards the antecedent of a counterfactual as epistemically live, the presence of non-epistemically live worlds among the accessible ones can be crucial, as in Anderson's example:
\begin{prop}
	\nitem
	If Jones had taken arsenic, he would have shown just exactly those symptoms which he does in fact show.
\end{prop}
As \citet{vonFintelPSC} notes, the speaker obviously knows that Jones is showing just exactly those symptoms which he does in fact show, so the conditional will be \emph{obviously}, and uninterestingly, true if we interpret it using a notion of accessibility that entails compatibility with the speaker's knowledge.  However, it's not clear that we need to give the generalisation that accessibility for counterfactuals does not entail liveness the same rule-like semantic status as the generalisation that accessibility for indicatives does entail liveness.  If we treat one of the generalisations as a rule, we could perhaps recover the other one using some general principle enjoining co-operative speakers to choose the more constrained of two possible forms when possible.%
\footnote{If we implemented one of the generalisations as a \emph{presupposition} (that accessibility entails or fails to entail epistemic liveness), we could derive the other via the principle \emph{Maximise Presupposition} \citep{HeimAD}, according to which when one of two sentences with the same “assertive content” has a stronger presupposition, speakers are expected to utter that one when its presupposition is in fact common ground in the conversation.}
We will not take a stand on whether this should be done.  

The idea that accessibility for indicative conditionals entails epistemic liveness fits naturally with an account of epistemic modals: ‘It must be that $P$’; ‘It might be that $P$’; ‘It is possible that $P$’, etc. In orthodox fashion, we take these to be context-sensitive, in a way that can be represented by an accessibility parameter: ‘It must be that $P$’ is true iff $P$ is true at all accessible worlds, and ‘It might be that $P$’ and ‘It is possible that $P$’ are true iff $P$ is true in some accessible world. The “epistemic” character of these modals consists in the fact that accessibility always has to entail epistemic liveness from the standpoint of a contextually relevant individual or group---in simple cases, this will be the speaker or a group containing the speaker (where liveness for a group is understood as entailing liveness for each member of the group).  Moreover, when one uses both epistemic modals and indicative conditionals, the default is for both to be interpreted uniformly (using the same notion of accessibility).  This means that the argument-schemas
\begin{prop}
	\sitem[Must-if] \label{mustif}
	\parbox[t]{\linewidth}{It must be that $Q$\\	
	Therefore, if $P$, $Q$}
\end{prop}
and 
\begin{prop} \label{mightpres}
	\sitem[Might-preservation]
	\parbox[t]{\linewidth}{It might be that $P$\\
	If $P$, $Q$\\
	Therefore, it might be that $Q$}
\end{prop}
are both valid, in the sense that on any uniform interpretation, the propositions expressed by the premises entail the proposition expressed by the conclusion.  Arguments of these forms certainly \emph{seem} valid, so getting them to come out valid is a significant advantage of the approach that appeals to variation in the accessibility parameter to explain the relevant context-sensitivity over the competing approach mentioned in the previous section that keeps accessibility constant and appeals to variation in the closeness relation.  Moreover, as we will see in \autoref{sect:quasivalidity}, the validity of \ref{mustif} helps to explain certain facts which have often been used to argue for competing views such as the theory that indicative conditionals are material conditionals.

%Thus, when this parameter of interpretation is held fixed, the proposition expressed by ‘It must be that Φ or~Ψ’ entails the one expressed by the indicative conditional ‘If not~Φ, then~Ψ’. Also, the propositions expressed by ‘It might be that~Φ’ and ‘If~Φ, then~Ψ’ jointly entail the proposition expressed by ‘It might be that~Ψ’.

One might worry that the proposed constraint on accessibility for indicative conditionals is too demanding.  Since an utterer of \ref{indoswald} (‘If Oswald didn't kill Kennedy, someone else did’) is clearly not presenting it as merely vacuously true (for reasons to be explored in  \autoref{sect:nonvacuity}), the operative notion of accessibility must be one on which they are assuming that there are accessible worlds where Oswald didn't kill Kennedy; given the proposed constraint, this means the speaker must be regarding the possibility that Oswald didn't kill Kennedy as epistemically live.  This seems strange: after all, most of us know perfectly well that Oswald did kill Kennedy and yet are happy to utter \ref{indoswald}, and not simply in the spirit in which we might utter ‘If Oswald didn't kill Kennedy I'm a monkey's uncle’. There are a few different routes one might follow in responding to this objection. On a radical approach, the proposition expressed by utterances of \ref{indoswald} by us knowledgeable folk is in fact vacuously true, but when we utter it, we take for granted the false proposition that it isn't.  (This could be a kind of pretended epistemic modesty, or else a briefly held false belief that we don't really know who shot Kennedy.)  Similarly, the radical view will say that despite the appeal of ‘I might be dreaming’ is natural, utterances of it will express a false proposition that is treated as acceptable either because we are pretending that it is true, or because we falsely believe that it is true (since at the time of utterance we are taken in by the thought that we don't know whether we are dreaming). On a less error-theoretic view, \ref{indoswald} would be nonvacuously true in the context that would be evoked if one of us were to assert it, and ‘It's possible that Oswald didn't killed Kennedy’ is also true by the standards of this context.  On this approach, what counts as “live” in the sense relevant to the constraint is itself a contextually variable matter, and speeches like \ref{indoswald} push us towards an especially permissive resolution of this context sensitivity.  One might want to link this contextual variation to context-sensitivity in the verb ‘know’, saying that ‘We don't know whether Oswald killed Kennedy’ expresses a truth in the relevant contexts (although it expresses a falsehood in many others).  Alternatively, one might conceive of liveness as setting a contextually flexible epistemic standard that does not always march in lock-step with the one associated with ‘know’: for example, liveness could in some cases be a matter of consistency with some more attenuated body of knowledge, or liveness might be a matter of consistency with a body of propositions of which one has an especially secure kind of knowledge.

Epistemic modals need not always be anchored to the facts about what is live at the time of speech. In a sentence like ‘It was possible that it had rained, since the ground seemed a bit damp’, the operative epistemic perspective is in the past: what we are saying, roughly, is that what the relevant people knew at the relevant past time was consistent with the proposition that it had rained earlier than that time. There is no suggestion that the speaker is ignorant or in any way open at the time of speech to the possibility that it had rained: what matters is just the epistemic standpoint of the salient person or group (or more abstract ‘perspective’) as of the target time.%
\footnote{In some cases one needs to work with such notions as ‘the evidence that was available at the time’ even when there is no relevant person around to gather it: ** Add example about the early universe **}
‘It was possible that it was raining’ and ‘It was possible that it was going to rain’ are similar. ‘Might have’ and ‘could have’ claims also have a reading that works like this: ‘It might have been raining’ can mean ‘It was possible that it was raining’, although it can also mean ‘It is possible that it has been raining’.%
\footnote{In some other natural languages where the analogues of ‘might’ are normal tensed verbs, this ambiguity is lexically resolved. In English ‘have to’ works like this: we distinguish ‘It has to have been raining’ from ‘It had to be raining’.{]}}

Given the intimate relation between epistemic modals and indicative conditionals, we would thus expect that in some cases, the accessibility parameter of an indicative conditional is tied to an earlier epistemic perspective. This does indeed seem to be going on in examples like the following:
\begin{prop}
	\nitem
	\begin{prop}
		\aitem \label{hadhad}
		If Oswald hadn't killed Kennedy, someone else had 
		\aitem \label{hadwould} 
		If Oswald hadn't killed Kennedy, someone else would 
		\aitem \label{didwould} 
		If Oswald didn't kill Kennedy, someone else would %was going to succeed in doing so
		\aitem \label{waswas} 
		If Oswald wasn't killing Kennedy, someone else was
		%If it wasn't Oswald killing Kennedy, it was someone else
	\end{prop}
\end{prop}
Just like ‘It was raining’ and ‘Einstein was going to win a Nobel Prize’, these are sentences that could not be felicitously asserted out of the blue: some particular past time has to be salient, and the proposition asserted is in some way about that time. (The required salience could be achieved either by earlier discourse or by nonlinguistic clues.) In the case of \ref{hadhad}--\ref{waswas}, the most natural reading is one on which one of the roles of this salient past time is that of providing the relevant epistemic perspective. None of these sentences commits the speaker to accepting ‘It \emph{is} possible that Oswald didn't kill Kennedy’, or to regarding the proposition that Oswald didn't kill Kenedy as a live possibility: but they do seem to commit us to a claim about what \emph{was} epistemically possible at the time in question. Note that one natural use for sentences like \ref{hadhad}--\ref{waswas} is in indirect speech reports of present-tense speeches made at the relevant time, namely
\begin{prop}
	\nitem
		\begin{prop}
			\aitem \label{hadhad'} 
			If Oswald hasn't killed Kennedy, someone else has 
			\aitem \label{hadwould'} 
			If Oswald hasn't killed Kennedy, someone else will 
			\aitem \label{didwould'} 
			If Oswald doesn't kill Kennedy, someone else will %is going to succeed in doing so 
			\aitem \label{waswas'} 
			If Oswald isn't killing Kennedy, someone else is
		\end{prop}
\end{prop}
Given the account we have outlined, this is to be expected, since the very propositions that would be asserted by \ref{hadhad'}--\ref{waswas'} can be expressed at later times by \ref{hadhad}--\ref{waswas}.

Although there are some syntactic similarities between the likes of \ref{hadhad}--\ref{waswas} and \ref{cfoswald} (‘If Oswald hadn't killed Kennedy, someone else would have’), there is a significant semantic gulf between them. On its natural interpretation, \ref{cfoswald} does not commit us to its being a live possibility that Oswald didn't kill Kennedy, either from our own perspective or from any other perspective---intuitively, its meaning feels altogether more “worldly” by comparison with the “perspectival” feel of \ref{hadhad}--\ref{waswas}. This is what we have been getting at in our use of the labels ‘indicative’ and ‘counterfactual’, roughly in line with the philosophical tradition.  Note however that examples like \ref{hadhad}--\ref{waswas} are rather different from the paradigmatic examples of indicative conditional sentences, and would provide counterexamples to many generalisations that philosophers have been wont to make about indicative conditionals (e.g.~generalisations about how the probability or “assertability” of an indicative conditional relates to the corresponding conditional probability), or about the superficial linguistic form distinctive of indicatives and/or counterfactuals (for example, \ref{hadwould} shows that a ‘would’ in the main clause does not suffice for counterfactuality).

In fact, once we recognise the category of past-perspective indicative conditionals, we can see that there is no failsafe way to distinguish counterfactuals from indicatives on the basis of superficial syntax, since many sentences admit of both kinds of interpretation, including the paradigm counterfactual \ref{cfoswald}:
\begin{prop}
	\sitem[\ref*{cfoswald}]
	If Oswald hadn't killed Kennedy, someone else would have 
\end{prop}
Although the dominant reading of this sentence is certainly counterfactual, it also has a past-perspective indicative meaning, which comes to the fore when we imagine using \ref{cfoswald} in reporting a past utterance of the following somewhat unusual but perfectly intelligible sentence:
\begin{prop}
	\nitem %\sitem[\chisholm{cfoswald}{'}]
	If Oswald hasn't killed Kennedy, someone else will have 
\end{prop}
And once this epistemic interpretation has been noticed, one can imagine contexts in which it is the intended interpretation even outside indirect speech reports.%
\footnote{\citet[p.\ 15]{KhooISC} also argues that \ref{cfoswald} has an “epistemic” reading.}

There are, however, some conditionals for which a counterfactual meaning is grammatically required. The most prominent examples are ones in which the main verb in the ‘if’-clause takes a subjunctive form:
\begin{prop}
	\nitem 
		\begin{prop}
			\aitem
			If Gore were president, he would deal with this problem
			\aitem 
			If I were to resign tomorrow, I would be hired by Google the day after tomorrow.
		\end{prop}
\end{prop}
Putting certain present-tense verbs in the main clause also rules out indicative readings:
\begin{prop}
	\nitem
	\begin{prop}
		\aitem
		If I resigned tomorrow, I reckon I would be hired by Google the day after tomorrow.
		\aitem 
		If I got a tattoo, it is unlikely that anyone would notice.
	\end{prop}
\end{prop}
%We won't try to give a theory of what's going on in these sentences yet, but 
It is plausible that the unavailability of indicative readings can be traced to the same source as the unacceptability of sentences like ‘It was possible that I am happy’.

There are also cases where a counterfactual reading seems to be required on semantic grounds:
\begin{prop}
	\nitem 
	\begin{prop}
		\aitem\label{married}
		If I had married someone other than the person I did marry, I would not be happy.
		\anitem \label{giraffes}
		If giraffes were any taller than they actually are, they wouldn't be able to pump blood up to their brains
	\end{prop}
\end{prop}
In these cases, it is plausible that what forces the counterfactual reading is the fact that the antecedent isn't live from any reasonable perspective.%
\footnote{When \ref{married} and \ref{giraffes} are embedded they need not always be counterfactuals: for example, ‘He said that if I had married someone other than the person I did marry, I would not be happy’ could be a felicitous report of a past utterance of the ‘If he has married someone other than Jessica, he will not be happy’.}

\citet{IatridouGIC} argues compellingly that what is distinctive of counterfactual uses of conditionals (in a range of different languages) is what she calls ‘fake past tense’: an extra layer of past tense morphology that does not carry the usual significance of temporal pastness.  Our observation that canonical counterfactuals like \ref{cfoswald} can also be interpreted as past-perspective indicatives supports Iatridou's claim.  When the sentences are used as indicatives, their past tense morphology manifestly \emph{does} have its usual temporal meaning, with the result that these uses are felicitous only when there is a salient past time for them to refer to; by contrast, when understood as counterfactuals they are perfectly fine out of the blue.  

As Iatridou notes, apparently non-temporal uses of past-tense morphology also crop up in certain other environments, as in the following examples.
\begin{prop}
	\nitem 
	\begin{prop}
		\aitem
		I wish they were here right now
		\aitem 
		Oh, that they had been here right now!
		\aitem \label{mighthave}
		They might have been here right now.
		\aitem \label{couldhave}
		They couldn't have been here right now.
	\end{prop}
\end{prop}
‘Might have been’ and ‘could have been’ also have epistemic uses where the past tense is non-fake---as noted earlier, ‘It might/could have been raining’ can mean ‘It is possible that it has been raining’ and ‘It was possible (then) that it was raining (then)’. But the use of ‘right now’ in \ref{mighthave} and \ref{couldhave} seems to preclude these readings, presumably for reasons similar to those that make for the badness of ‘It is possible that they have been here right now’ and ‘It was possible that they are here right now’.

% \begin{itemize}
% 	\item
% 	Find some cross-linguistic examples, and discuss Iatridou's cross-linguistic evidence.
% \end{itemize}

In many paradigmatically counterfactual conditionals, the ‘if’-clause involves pluperfect morphology, which can be thought of as involving two “layers” of past tense. In an ordinary use of the pluperfect both layers play a temporal role: ‘I had eaten breakfast’ takes us to a past “reference time” and then places an eating event earlier than that time. However, as Iatridou notices, there is no similar sense of double pastness in the natural counterfactual use of ‘If I had eaten breakfast this morning, I would have skipped lunch’. Her view of these cases is that at least one layer of past tense is fake past, while at most one is the usual temporal past. (In ‘If it had been raining right now, the ground would have been wet’, both layers seem to be fake.)

Why should the past tense be ambiguous in this way? An intriguing but elusive idea of Iatridou's is that the past tense has a more skeletal core meaning of “distance”, which can be cashed out either temporally or modally.  In our framework, the relevant thought in the modal case would presumably be one to the effect that non-live possibilities are “distant”, so that invoking a notion of accessibility that extends to the non-live requires reaching out to the distant. Obviously this picture raises many questions---for example, why do actual \emph{future} events not count as “distant” and thus apt to be described by verbs in the past tense? But we won't try to address such questions here: as with many other facts about linguistic structure, recognising that the phenomenon of fake past tense exists does not require having an explanation of why language is so configured.

We have posited a interpretative link between indicative conditionals and epistemic modals: barring context-shift, they invoke the same accessibility relation. Given that fake past tense can occur with modals, as in \ref{mighthave} and \ref{couldhave}, it is plausible that these modals stand in an analogous relation to counterfactual conditionals. If so, the non-epistemic reading of ‘It couldn't have been that $P$ and $Q$’ will (holding context fixed) entail ‘If it had been that $P$, it would have been that not $Q$’. Also, ‘It could have been that $P$’ and ‘If it had been that $P$, it would have been be that $Q$’ will jointly entail ‘It could have been that $Q$’.%; and because of the presupposition of non-vacuity, ‘If it had been that $P$, it would have been that $Q$’ will be presuppositionally bad unless ‘It could have been that $P$’ is true. 
These claims seem no less plausible than the corresponding claims about indicatives.

There is a competing picture of the role of the past-tense morphology in counterfactual conditionals which rejects the idea of ‘fake past’, and instead regards the relevant uses of the past tense as introducing reference to genuine past times (just as it uncontroversially does in past-perspective indicative conditionals like \ref{hadhad}--\ref{waswas}). As developed, e.g., by \citet{CondoravdiTIM} and \citet{KhooISC}, the idea is that just we can talk about a possibility being epistemically live or not at a given time, we can talk about a possibility being “metaphysically live” or not at a given time, where the metaphysically live possibilities at a time are (roughly) those which share their history up to that time. The picture is that when we utter a counterfactual, there is a particular time earlier than the time of utterance such that the accessible worlds are all and only those that are metaphysically live at that time, and at least one of the past-tense morphemes in the conditional refers to that past time.

There are several problems with this view. First, it will have trouble explaining why standard counterfactuals can be acceptable even when no particular past time has been raised to salience as a target to be referred to by the relevant morphemes---by contrast, ‘It was raining’ clearly does require such salience to be in place, and so do past-perspective indicatives like \ref{hadhad}--\ref{waswas}.%
\footnote{Note that in \ref{hadwould} and the past-perspective indicative reading of \ref{cfoswald}, the time of the epistemic perspective is most naturally taken to be a time after the (actual-world) killing of Kennedy. This makes it implausible that the counterfactual reading of \ref{cfoswald} is just the same thing with “metaphysical” rather than epistemic accessibility, since the worlds metaphysically accessible at these post-killing times are all worlds where Oswald did kill Kennedy.}
Second, consider counterfactuals about the future like ‘If I resigned next week I would be hired by Google the week after’. Even granting that the set of accessible worlds consists in all those whose history matches that of the actual world up to some given time, there is little reason to think that the time in question is in the past: it is much more plausible that the time of divergence is identical to or after the time of utterance, given that in evaluating such counterfactuals, we freely draw on known truths about the history of the actual world right up to the time of utterance. Third, it is just not plausible the notion of accessibility relevant to the evaluation of counterfactuals always requires matching any part of the history of the actual world: consider ‘If gravity had obeyed an inverse cube law, stars would have been unstable’, ‘If there had always been infinitely many stars, then there would always have been infinitely many planets’, etc. Fourth, it is very hard to see what reference to a past time could be going on in ‘I wish he were here right now’; but once we admit that the past tense is fake here, what reason is there to think it is genuinely temporal in ‘If he were here right now, then he would be happy’ (especially considering that the word ‘were’ can do double duty in ‘If, as I wish, he were here right now, he would be happy’).%
%\footnote{Condoravdi and Khoo argue for their view on the grounds that it explains why a present-present conditional like ‘If it's raining the ground is wet’ can only have epistemic readings.  Why we are not impressed by this.}
%  You might think that the view was getting you some kind of economy by letting you treat ‘had been’ as functioning in the same way in conditionals as elsewhere, but it's not at all.

% \begin{itemize}
% 	\item
% 	Present tense perspective-shifted uses
% 	\item
% 	Note somewhere or other: you can quantify into the accessibility parameter
% \end{itemize}

\section{The presupposition of non-vacuity}
\label{sect:nonvacuity}
An obvious worry about the proposal that the interpretation of conditionals typically involves a non-trivial accessibility relation is that, when combined with the view that a conditional is true when its antecedent is true in no accessible world, it seems to predict that conditionals whose consequents are contradictory or otherwise bizarre, or whose consequents contradict their antecedents, will often be (vacuously) true.  This may seem problematic, given that such conditionals are almost always unacceptable, whether indicative or counterfactual.  There are a few exceptions---notably conditionals like ‘If $P$, I'm a monkey's uncle’ and ‘If $P$, I'll eat hat’---but they are rare in a way that cries out for explanation.%
\footnote{Some exceptions: conditionals with contradictory consequents may occasionally be uttered in the course of giving formal proofs of the falsehood or impossibility of $P$.  Also, counterfactuals whose consequents and antecedents are both contradictory (like ‘If everything had been the case, it would have been that $Q$ and not $Q$’) are sometimes acceptable to assert even when one is not making an argument.}

We propose an explanation in terms of \emph{semantic presupposition}: a conditional interpreted with a particular accessibility parameter has a presupposition that its antecedent is true in some accessible world.  ‘If $P$, $Q$’ thus \emph{presupposes} what ‘Not (If $P$, $Q$ and not $Q$)’ semantically expresses.  Also, assuming the accessibility parameter for a modal ‘might’ or ‘could have’ is interpreted uniformly with the conditional, ‘If $P$, $Q$’ presupposes what ‘It might be that $P$’ or ‘It could have been that $P$’ expresses.  Conditionals whose consequents are contradictory, or which contradict their antecedents, are thus guaranteed to have false presuppositions whenever they are not themselves false: this explains why they are bad to assert.

While we expect speakers to strive to avoid uttering sentences with false presuppositions just as they strive to avoid uttering sentences which express false propositions, presuppositions are distinctive in that they lie “in the background” rather than being “at issue”: in uttering a sentence that has $p$ as a semantic presupposition, one signals that $p$ is not just true, but something appropriately \emph{taken for granted}.  In stereotypical cases, what makes this appropriate is that $p$ is something ones interlocutors are \emph{already} taking for granted; when they are not, but they are trusting (or careless), they will often begin doing so upon receiving the signal (the process of “presupposition accommodation”).  This makes presupposition a particularly helpful tool for driving context-shift (non-uniformity among different occurrences of context-sensitive expressions).  For example, if we have been working with a relatively narrow notion of accessibility and I suddenly say something like ‘If Oswald didn't kill Kennedy, someone else did’, listeners are less likely to conclude that I, surprisingly, regard Oswald's not having killed Kennedy as a live possibility in the demanding sense (since even if I held this view I would be unlikely to assume the listeners' willingness to take it for granted), and more likely to instead conclude that the intended interpretation of my speech involves a new and broader notion of accessibility; by contrast, if I come right out and say ‘Oswald might not have killed Kennedy’, I am more likely to be interpreted as making the controversial claim that this is true in some world accessible in the old sense.  

(The notion of presupposition is often modelled in a trivalent framework, where having a false presupposition is equated with being neither true nor false.  In this framework, positing the presupposition of nonvacuity would require some revisions to the statement of our schematic analysis of conditionals from the Introduction, and a rethinking of a fair amount of what we will be saying about the logic of conditionals.  But we will not be using the trivalent system, which raises many foundational difficulties we would prefer not to have to deal with.  Rather, we will treat presupposing and expressing as logically independent relations between sentences and propositions (assumed to be always true or false), so that there is a fourfold classification of sentences with respect to a given interpretation: true with only true presuppositions, true with false presuppositions, false with only true presuppositions, and false with false presuppositions.  Note that this leaves it open whether the presupposing relation can somehow be explained in terms of the expressing relation together with general pragmatic principles, or needs to be taken as an additional component of conventional meaning.  Although most of what we say could survive being transplanted into other ways of thinking of presupposition, we will leave this task to proponents of those frameworks.)

The presupposition of nonvacuity also helps explain why certain inferences which come out valid given our truth-conditions in fact seem odd or problematic.  One example is the inference from ‘It must be that not-$P$’ (or ‘It can't be that $P$’) to the indicative ‘If $P$, $Q$’, for arbitrary $Q$.   Similarly in the counterfactual case, ‘It couldn't have been that $P$’ will entail ‘If $P$ it would be that $Q$’ for any $Q$.  But arguing in this way seems intuitively bizarre.  We explain this by saying that if accessibility relation for the conditional is interpreted as the same one that matters for the modal, the premise entails that the presupposition of the conditional is violated; so the discourse puts pressure on us to invoke two different accessibility relations, in which case the inference will be simply invalid.

Finally, the presupposition of nonvacuity can also be motivated in a way that does not depend on any assumptions about the truth values of conditionals whose antecedents are true in no accessible world, by applying the standard tests for presupposition.  The primary such test looks for inferences which a sentence gives rise to both when asserted, when asked as a polar (yes-no) question, and when embedded under various negation-like operators such as ‘I doubt that’.  As far as we can tell, the inferences from the indicative ‘If $P$, $Q$’ to ‘It might/could be that $P$’/‘it's possible that $P$’/‘there's some chance that $P$’, and from the counterfactual ‘If it were that $P$ it would be that $Q$’ to ‘It could have been that $P$’ pass these tests quite well---consider the oddity of ‘I doubt that he's having fun if he's in the pub, and there's no chance that he's in the pub’, or the naturalness of responding to ‘Would you have agreed if I had bought you a yacht?’ with ‘Could you really have done that?’.  Another useful test \citep{FintelWYBI} is the ‘Hey wait a minute’ test: the inference from $P$ to $Q$ is presuppositional when it's natural to object to $P$ by saying something like ‘Hey wait a minute, who said that $Q$?’, indicating a refusal to go along with the suggestion that $Q$ should be \emph{taken for granted}.  Again, the inferences in question seem to pass the test: consider ‘If the Pope comes to my party I'll be delighted---Wait, is there is some chance that he will?’.  

One could instead explain the badness of ‘If $P$, $Q$ and not $Q$’ and the inference from ‘It can't be that $P$’ to ‘If $P$, $Q$’ by strengthening our truth-condition to make a conditional false when its antecedent is true in no accessible worlds.  On this view, ‘If $P$, $Q$ and not $Q$’ is in fact contradictory, and the inference-form is invalid.  But proponents of the stronger truth-condition still have prima facie reason to accept the presupposition of non-vacuity, stemming from the fact that the inferences survive embedding and the ‘Hey wait a minute’ test.  So making a positive argument for the stronger truth condition based on its capacity to explain these facts will require somehow undermining this prima facie case.  The chief disadvantage of the strong truth condition, meanwhile, is that it yields a much less attractive logic than a view on which vacuity makes for truth.  For example, \emph{Identity} (‘If $A$, $A$’) and \ref{mustif} (‘Must $Q$, so if $P$, $Q$’) are no longer valid.  While the explanatory work that we have done by appealing to these argument-schemas could probably be done by other means (e.g.\ appealing to statuses like Strawson-validity), the resulting explanations will be more complex and weaker.  Overall, the strong truth condition for conditionals is subject to many of the same objections as the view that ‘Every $F$ is $G$’ is false when ‘Nothing is an $F$’ is true; once one has admitted the status ‘true with a false presupposition’, the advantages that these views might seem to bring can be gained at less cost.

A competing account of the inference from ‘If $P$, $Q$’ to ‘Might $P$’ treats it neither as a semantic presupposition nor as an entailment, but as a conversational implicature.  The following Grice-style reconstruction of the inference looks initially promising:
\begin{prop}
	\item
	‘Must not-$P$’ is strictly stronger than ‘If $P$, $Q$’, and considerably simpler.  So, a co-operative speaker who was in a position to assert ‘Must not-$P$’ would do so, rather than just asserting ‘If $P$, $Q$’.   Thus when we hear someone assert ‘If $P$, $Q$’, we will tend to assume that they were not in a position to assert ‘Must not-$P$’.  But typically, if ‘Must not-$P$’ were true, we would expect that the speaker would be in a position to assert it.  So an utterance of ‘If $P$, $Q$’ will tend to give rise to the inference that ‘Must not-$P$’ is false, i.e.\ that ‘Might $P$’ is true.  
\end{prop}
The weakest point here is in the second last sentence.  In the case of epistemic modals, the view that when ‘Must not-$P$’ expresses a truth it is a truth that the speaker knows and is in a position to assert is not crazy, though in our view it is incorrect.  But the phenomena we are interested in apply to counterfactuals as well as indicatives, and for them the relevant competitor sentence is a non-epistemically modalized sentence along the lines of “It had to be that not-$P$” or “It couldn't have been that $P$”.  There is clearly no temptation to think that the propositions expressed by sentences of these forms are always known when true.  So, at least for counterfactuals, the most we could get out of Gricean reasoning is a weaker inference from the assertion of a conditional to the claim that it is not \emph{known} to be vacuously true, i.e.\ that it is compatible with the speaker's knowledge that there are some accessible worlds where the antecedent is true.%
\footnote{In the case of scalar implicatures, there seems to be some tendency to proceed as if there was a presumption that speakers are well-informed, so that speakers who are ignorant are expected to explicitly signal this fact.  For example, from ‘He ate some of the cake’ we will draw the inference that ‘He ate all of the cake’ is false, rather than merely not known to be true by the speaker.  However, these inferences are quite easily cancelable: ‘He ate some of the cake and I'm beginning to suspect that he ate all of it’ is fine, whereas ‘If it's raining the flight will be delayed and I'm beginning to suspect that if it's raining the flight won't be delayed’ is certainly not fine.}  
Moreover, conversational implicatures are characteristically \emph{cancelable}: while ‘He ate some of the cake’ implicates that ‘He ate all of the cake’ is false, ‘He ate some of the cake---in fact he ate all of it’ is perfectly fine.  Similarly, ‘He ate all of the cake; therefore, he ate some of it’ strikes us as good, not bad.  Conversational implicature is thus not up to the job of explaining why ‘If $P$, $Q$ and not $Q$’ sounds bad: given that the only way for the conjunction to be true is for both conjuncts to be vacuously true, one would expect the assertion of the conjunction to cancel any implicature of nonvacuity that would be carried by the assertion of either conjunct alone.  For the same reason, conversational implicature doesn't help explain why ‘It couldn't be that $P$, so if $P$, $Q$’ would seem odd.  

% \footnote{One could try to make trouble for this explanation by appealing to the fact that ordinary people may not make the relevant inference and hence may not speaking speaking have any view at all about whether vacuity holds; but we don't think this gets to the heart of the matter, since in semantics it is often to helpful to use somewhat technical vocabulary in one's representation of what language users tacitly know.}

\gap\begin{comment}
These phenomena could, however, be explained by ascribing conditionals a weaker semantic presupposition of lack of \emph{known} vacuous truth: ‘If $P$, $Q$’ presupposes the proposition that the speaker does not know that the proposition expressed by $P$ is true at every accessible world.  A weaker presupposition along these lines does provide a reasonable account of why conditionals with an obviously contradictory consequent seem bad to assert.  In asserting such a conditional one represents oneself one represents oneself as knowing it; but since it can obviously only be true by being vacuously true, this knowledge will easily yield knowledge of vacuity.%
\footnote{One could try to make trouble for this explanation by appealing to the fact that ordinary people may not make the relevant inference and hence may not speaking speaking have any view at all about whether vacuity holds; but we don't think this gets to the heart of the matter, since in semantics it is often to helpful to use somewhat technical vocabulary in one's representation of what language users tacitly know.}
Similarly, the weaker presupposition provides a reasonable account of the badness of the inference from ‘it can't be that $P$’ to ‘If $P$, $Q$’: in putting forth the premise one represents oneself as knowing it, which entails (holding fixed the accessibility relation) that the ignorance presupposition carried by ‘If $P$, $Q$’ is false.  However, the presupposition of lack-of-known-vacuity does not explain why there would be anything wrong with speeches like ‘If $P$, $Q$, and for all I know, if $P$, not $Q$’ or with reasoning like ‘It's beginning to look like it couldn't have been that $P$, so it's beginning to look like if it had been that $P$, it would have been that $Q$’.  

We can also consider examples which make it clear that conditionals really do give rise to inferences of non-vacuity, not merely lack of known vacuity.  For example, \ref{half} is clearly a terrible thing to assert unless one believes that the number of full-time people in the department is even:
\begin{prop}
	\nitem \label{half}
		If there had been exactly half as many full-time people in the department as there actually are, then we would have had to hire at least three adjuncts.
\end{prop}
This is readily explained by the presupposition of nonvacuity, since the antecedent is metaphysically impossible if the relevant number is odd. But so long as you don't know whether the number is odd or even, you don't know whether the conditional is vacuous, so the weaker epistemic presupposition does not account for the infelicity of \ref{half}. Along similar lines: suppose that you know one of Jack and Jill had three jacks and a queen and the other had a worthless hand with no queens. Both fold before the final draw.  You look at the top of the deck and see that each would have been dealt a queen. If the only relevant barrier were the weak presupposition, it's hard to see why the following wouldn't be a good thing to say in the light of your evidence:
\begin{prop}
	\nitem \label{queens}
		If Jill had kept playing with those four cards and had ended up with two queens, she would have won.
\end{prop}
But in fact \ref{queens} is bizarre. The presupposition of non-vacuity can help to explain this.  To interpret the conditional in such a way that it could legitimately be taken for granted that there are accessible worlds where the antecedent was true, we need a quite permissive notion of accessibility which does not hold fixed such facts as that a poker hand only contains five cards, or does not hold fixed the face values of the particular cards involved; but your evidence does not support the conditional on these interpretation.  (If Jill in fact had the worthless hand, then the closest six-card poker world where she has two queens in addition is probably a losing world.)

% \begin{itemize}
% 	\item
% 	\textbf{two more examples: (Setting: I'm not sure whether today or yesterday is the first day of the semester. I know Ted was in town until yesterday.) I say ‘If Ted had left on the first day of the semester, he would have been in the airport this morning’. ‘If he had won for the first time yesterday, he would barely have noticed the increase in his wealth’.}
% \end{itemize}

The above examples involved counterfactual conditionals. With indicative conditionals, it is much more delicate to distinguish the effects of the two candidate presuppositions. If we know which worlds are and are not epistemically possible, then whenever it is consistent with our knowledge that there is an epistemically possible world where the antecedent of a given conditional is true, we will know that there is such a world, so we will know what we need to know to satisfy the stronger presupposition. Thus to distinguish the proposals one must consider either cases where accessibility requires the application of some further, epistemically non-transparent constraint, or else cases where what's epistemically possible is not transparent. We will be considering the former phenemenon in \autoref{chap:prob}.  The most straightforward examples of the latter phenomenon involve conditionals like \ref{hadhad}--\ref{waswas} from \autoref{sect:indcf}, where the relevant epistemic perspective is in the past. Consider for example
\begin{prop}
	\sitem[\ref{hadhad}] 
	If Oswald hadn't killed Kennedy, someone else had 
\end{prop}
Consider a scenario where (we know) we aren't sure which of two scenarios obtain. In scenario A, the relevant past people knew that someone or other had killed Kennedy, and were unsure whether or not it was Oswald. In scenario B, they knew Oswald had killed Kennedy, and this was their entire basis for thinking that anyone had. Assert \ref{hadhad} while both of these possibilities are open would be quite odd.  But if the only relevant presuppositional requirement were one of lack of \emph{known} vacuity, there should be no problem asserting \ref{hadhad} on an interpretation where accessibility is just consistency with what the relevant people knew at the relevant time: on this interpretation, \ref{hadhad} is non-vacuously true in scenario A and vacuously true in scenario B, so we are in a position to know that it is true, and that we don't know that it is vacuously true.  By contrast, the view that there is a presupposition of nonvacuity predicts infelicity in this case, since according to it, the presupposition of \ref{hadhad} on this interpretation is something we do not know to be true, and are in no position to take for granted.  Given our ignorance about what was \emph{known} at the past time, the presupposition would send us looking for a wider interpretation of accessibility on which it would be appropriate to take for granted that some worlds where Oswald didn't kill Kennedy are accessible.%
%\footnote{*** A view where we are somehow allowed to substitute lack of known vacuity at the past time for lack of vacuity known now would be even worse, allowing lots of material-conditional-like interpretations.}
,%
%\footnote{??? Saying ‘we are playing a game where you are only allowed to say something if it was known at the time’ is no good, obviously.}
\end{comment}

As we noted at the beginning of this section, there are some cases where it seems acceptable to utter a conditional (with a particular setting of the accessibility parameter) even though the proposition that the antecedent is true in some accessible world is not known, or even known to be false.  ‘Monkey's uncle’ conditionals are one example; indicatives with absurd consequents uttered as a prelude to an argument by \emph{modus tollens} may be another.%
\footnote{This is not inevitable: we might instead invoke involve a wide domain of accessible worlds that includes impossible worlds at the absurd consequents are true (see chapter 4).}
This would be problematic for the thesis that conditionals presuppose non-vacuity if we thought that uttering a sentence semantically presupposing a certain proposition was robustly associated with commitment on the speaker's part (in the same way as uttering a sentence semantically \emph{expressing} a proposition).  It is often said that presuppositions, or at least certain kinds of presuppositions, are robust in this way, on the basis of the oddity of speeches like ‘The king of France is bald and there is no king of France’ or ‘He stopped smoking but I don't mean to suggest that he used to smoke’.  But the presuppositions in these cases are also entailed by the proposition expressed, so the oddity of these sentences is no surprise---they are contradictory.  Once we turn to presuppositions that are \emph{not} also entailments, we see a spectrum of robustness.  In some cases, the inferences are quite robust, and attempts at cancellation are pretty befuddling to ordinary speakers: consider ‘Every burglar who entered the White House was sent by the FBI, because there weren't any burglars in the White House’.  In other cases, the tendency for speakers to assume the truth of the standardly-predicted presupposition seems to disappear altogether in certain contexts, e.g.\ when certain questions are made salient---consider ‘He doesn't \emph{know} it's raining!  Remember that he's relying on those notoriously unreliable instruments’.  Perhaps the lesson to draw from this variety is that the category of “presupposition” is something of a grab bag within which important distinctions need to be drawn.  In any case, the existence of this spectrum means that what we seem to find with conditionals---namely, a generally quite robust association with a few circumscribed exceptions, some conventionally marked---is not especially problematic for, though also not explained by, the presuppositionality hypothesis.  

Asserting a conditional doesn't just tend to convey that the antecedent is true in some accessible worlds: it also tends to convey that the \emph{negation} of the antecedent is true in some accessible worlds.  Consider: ‘If the trains aren't running we can just take a taxi.  ---Hey wait a minute, I didn't know there was a chance that the trains weren't running!’  Should this kind of inference also be explained by positing a presupposition (which, in conjunction with the presupposition of nonvacuity, would amount to a presupposition of ‘antecedent diversity’, to the effect that the accessible worlds are not all alike with respect to the truth value of the antecedent)?  We are not sure.  The generalisation that hearers infer ‘Might not $P$’ from ‘If $P$, $Q$’ does seem to have many more exceptions that the generalisation that they infer ‘Might $P$’: for example, there is no temptation to draw such an inference when ‘If $P$, $Q$’ is uttered immediately after $P$ as a prelude or invitation to a \emph{Modus Ponens} inference (to $Q$).  (‘...So the butler did it; but if the butler did it, I'm innocent and deserve to be freed’).  A promising alternative way of explaining this inference is to invoke conversational implicature (“competition effects”) instead of presupposition, the idea being that if speakers are in a position to assert ‘Must $Q$’ we expect them to do so rather than asserting the weaker and more complex ‘If $P$, $Q$’.  As noted above, it is hard for this kind of reasoning to get beyond a weak conclusion to the effect that the speaker doesn't \emph{know} the proposition expressed by ‘Must $Q$’.  But when we look at cases where ignorance is a salient possibility, it looks like the weak inference may be all we get.  For example, it seems fine for a speaker who isn't sure whether Clark Kent is Superman to say ‘If Clark looked similar to Superman, lots of people would think he was Superman’, even though for all they know the antecedent is metaphysically necessary (and hence presumably true in all accessible worlds).  So, despite the aesthetic appeal of a view that gives the two inferences the same status, we will refrain from positing a semantic presupposition of antecedent diversity (though we will also not say anything that depends on there \emph{not} being such a presupposition).



% \begin{itemize}
% 	\item
% 	Other examples that are prima facie problematic for our presupposition but where we think that a proper delicacy about the strength of the thing is all you need:
%
%  (He knows where the keys are and is going to get them.) ‘Wherever he goes he'll open a drawer, because he knows that if it's in the kitchen it's in the kitchen drawer and if it's in the living room it's in the living room drawer’.
% 	\item
% 	 (Kratzer, ‘Chasing Hook’ argues that apparent counterexamples to contraposition are mere presuppositional effects, but this diagnosis does not apply when we're looking at confidence judgments.)
% 	\item
% 	mention monkey's uncle conditionals. {[}Puzzle - why don't we get them with counterfactuals? How do they relate to crazy conditional promises like ‘If \ldots{} I'll eat my hat’ / ‘If I'm wrong about this just shoot me’.
% 	\item
% 	Maybe come back later to other common presuppositions/implicatures: inhomogeneity, falsity of antecedent for counterfactuals.
% \end{itemize}



\section{Quasi-validity and materialism}
\label{sect:quasivalidity}
In this section, we will show how associating indicative conditionals with the same context-sensitive accessibility parameter that features in the truth-conditions of epistemic modals can help to explain some puzzling features of indicative conditionals which have often been used to motivate the view that such conditionals have the same truth conditions (in all contexts) as the corresponding material conditionals.  

Considering ‘arguments’ as consisting of a set of declarative sentences called ‘premises’ and another declarative sentence called a ‘conclusion’, we have been counting an argument as \emph{valid} just in case, on any uniform resolution of context-sensitivity, the propositions expressed by the premises entail the proposition expressed by the conclusion.  Let's say that an argument is \emph{quasi-valid} iff its \emph{premise-modalisation} is valid, where the premise-modalisation of an argument is the argument derived from it by prefixing ‘It must be that\ldots{}’ to each premise.  Quasi-valid arguments tend to feel intuitively like excellent deductive arguments, even when they are not valid. For example:
\begin{prop}
	\nitem \label{eithermust}
		Either this is a horse or it's a donkey

		It's not a horse.

		Therefore it must be a donkey.
\end{prop}
Even someone with an excellent logical training might naturally and unreflectively answer ‘yes’ when asked whether this is valid.  Nevertheless, it is merely quasi-valid, and not valid.  Clearly, if ‘It must be a donkey’ is in the business of expressing a proposition at all, there are possibilities where the propositions that it is eligible to express are false even though those expressed by ‘It is a donkey’ (on the same interpretation of the pronoun) is true.  And since being a donkey is incompatible with being a horse, any such possibility will be a counterexample to the validity of \ref{eithermust}.%
\footnote{Many authors have proposed that ‘must’ sentences do not express propositions, and extend the meaning of ‘valid’ in such a way that some arguments including non-proposition-expressing sentences, including \ref{eithermust}, count as ‘valid’.  We will have more to say about these views in chapter ***; for now, we just want to point out that the mere fact that these views enable one to say that arguments like \ref{eithermust} are ‘valid’ is not by itself a reason to accept them, since those who think that ‘must’ sentences express propositions can say the same thing just by identifying validity not with truth-preservingness but with the status we are calling quasi-validity, i.e.\ truth-preservingness of the argument got by prefixing the premises with ‘must’.}

A full account of epistemic modality needs to explain what's so good about merely quasi-valid arguments like \ref{eithermust}, and why this kind of goodness is so readily mistaken for validity.  Plausibly, the answers to these questions will involve fleshing out in some way or other the kinds of ideas sometimes discussed under the heading of ‘the knowledge norm of assertion’---the thought being that in uttering a declarative sentence assertively, one is in some sense committed not merely to the truth of the proposition semantically expressed, but to its being known, or meeting some contextually flexible evidential standard that's also picked up by words like ‘must’.  One might also want to make a connection to the idea \citep{StalnakerAssertion} that the characteristic function of an assertion is to add a proposition to the “common ground” of a conversation, and that the presence of the proposition expressed by $P$ in the common ground is at least a sufficient condition for the truth of ‘It must be that $P$’.%
\footnote{Cite: Mandelkern,…}  
But for present purposes it doesn't much matter how exactly we nail these ideas down.  
% (contrast? the result of replacing ‘must’ with ‘we ought to believe that’)

One prominent argument for the ‘materialist’ view of indicative conditionals, according to which the indicative ‘If $P$, $Q$’ and ‘Either not-$P$ or $Q$’ express necessarily equivalent propositions in all contexts, appeals to the seeming excellence of instances of the following argument-schema:
\begin{prop}
	\sitem[Or-to-if] \label{ortoif}
	\parbox[t]{\linewidth}{$P$ or $Q$\\
	Therefore if not-$P$, $Q$}
\end{prop}
As many authors have noted, instances of this form just “feel valid”: \citet{StalnakerIC} gives the example ‘Either the butler or the gardener did it.  Therefore if the butler didn't do it, the gardener did.’  But if we took this appearance at face value, it would be a short step to materialism.  (Given the logical equivalence of ‘not-not-$P$’ and $P$ and the principle of Substitution in the Antecedent, the validity of Or-to-if guarantees that of the schema ‘Not-$P$ or $Q$; therefore if $P$, $Q$’.  In the other direction, suppose that ‘if $P$, $Q$’ is true.   ‘Not-$P$ or $P$’ is true by the Law of Excluded Middle, so by Modus Ponens and Proof By Cases, ‘Not-$P$ or $Q$’ is also true.)  Thus if we want to deny the logical equivalence of ‘If $P$, $Q$’ and ‘Either not-$P$ or $Q$’, we had better deny the validity of \ref{ortoif}.  

Similar remarks apply to the following schema
\begin{prop}
	\sitem[Unconditional Agglomeration] \label{qsoifppandq}
	\parbox[t]{\linewidth}{$Q$\\
	So if $P$, $P$ and $Q$}
\end{prop}
Instances of this form---for example ‘He is an idiot; so if he is rich, he is a rich idiot’---have a similar feeling of validity to them.  But again, the view that the schema is really valid leads to materialism under minimal additional assumptions.  (Suppose that the material conditional ‘Either not-$P$ or $Q$’ is true; then by \ref{qsoifppandq}, ‘If $P$, $P$ and (either not-$P$ or $Q$)’ is true; so by \emph{Deduction in the Consequent}, ‘If $P$, $Q$’ is true too.  For the argument from ‘If $P$, $Q$’ to ‘Either not-$P$ or $Q$’, see the previous paragraph.)

However, given that quasi-valid arguments as well as valid ones can be expected to strike us as excellent, our view provides a ready rejoinder to these arguments for materialism. For it is part of the view the following argument-schema is valid:
\begin{prop}
	\sitem[Must-if] \label{mustif}
	\parbox[t]{\linewidth}{It must be that $Q$\\
	So if $P$, $Q$}
\end{prop}
Given \emph{Identity} and \emph{Deduction in the Consequent}, it follows that both of following argument-schemas are also valid:
\begin{prop}
	\sitem[Modalised Or-to-if] \label{modalortoif}
	\parbox[t]{\linewidth}{It must be that ($P$ or $Q$)\\
	So if not $P$, $Q$}
	\sitem[Modalised Unconditional Agglomeration] \label{modalthing}
	\parbox[t]{\linewidth}{It must be that $Q$\\
	So if $P$, $P$ and $Q$}
\end{prop}
Thus both \ref{ortoif} and \ref{qsoifppandq} are quasi-valid, and just as excellent as the intuitively excellent argument \ref{eithermust} above. We don't see any pre-theoretic pressure to think that the relevant arguments are any better than the likes of \ref{eithermust}.%
\footnote{Our view also provides the resources to describe a way in which \ref{ortoif} is better than \ref{qsoifppandq}.  Although \ref{modalthing} is valid, asserting its premise does not in any way imply that the \emph{presupposition} of its conclusion is true.  In this respect it is like ‘He doesn't regret anything, so he doesn't regret that he cut off his arm’ or ‘Everything in my garden is green, so every dog in my garden is green’.  By contrast, asserting a disjunction carries the implicature that both disjuncts are epistemically possible for the speaker (readily explicable in neo-Gricean fashion), and this presumably also holds for ‘It must be that ($P$ or $Q$)’.  So while the mere truth of the premise of \ref{modalortoif} does not guarantee the truth of the presupposition of its conclusion, the implicatures carried by an assertion of the premise do guarantee this, at least in a standard context where epistemic possibility for the speaker suffices for accessibility.}

Note that the validity of \ref{modalortoif} and \ref{modalthing} depends on our decision to let the conditional be true when there is no accessible world in which the antecedent is true. If we had instead gone for a view on which conditionals are false in this case, we would need to say something more complicated and less satisfying about the status of \ref{ortoif} and \ref{qsoifppandq}.%
\footnote{This variant of our view is also unappealing in other ways: most importantly, it entails that sentences of the form ‘If $P$, $P$’ are false in many contexts.}

One way to make a positive case against the materialist view on which the relevant arguments are in fact valid is to see what happens when premises and conclusions are embedded under operators like ‘For all I know’. When the argument from $P$ to $Q$ is really valid (and easily recognised as such), ‘For all I know $P$, so for all I know $Q$’ should sound like a good piece of reasoning. But speeches like ‘For all I know he is at the match, so for all I know if he had a car crash this morning he had a car crash and is at the match’ and ‘For all I know, it's true that she's either in Paris or on Mars (because for all I know she's in Paris); so for all I know, if she's not in Paris she's on Mars’ seem bad. This is good prima facie evidence that the felt goodness of the relevant inferences should be explained by some feature other than straightforward validity.

(A related argument for materialism appeals not to intuitions about the validity of \emph{arguments}, but to intuitions about the logical truth of single sentences of the form ‘If $P$ or $Q$, then if not-$P$, $Q$’. The notion of quasi-validity does not provide a response to this form of argument, since there is no corresponding notion of ‘quasi-logical-truth’ distinct from logical truth proper.  One can certainly allow for arguments with zero premises, and identify logical truth with validity of the corresponding zero-premise argument; but quasi-validity coincides with validity for zero-premise arguments, since the result of prefixing ‘must’ to every member of the empty set is still the empty set.  We think there is a good response to this argument too, but it requires some general points about conditionals which embed other conditionals which we will take up later.  
% in \autoref{chap:embedding}. 
For the present, just note that it would be difficult to endorse this as an argument for materialism unless one were also willing to endorse the apparent logical truth of ‘If this is either a horse or a donkey and it's not a horse, then it must be a donkey’ as an argument for the view that ‘It must be that $P$’ expresses a true proposition whenever $P$ does.)

It is worth comparing our response to the validity-theoretic arguments for materialism with a somewhat similar response offered by Robert Stalnaker.  Stalnaker's view can be formulated in a way that fits our basic accessibility-and-closeness template, and he agrees with us that a certain notion of epistemic possibility plays a distinctive role in the semantics of indicative conditionals.  (Stalnaker takes the worlds that are epistemically possible in the relevant sense to be those that belong to the ‘context set’ of the relevant conversation, i.e.\ are consistent with everything “commonly presupposed” by those involved.  For Stalnaker, however, the distinctive role that a notion of epistemic possibility plays the semantics for indicative conditionals consists not in the exclusion of epistemically impossible worlds from the domain of accessibility (as on our view), but in the fact that indicative conditionals impose a distinctive kind of closeness ordering (or equivalently, a distinctive kind of selection function).   Expressed in terms of closeness, his proposal is that whenever $w_1$ and $w_2$ are in the context set of a certain conversation (taking place at the actual world, which may be distinct from both $w_1$ and $w_2$), and $w_3$ is not in the context set of that conversation, $w_2$ is closer to $w_1$ than $w_3$ is according to the closeness relation operative in the relevant conversation.  There is no special connection between epistemic possibility in any sense and accessibility---in fact Stalnaker proposes in effect that every metaphysically possible world always counts as accessible, so that ‘If $P$, $Q$ and not $Q$’ is true in a context only if $P$ expresses a metaphysical impossibility in that context.  Thus, on Stalnaker's view, \ref{mustif} is \emph{not} valid, even if we stipulate an interpretation for ‘must’ on which the truth of ‘It must be that $P$’ requires $P$ to be true throughout the context set.%
\footnote{This is not a very plausible theory of ‘must’---it struggles with the fact that it's fine to say ‘It must be raining outside’ when the speaker can see people walking in with umbrellas, but the audience cannot see them.  Mandelkern (***) suggests a view where what matters for the truth of ‘It must be that $P$’ is that $P$ be true in all worlds in the “prospective” context set.}  
So long as the truth of ‘Must $P$’ does not require that $P$ is true in all accessible worlds (which it certainly does not if all metaphysically possible worlds are accessible), any case where ‘Must $P$’ is true while the closest not-$P$ world is a not-$Q$ world will be one where ‘Must ($P$ or $Q$)’ is true and ‘If not-$P$, $Q$’ is false.

Stalnaker thus cannot agree with us that \ref{ortoif} is quasi-valid.  Nevertheless, as his view does accord \ref{ortoif} the status of a “reasonable inference”, defined as one where ‘every context in which the premises could appropriately be asserted or explicitly supposed, and in which it is accepted, is a context which entails the proposition expressed by the corresponding conclusion’ \parencite{StalnakerIC}.  The crucial extra ingredient needed to derive this is the Gricean thought that asserting a disjunction ‘$P$ or $Q$’ carries the implicature that both not-$P$ and not-$Q$ are epistemically possible (which for Stalnaker requires that true at some worlds in the context-set).  Thus, any case where ‘$P$ or $Q$’ is both appropriately asserted and accepted will be one where ‘$P$ or $Q$’ is true at every world in the context-set, while both ‘not $P$’ and ‘not $Q$’ are true at some worlds in the context-set.  So in any such case, ‘if not-$P$, $Q$’ will be true at every world in the context set, since the closest $P$ world to each of these worlds is one of the ‘not-$P$’-worlds in the context set, all of which are $Q$-worlds.  
 
A similar implicature-based explanation of the positive status of \ref{ortoif} will be available to anyone who regards the following argument as valid:
\begin{prop}
	\sitem[Must-might-if] \label{mustmightif}
	Must $Q$
	
	Might $P$
	
	Therefore, if $P$, $Q$
\end{prop}
So long as this is valid, \ref{ortoif} will have whatever good status is shared both by the inference from $P$ to ‘Must $P$’ and by the inference from ‘$P$ or $Q$’ to ‘Might $P$ and might $Q$’.%
\footnote{Indeed, Stalnaker could agree that \ref{mustmightif} is valid if he held that ‘Must $P$’ requires $P$ to be true throughout the context set, that ‘Might $P$’ requires $P$ to be true somewhere in the context set, and that the actual world is always in the context set.  His actual view, however, is that the actual world need not be in the context set, so there is no obvious way for him to agree that \ref{mustmightif} as valid.  True, if he treated the truth of $P$ throughout the context set as a sufficient as well as necessary condition for the truth of ‘must $P$’, he would regard the variant argument form derived from \ref{mustmightif} by prefixing the conclusion with ‘must’ as valid; however, insofar as there are reasons to think that ‘must $P$’ implies $P$, this theory of ‘must’ especially unpromising given Stalnaker's overall view.
	
	Stalnaker could validate \ref{mustmightif} by adopting, instead of or in addition to the constraint on closeness already discussed, a constraint according to which all worlds in the context set have to be closer \emph{to the actual world} than any world not in the context set.  But he had a good reason for not adopting that constraint, namely that it is incompatible with the claim that no world is closer to any world than that world itself, without which there will be counterexamples to modus ponens.} 

Unfortunately, implicature-based accounts of the positive status of \ref{ortoif} will not carry over to our other good-seeming argument-form, \ref{qsoifppandq}.  On Stalnaker's view, the premise $Q$ could be true, and true at every world in the context set, even though the conclusion ‘If $P$, $P$ and $Q$’ is false.  This will happen when $P$ is true at no world in the context set and the closest $P$ world is a not-$Q$ world.  This is a major limitation in Stalnaker's diagnosis, since as we noted above, instances of \ref{qsoifppandq} feel intuitively on a par with instances of \ref{ortoif}.  

A different tack that Stalnaker could take with \ref{qsoifppandq} is to appeal to the presupposition of validity.  Typically, when we hear someone utter the conclusion ‘If $Q$, $P$ and $Q$’, we will start taking it for granted (if we weren't already taking it for granted) that $Q$ is an epistemic possibility---this process of “presupposition accommodation” is one of the characteristic communicative functions of presupposition-carrying sentences.  But given this presupposition and the epistemic necessity of the premise $P$, the conclusion must be true.  To make this precise, say that an argument is \emph{Strawson-valid} just in case on every uniform interpretation, the proposition expressed by its conclusion is entailed by the conjunction of all the propositions expressed by the premises, presupposed by the premises, or presupposed by the conclusion.%
\footnote{See \cite{vonFintelNLSECD}.}  
And say that an argument is \emph{Strawson-quasivalid} just in case its premise-modalisation is Strawson-valid.  The observation, then, is that since ‘If $Q$, $P$’ presupposes what ‘Might $Q$’ expresses, the validity of ‘Must $P$; might $P$; therefore if $Q$, $P$ and $Q$’ guarantees the Strawson-quasivalidity of \ref{qsoifppandq}.  

Unfortunately, this status does not seem to be enough to account for the intuitive goodness of instances of \ref{qsoifppandq}, since many arguments that are Strawson-valid (and hence a \emph{fortiori} Strawson-quasivalid) don't seem intuitively that great at all.  Consider for example:
\begin{prop}
	\nitem 
	No-one will be smoking in the year 2050; therefore Fred will have stopped smoking by 2050
	\nitem 
	John regrets everything he has ever done; therefore John regrets drinking petrol
	\nitem 
	Every creature in the room is purring; therefore, the elephant in the room is purring
\end{prop}
Given the standard view that the conclusions of these arguments presuppose, respectively, the propositions expressed by ‘Fred will have been a smoker before 2050’, ‘John killed his mother’, and ‘There is an elephant in the room’, these arguments are all Strawson-valid.  So \emph{prima facie}, our view does a better job than Stalnaker's at explaining away the appearances of validity that drive the appeal of materialism.

% \begin{itemize}
% 	\item
% 	\textbf{Note: check on Stalnaker's paper, also Moritz Shultz \& Robbie.}
% 	\item
% 	Maybe: give another argument why we don' t like Stalnaker's ‘demoting’ approach, namely that it makes ‘If P, actually P’ false in cases where not-P is epistemically necessary. Come back to this in chapter 3.
% 	\item
% 	Make sure not to forget to talk about Stalnaker's distinctive claims about ‘actually’.
% \end{itemize}

One might worry that in going as far as we are going to vindicate the validity-judgments that motivate materialism, we will also be saddling ourselves with some of the classic problems of materialism---specifically, the so-called “paradoxes of material implication”, arguments that are valid according to materialism but don't seem intuitively good at all.  (*cite*)  Here is the first of them:
\begin{prop}
	\sitem[First “Paradox”]
	\parbox[t]{\linewidth}{$Q$\\
	So if $P$, $Q$}
\end{prop}
On our view, this is quasi-valid.%
\footnote{Similarly, Stalnaker's view makes this argument form Strawson-quasi-valid.}
We admit that it would be odd and suspicious if someone actually produced such an argument in the course of any piece of reasoning. But we don't think this is a good reason to deny that such arguments are quasi-valid: for ‘This is a horse; so this must be a horse’ is uncontroversially quasi-valid but also seems quite bizarre. If someone were to utter this sequence of sentences, we would feel pragmatic pressure to interpret the ‘must’ in some way that would give the conclusion some conversational point, and this will require some interpretation that doesn't tie the ‘must’ in a flat-footed way to what has been established at that point in the conversation.%
\footnote{*cite Mandelkern on the need for a salient argument*} 
In general, when we are dealing with super-simple arguments, intuitions of excellence will be confounded by the fact that such arguments have so little conversational point that they send us looking for non-obvious interpretations under which they are more tendentious or informative. Notice once we turn to only slightly more complex forms like \ref{qsoifppandq}, which clearly stands or falls with the First Paradox, the intuitions of excellence start to fall into place.  (Or in a similar vein, consider: ‘John is in the room; so if Jill is in the room, two people are in the room’.)

The other classic objection to materialism turns on the following inference scheme:
\begin{prop}
	\sitem[Second “Paradox”]
	\parbox[t]{\linewidth}{Not $P$\\
	So if $P$, $Q$}
\end{prop}
Again, this is valid on the materialist view and quasi-valid on our view, and it sounds like a truly awful template for argumentation. But note that given the presupposition of non-vacuity, the argument ‘It must be that not-$P$, therefore if $P$, $Q$’ has the bad-making feature that if the premise is true, the presupposition of the conclusion is violated. In this respect it is similar to arguments like ‘She hasn't stolen from anyone; therefore she doesn't regret stealing from her employer’ (assuming a standard view about the presuppositional behaviour of ‘regret’); ‘No-one met anyone; therefore no-one met the King of France’ (assuming a standard view about the presuppositional behaviour of ‘the’); or ‘Everyone is having a good time; therefore everyone who isn't having a good time is a terrorist’ (assuming a somewhat more controversial view about the presuppositional behaviour of ‘everyone’). The latter seems especially apt as a model for arguments involving conditionals.  If someone actually produced such an argument, we would naturally reach for an interpretation on which the domain of the quantifier in the conclusion is wider than the domain of the quantifier in the premise, since there is clearly a strong tendency to interpret the quantifier in ‘every $F$ is $G$’ in such a way that the truth of ‘something is $F$’ can be taken for granted.%
\footnote{There has been considerable disagreement as regards whether this effect is properly explained by saying that ‘Every $F$ is $G$’ \emph{presupposes} the falsity of ‘Nothing is an $F$’.  For an overview see \cite[§6.8.2]{HeimKratzerSGG}.  The most widely cited opponents of the presuppositional view, \citet{LappinReinhartPESD}, are primarily motivated by the combination of orthodox logical views with which we agree---such as that ‘Every $F$ is $F$’ is valid---with a trivalent theory of presupposition (equating presupposition-failure with being neither true nor false); given this theory, the presuppositional view requires denying that all sentences of the form ‘Every $F$ is $F$’ are true, which seems to conflict with the claim that such sentences are valid.  This argument against the presuppositional theory of ‘every’ has no force in the non-trivalent framework we are working in.}
And if we do widen the domain of quantification in this way, the proposition expressed by the conclusion will not be entailed by the proposition expressed by the premise.  This seems like a reasonable way of vindication of our intuitive sense that there's something wrong with the Second Paradox as an argument-template, and driving a wedge between this argument and the intuitively good \ref{ortoif} and \ref{qsoifppandq}.%
\footnote{The fact that intuitions of validity are sensitive to presupposition-theoretic facts in this way might suggest that what these intuitions generally track is either Strawson-validity (explained earlier) or “presuppositional validity”, where an argument is presuppositionally valid just in case on every uniform interpretation, the conjunction of all the propositions expressed or presupposed by the premises entails the conjunction of the propositions expressed or presupposed by the conclusion.  (In the framework where presupposition-failure is identified with truth-valuelessness, this corresponds to the conclusion being true at every world where all the premises are true.)  We are skeptical: the argument ‘She doesn't regret stealing from her mother; therefore she stole from someone’ seems fishy despite being presuppositionally valid.}  

%is a very common way for presuppositions to operate is how presupposition typically operates as a way of allowing an audience to recognise lack of uniformity across time in the interpretation of context-sensitive elements. 

The idea that sometimes intuitively excellent arguments are quasi-valid rather than valid can also be used to undercut a certain way of arging for “strictism”, the view that the indicative conditional ‘If $P$, $Q$’ has the same truth conditions as the so-called “strict conditional” ‘It must be that (either not-$P$ or $Q$)’.  Someone might argue for strictism on the grounds that the argument ‘If $P$, $Q$; therefore it can't be that $P$ and not-$Q$’ seems excellent in a way that requires treating it as valid---in which case the argument ‘If $P$, $Q$; therefore it must be that (not-$P$ or $Q$)’ is surely also valid.  Our view makes these arguments merely quasi-valid: since ‘If $P$, $Q$’ logically entails ‘Not-$P$ or $Q$’, ‘It must be that (if $P$, $Q$)’ logically entails ‘It must be that (not-$P$ or $Q$)’.

(Some other argument-forms that are valid according to materialists and strictists, but merely quasi-valid on our account (for present-perspective indicative conditionals), are: contraposition (‘if $P$, $Q$; therefore if not-$Q$, not-$P$’), transitivity (‘If $P$, $Q$; if $Q$, $R$; therefore if $P$, $R$’), and antecedent strengthening (‘If $P$, $R$; therefore if $P$ and $Q$, $R$’).)

\section{How common are material-like readings?}
\label{sect:antimaterial}
We have said why we are not convinced by the central argument for materialism.  However, the foregoing discussion also suggests that the task of arguing \emph{against} materialism is going to be somewhat delicate.  Given that opponents of materialism deny that the indicative conditionals are entailed by the corresponding material conditionals, one might prima facie expect that they would be willing to provide counterexamples to this entailment, where providing a counterexample would involve making a speech of the form ‘$P ⊃ Q$, but it is not the case that if $P$, $Q$’.  But our account suggests that speeches like this will generally be unacceptable.  Since \ref{mustif} is valid, so is the inference from ‘Not (if $P$, $Q$)’ to ‘Not must (not-$P$ or $Q$)’, i.e.\ ‘It might be that $P$ and not $Q$’.  So the counterexample-offering sentence validly entails something of the form ‘$P$ but it might be the case that not-$P$’.  And sentences of this form are generally quite odd.  (The task of explaining why they are odd is of a piece with the task of explaining what's good about the inference from $P$ to ‘must $P$’, and about quasi-valid inferences in general.  A first-pass explanation appeals to the knowledge norm of assertion: in asserting $P$ one represents oneself as knowing that $P$, while ‘might $P$’ entails that one does not know that $P$, so the overall impact of the speech act is incoherent  A more sophisticated explanation will have to take account of the context-sensitivity of ‘must’ and ‘might’: in some way we will not here attempt to fully work out, resolving the context sensitivity of modals in a way that makes it hard for ‘must $P$’ to be true also raises the bar for a flat-out assertion of $P$.)

We don't want to overstate the problem posed by the relevant kind of oddity.  Speeches of the form ‘$P$ and if not $P$, $Q$’ are sometimes fine.%
\footnote{It is interesting that such sentences don't sound as bad as ‘$P$ and might not-$P$’, e.g.\ ‘Oswald shot Kennedy but he may not have’.  We suggest that this is because, although both sentences require something akin to non-uniform interpretation---a transition from the relatively demanding notion of epistemic possibility implicitly invoked by the first conjunct to the relatively lax notion explicitly invoked by the second conjunct---this kind of transition is more natural when it is driven by presuppositions than when it is required by expressed content.  The fact that presuppositions are in the background, and are signaled as things to be taken for granted rather than up for debate, makes them especially useful as a way of guiding hearers towards non-uniform interpretations: contrast ‘Everyone came to my party, and everyone who wasn't at the party but read about it in the newspapers was jealous’ with ‘Everyone came to my party, and some people weren't at the party but read about it in the newspapers’.} %***MORE EXAMPLES***}
And one could perhaps in the same mood accept speeches of the form ‘$P$ and not (if not $P$ then $Q$)’---for example, ‘Oswald shot Kennedy, but it's not the case that if he didn't, no-one did’.
But we still wouldn't want to lean too heavily on arguments against materialism based on such premises.  Materialists might reasonably respond that we are especially prone to error in deploying the stilted locution ‘it is not the case that’---note for instance that ‘It's not the case that every unicorn that he owns is coming to the party, since he doesn't own any unicorns’ sounds initially fairly appealing.%
\footnote{Arguments against materialism that turn on conditionals embedded under quantifiers provide one way around the problem of assertability.  For example, one might object that materialism entails that in a setting where exactly one of several candidates won the election, ‘Some candidate lost if she won’ is true, whereas in fact no candidate lost if she won.  However, as we will be discussing in chapter \autoref{chap:embedding}, materialists have some defensive resources involving domain restriction.  More generally, sentences in which conditionals are embedded under quantifiers turn out to be challenging not just for materialism but for many competing views, so the dialectic concerning them will have to take the form of a careful comparative investigation.}  

In our own thinking about materialism, we have been particularly moved by a well-known argument having to do with degrees of confidence.  Consider:
\begin{prop}
	\nitem \label{moderatelyconf}
	I'm moderately confident that if they did go out they went to the movies, but I am even more confident that they didn't go out.  
\end{prop}
The psychological state reported by this sentence is one that could not be maintained by someone who was certain of the material conditional theory, and formed credences about propositions expressed by indicative conditionals in accordance with the dictates of that theory.  This pattern of credence formation involves never being less confident in an indicative conditional than in the negation of its antecedent, since according to the material conditional theory, the conditional is entailed by the negation of its antecedent.  But it seems obviously fine to be the kind of person of whom \ref{moderatelyconf} is true.  Similarly, it seems paradigmatically fine to be the sort of person that is 50\% confident that if a certain fair coin was tossed it came up tails, but less than 100\% confident that it was tossed, and thus less than 50\% confident that it was tossed and came up tails.  But again, since according to the material conditional theory the conditional is entailed by its consequent, forming credences in accordance with that theory would involve never being more confident in a conditional than in its consequent.  

The point here is not just about sentences in which indicative conditionals occur embedded under ‘confident that’.  For one thing, it arises with many different kinds of embeddings, including those whose connection to the concept of confidence is not so straightforward.  For example, suppose that we know that Smith was playing poker, and we now learn that his opponent is Fred, a mediocre but rich player who generally bets big.  The following speech is natural in this setting:
\begin{prop}
	\nitem \label{startingtolook}
	It's starting to look like Smith won big if he won.  
\end{prop}
But given Fred's mediocrity, the evidence that he was Smith's opponent may actually be evidence \emph{against} the corresponding material conditional (\emph{Smith either lost or won big}), in virtue of being evidence against the proposition that Smith lost; the acceptability of \ref{startingtolook} is thus mysterious from the materialist point of view.  

The pattern of materialism-unfriendly confidence judgments also shows up in cases where, instead of having a conditional embedded under an operator like ‘confident’, we have anaphoric reference back to the proposition expressed by a bare conditional earlier in the discourse.  Consider:
\begin{prop}
	\nitem \label{anaphora}
	If they went out they went to the movies.  
	
	--- You're probably right / I'm pretty confident of that / There's a good chance that they did / That's fairly likely to be true / … %I'm moderately inclined to agree with what you just said…  
\end{prop}
It is easy to spell out the scenario in such a way that these replies are all in order, but where the second speaker is extremely confident that they didn't go out at all, and so is clearly not forming credences in accordance with the dictates of the material conditional theory.%
\footnote{These anaphoric cases are important, since some authors (***cite Kratzer, Rothschild, …) have proposed accounting for the behaviour of conditionals under ‘confident’ and other such operators by adopting a logical form for the ‘confident’ sentences which does not have the logical form of the bare conditional sentence as a constituent, and correspondingly denying that the truth-conditions of the ‘confident’ sentences involve the relevant subject's relations to the propositions that would be expressed by utternces of a bare conditional.  \citet{vonFintelIf, vonFintelConditionals} also notes the failure of such accounts to extend to examples like \ref{anaphora}.}  

Similar points can be made about desire-like attitudes.  Consider hope:
\begin{prop}
	\nitem \label{hopeblue}
	I'm hoping that I've won a blue car if I've won a car
\end{prop}
On the material conditional account, the proposition expressed by ‘I've won a blue car if I've won a car’ is equivalent to the one expressed by ‘Either I've won a blue car or I haven't won a car’.  But on the natural way of fleshing out the scenario, \ref{hopea} would seem to be false:
\begin{prop}
	\nitem \label{hopea}
	I'm hoping that I have either won a blue car or haven't won a car	
\end{prop}
We can sharpen the point by assuming it's part of the speaker's background knowledge that the cars on offer are red and blue.  Then, given the plausible assumption that hope is closed under substitution of propositions that are equivalent relative to such background knowledge, \ref{hopeb} stands or falls with \ref{hopea}:
\begin{prop}
	\nitem \label{hopeb}
	I'm hoping that I didn't win a red car
\end{prop}
Forming pro-attitudes in accordance with the material conditional theory would thus involve patterns of hopes (and presumably other pro-attitudes) that are quite out of step with the patterns we actually seem to see, even amongst apparently rational people.  

Of course, it is open to materialists to admit that people do in fact form credences and pro-attitudes in the ways that we have described, but to put this down to their blindness to the truth of materialism.  Perhaps, once one has become enlightened, one should simply revise one's credences (and hopes, etc), so that there will no longer be cases where one is less confident in a conditional than in its consequent or the negation of its antecedent.%
\footnote{It is an open question whether the update involved in such enlightenment is appropriately modeled using standard conditionalisation.}  
This need not involve saying that ordinary folk are \emph{irrational}, since failure to know the truth of a true theory, even in philosophy, need not be a failure of rationality.  Nevertheless, there is generally a strong presumption against the truth of semantic theories that would have to induce large-scale revision of ordinary patterns of usage if taken to heart.  Consider, by analogy, the theory that ‘know’ expresses the relation of truly believing: if taken to heart, this would recommend a substantial overhaul of a central pattern of judgments that have guided users of ‘know’ and its cognates since antiquity.  This looks like a powerful objection to the view.  Proponents will say things like ‘My view is more simple, and we already know that ordinary people make mistakes all the time’, but this does not seem adequate to the force of the objection.

The attitude-based objection to materialism would of course have little force if it turned out that no theory could underwrite the relevant patterns of attitudes. Indeed, there is a body of literature (going back to \cite{LewisPCCP} and \cite{StalnakerLVF}) that attempts to identify the relevant pattern, and then argue that no way of associating propositions with conditionals can vindicate that pattern.  We will discuss this at length in \autoref{chap:prob}.  For now, we just want to remark, first, that some of the literature in question assumes the falsity of the kind of fine-grained context-sensitivity that we have been advocating in this chapter, and second, that the particular confidence-judgments about particular cases that we were relying on need not be significantly undermined by the difficulty of subsuming them under some completely general principle that applies to all indicative conditionals, whatever their logical structure.  


	% - give it both object-level and metalinguistically
	% - be careful to make the contrast between how confident we *are* and how confident we should be.
	% - anaphoric ‘that’ - even if we can sometimes pick up on other propositions, surely we should be *able* to pick up on the one semantically expressed.
	% - if there is this other proposition floating around why not say that it's the one expressed

The confidence-theoretic arguments suggest that very often, an indicative conditional expresses a proposition that is not known to have the same truth value as the corresponding material conditional, and that fairly often, the proposition expressed by the indicative conditional is actually false even when the antecedent is false or the consequent true.  Of course there are also many occasions where an indicative conditional is used to express a proposition that is known to have the same truth value as the corresponding material conditional.  Boringly: when the material conditional is known to be false, the corresponding indicative conditional can also be known to be false (since it entails the material conditional).  Slightly less boringly: if the operative notion of accessibility is such that the material conditional is known to be true in all accessible worlds, then the indicative conditional can be known to be true, since its truth is entailed by the proposition that the material conditional is true in all accessible worlds.%
\footnote{If accessibility is just being consistent with what we know, this condition boils down to our knowing that we know the material conditional.}
But the structure of our analysis also allows that in some contexts, the accessibility parameter may be set in such a way that we can know that the conditional has the same truth value as its material counterpart without knowing which truth value this is.  For a conditional ‘If $P$, $Q$’, this will happen whenever the operative accessibility relation is such that no world compatible with what we know is a ‘not-$P$’-world from which a ‘$P$ and not~$Q$’-world is accessible.  This condition guarantees that at every ‘not $P$’-world compatible with what we know ‘If $P$, $Q$’ is true (and thus has the same truth value as ‘$P⊃Q$’).  Meanwhile, the requirement that every world is the closest world to itself guarantees that ‘If $P$, $Q$’ always has the same truth value as ‘$P⊃Q$’ at $P$-worlds (whether or not they are compatible with what we know).  

The theory that accessibility is sometimes constrained by salient questions under discussion suggests a way in which this might happen.  Suppose for example that after Smith left on a parachute jumping trip, we found out that there were some non-working parachutes lying around the airport.  The question whether the parachute Smith took with him was working is a salient and pressing one from our point of view.  We think Smith is likely, but not certain, to notice a broken parachute and so is unlikely to jump to his death.  As we investigate further, we find to our dismay that \emph{most} of the parachutes were broken.  Suppose one of us pessimistically utters ‘If Smith jumped, he died’.  There are two natural ways of reacting to this speech.  On the one hand, we could say ‘That's quite improbable---remember that Smith is a pretty experienced skydiver and is likely to have thoroughly checked his parachute before deciding whether to jump’.  But on the other hand, faced with the evidence of the abundance of broken parachutes, we might say ‘More likely than not you're right, but don't forget the possibility that he got one of the good parachutes’.  Plausibly these two reactions correspond to two resolutions of the context-sensitivity of ‘If Smith jumped, he died’.  A natural diagnosis is in the second case, the accessibility relation requires match with respect to the salient question under discussion, namely whether Smith's parachute was working.  Since our knowledge rules out the possibility that Smith died and had a working parachute, the possibility that Smith jumped without a working parachute and din't die, and the possibility that Smith had a working parachute and didn't jump, the effect of imposing this constraint on accessibility is to ensure that no worlds where Smith jumped and didn't die are accessible from any of the worlds compatible with our knowledge where Smith didn't jump (since all of these worlds are worlds where Smith didn't have a working parachute).  By the observation in the previous paragraph, this is sufficient for the relevant reading of ‘If Smith jumped, he died’ to agree in truth value with the material conditional ‘Smith jumped ⊃ Smith died’ at all epistemically possible worlds.

One might worry that the proposal that the accessibility parameter for indicative conditionals is readily constrained by questions under discussion would predict that it is much more common than it in fact seems to be for indicative conditionals to behave in a material-like way.  After all, when someone utters ‘If $P$, $Q$’ the question whether $P$ is true is always salient during the utterance, and often salient prior to it.  But if the conditional is interpreted relative to an accessibility parameter requiring match with regard to whether $P$, it must agree in truth value with ‘$P⊃Q$’ at every world, since there is no way for a $P$-and-not-$Q$ world to be accessible from a not-$P$ world.  However, given the presupposition of nonvacuity, this particular kind of constraint by a question under discussion will not take be intended in normal circumstances: if being accessible requires agreeing with actuality with regard to the truth value of $P$, the only way we could reasonably take it for granted that there are accessible $P$-worlds would be if we could reasonably take it for granted that $P$ is true.  But if we are taking it for granted that $P$ is true we are unlikely to assert ‘If $P$, $Q$’, since we will then be in a position to assert something stronger and simpler (or equally simple) such as ‘$P$ and $Q$’ (or ‘Must $Q$’, or perhaps just $Q$).  Similarly, although the question whether $Q$ is true is often under discussion in the setting of an utterance of ‘If $P$, $Q$’, the accessibility parameter is unlikely to be constrained by this question: this interpretation makes ‘If $P$, $Q$’ equivalent to the disjunction ‘(Must not $P$) or $Q$’, which is in turn equivalent to $Q$ modulo the presupposition of nonvacuity---but clearly there would be no point in uttering ‘If $P$, $Q$’ to convey a total content that could equally well have been conveyed by just uttering $Q$.  


% So, are material conditional readings common?
% \begin{prop}
% 	\item
% 	Very implausible for counterfactuals - many counterfactuals with false antecedents seem straightforwardly false.
% 	\item
% 	? An argument from betting (if you bet on a conditional you don't get to collect if the antecedent turns out to be false)?  This is obviously not that strong an argument, the fact that no money changes hands in this case could just be a local convention about the practice of making bets.
% \end{prop}

% * Example: the double-black versus red-blue coin.
% \begin{prop}
% 	\nitem
% 	[More likely than not] if the coin didn't land blue it landed red.
% \end{prop}
% ??? John argument from ‘desparately hoping’ (not so good)

% *****

% The foregoing reasons for  concerned cases where a conditional is asserted, as opposed to being a constituent of a more complex assertion.  In general, there is no guarantee that Gricean reasoning as applied to asserted sentences will carry over to embedded occurrences.  So let's turn to considering some examples of embedded conditionals.  For now we will look at conjunctions and disjunctions, setting aside cases where conditionals are embedded under quantifiers, modals, or other conditionals.%
% \footnote{Since sentence-level negation operators like ‘It is not the case that’ are only marginally part of ordinary English, our reactions to conditionals embedded under such operators are too theory-laden to be fairly thought of as natural language data.}
% Beginning with the case of conjunction, there are many cases where material-like truth conditions for conditional conjuncts are no more plausible than they would be for unembedded conditionals, and where the explanation in terms of presuppositions and implicatures seems to extend pretty well.  In lots of cases, conjunctions seem to inherit the presuppositions of both of their conjuncts.  For example,
% \begin{prop}
% 	\nitem \label{conjpresupposition}
% 	Mary didn't feed her cat and Susan didn't feed her chickens
% \end{prop}
% pretty clearly presupposes that Mary has a cat and Susan has chickens.%
% \footnote{The orthodox lore discussed in the previous section predicts something a little weaker: namely, that Mary has a cat and if Mary didn't feed her cat Susan has chickens.  Some have used sentences like \ref{conjpresupposition} to argue against the orthodox lore; defenders of the lore (***) will say that the conjunctive presupposition can be recovered from the predicted one given background knowledge.  We we will not need to take sides in this debate since our discussion could easily be adapted either way.}
% Consider:
% \begin{prop}
% 	\nitem \label{boringand}
% 	If Jones bet on Ted in Race 1 he won, and if Smith bet on Ned in Race 2, he lost.
% \end{prop}
% If you don't like a material conditional view of unembedded indicative conditionals, there is no additional temptation in this case to think that the conditionals in \ref{boringand} are material---in particular, there is no temptation to think that \ref{boringand} is automatically true if neither bet was made.  We can explain this by postulating that \ref{boringand} inherits the presuppositions of its conjuncts in the same way as \ref{conjpresupposition}: we presuppose, that is, that there are some worlds that are accessible in the sense relevant to the first conjunct where Jones bet on Ted in race 1, and some worlds that are accessible in the sense relevant to the second conjunct where Smith bet on Ned in race 2.  But for Gricean reasons, uttering \ref{boringand} will implicate that one does not know either that Jones made his bet or that Smith made his bet, since otherwise one could simply delete the relevant ‘if’ clause, or replace it with a conjunction.  Thus, if the accessibility relation for the first conditional is one that worlds where Jones doesn't bet on Ted never bear to worlds where he does bet on Ted, or the  accessibility relation for the second conditional is one that worlds where Smith doesn't bet on ned never bear to worlds where he does bet on Ned, one will be risking violating the presuppositions of \ref{boringand} in uttering it.

\citet{McDermottTCS} introduces some interesting examples where conjunctions of conditionals seem to have the simple truth-conditions that a material-conditional analysis would predict.  Suppose that after a die has been rolled but before the result has been revealed, someone asserts \ref{mcdermott}:
\begin{prop}
	\nitem \label{mcdermott}
	If the number showing is even it's 6, and if's odd it's 1.
\end{prop}
(Maybe the speaker thinks the die was weighted, or is hoping to get a reputation for  paranormal abilities….)  In this setting, \ref{mcdermott} feels straightforwardly equivalent to the claim that the number showing is 1 or 6.  In particular, if it turns out that the die landed on 6, we don't think along the following lines: the first conjunct is true, but the status of the conjunction is still unclear, because what we've learnt by looking at the die doesn't settle whether the number showing was odd if it was 1---in the terms of our analysis, it doesn't settle whether the die lands 1 at the closest accessible world where it lands on an odd number.  The equivalence in question is smoothly explained by a view on which the conditionals are interpreted as material conditionals: on such a view, the second conjunct is automatically true in worlds where the die landed on an even number, and the first conjunct is automatically true in worlds it landed on an odd number.  
%suggests that on the operative interpretation, the second conjunct is automatically true if the die doesn't land on an odd number.  Similarly, the fact that \ref{mcdermott} seems straightforwardly true if it turns out that the die landed on 1 suggests that the first conjunct is automatically true if the die doesn't land on an even number.  

There are many possible settings of the accessibility parameters for the conditionals in \ref{mcdermott} that would make the conjunction equivalent to ‘It was a 1 or a 6’, some of which seem more natural than others.  One that seems quite natural takes accessibility (for both conditionals) to require (in addition to epistemic liveness) match with regard to the very salient question whether the die landed on an even or odd number.  This will make the conditionals necessarily equivalent to the corresponding material conditionals: for example, in the scenario where the die lands on 1, there are no accessible worlds where it lands on an even number, so the first conjunct is vacuously true (while the second is non-vacuously true).  

An objection to the proposal that the relevant interpretation of \ref{mcdermott} is constructed in this way is that it conflicts with the presupposition of nonvacuity.  Of course, \ref{mcdermott} is a conjunction of conditionals, not a conditional, so to extract any predictions for its presuppositions, we will need to appeal to some claims about “presupposition projection”, the manner in which the semantic presuppositions of compound sentences are determined by those of their constituents.  For the case of conjunction, there is some dispute in the literature about how this should work.  Some examples are well explained by the simple theory that when $P$ presupposes $p$ and $Q$ presupposes $q$, ‘$P$ and $Q$’ presupposes the conjunction of $p$ and $q$.  (For example, ‘Her house is expensive and her car is cheap’ seems to presuppose what ‘She has a house and she has a car’ expresses.)  But there are other well-known examples are problematic for this theory, and suggest the following more complicated theory: when $P$ expresses $p$ and presupposes $p'$ and $Q$ expresses $q$ and presupposes $q'$, ‘$P$ and $Q$’ presupposes $p'∧((p∧p')⊃q')$, the conjunction of $p$ with a material conditional whose consequent is $q'$ and whose antecedent is the conjunction of $p$ and $p'$.%
\footnote{While the standard version of this story uses a material conditional, other interpretations of the conditional would do just as well with the data motivating the theory.}  
For example, ‘He has a car that he keeps in his garage and his car is expensive’ does not seem to presuppose what ‘He has a garage and he has a car’ expresses, but only what ‘He has a garage’ expresses---which is logically equivalent to ‘He has a garage ∧ ((he has a garage ∧ he has a car that he keeps in his garage) ⊃ he has a car)’.  The question how to reconcile this impasse (the ‘proviso problem’) is one of the major current debates in the theory of presuppositions (**cites**).  But for our purposes, it suffices to observe that on both the simple theory and the more complicated theory, conjunctions always inherit the presuppositions of their \emph{first} conjunct.  So we would predict that \ref{mcdermott} will semantically presuppose that its first conjunct isn't vacuously true, which on an interpretation where accessibilityrequires match with regard to whether the die lands on an odd number would require that the die lands on an even number.  But clearly the utterance of \ref{mcdermott} carries no suggestion that we can take it for granted that the die lands on an even number---rather the contrary.   Moreover, even on the more complicated theory, we would also predict that due to its second conjunct \ref{mcdermott} would presuppose the material conditional whose antecedent is the conjunction of the propositions expressed and presupposed by the first conjunct, and whose consequent is the presupposition of ‘if the number showing is odd it's 1’, which on the proposed accessibility relation entails that the number showing is odd.  So on either theory, it will be impossible for \ref{mcdermott} to have the status ‘true with only true presuppositions’ on the proposed interpretation.  

How big a problem this is for the proposal depends on how willing we are to think of the association between semantic presuppositions and actual speaker commitments as a contextually defeasible matter rather than something as robust as the association between semantically expressed content and speaker commitments.  As noted in \ref{sect:nonvacuity}, some of the phenomena that get standardly lumped together under the label ‘presupposition’, including the tendencies for hearers to infer from ‘He doesn't know $P$’ to $P$ and from ‘I have not stopped smoking’ to ‘I used to smoke’, do in fact seem rather delicate and easily defeated.  Moreover, interestingly, there are examples which suggest that one possible defeating factor is the utterance of a conjunctive sentence whose conjuncts have conflicting presuppositions: sometimes, conjunctions for which the standard theories of presupposition projection for conjunctions would predict a contradictory or otherwise false presupposition seem perfectly fine.  For example
\begin{prop}
	\nitem \label{innocent}
	He doesn't know she is guilty, and he doesn't know she is innocent
\end{prop}
would be standardly predicted to presuppose either the contradictory proposition expressed by ‘She is guilty and she is innocent’, or something equivalent to the proposition expressed by ‘She is guilty and he knows she is guilty’ which contradicts the asserted content, whereas in fact it is perfectly fine and indeed seems to carry no suggestions in particular that anything is to be taken for granted (other than the genders of the referents of the pronouns). Similarly,
\begin{prop}
	\nitem \label{regent}
	I won't salute the king and I won't salute the regent
\end{prop}
would be standardly predicted to presuppose that there is a king; but if it's taken for granted that if there's a king there isn't a regent and vice versa, an utterer of \ref{regent} will probably not be taken to be committed to the presupposition predicted by standard lore.%
\footnote{We had better not overgeneralise the lesson here: sometimes, conjunctive sentences for which we would predict presuppositions which are contradictory, or which contradict their assertive content, really sound terrible, and we still want to be able to appeal to their presuppositions to explain why they are terrible.  For example, we want to be able to say that ‘If $P$, $Q$ and if $P$, not $Q$’ is bad because the only way for it to be true is for it to have a false presupposition.  (***Is there an interesting contrast to be drawn here between sentences whose presupposition is contradictory by itself and sentences whose presupposition merely contradicts their assertive content?) }  

% \begin{prop}
% 	\nitem
% 	(Maybe we'll think of more compelling examples, with universal quantifiers, definites, etc.)
% 	? She didn't stop being a kind person and she didn't stop being a mean person
%	? Every F person is G and every non-F person is G
% \end{prop}


The presupposition of nonvacuity seems to sometimes disappear in similar ways in conjunctions. Suppose that in a poker game, I know that Sally's four cards are either the ten, jack, queen, and king of spades, or two nines and two eights. I have two kings and three aces. Sally folds; I look at the top card and see that it is the nine of spades. I say
\begin{prop}
	\nitem \label{cards}
	If she had drawn that card and had a straight she would have won, but if she had drawn that card and had a full house she would have lost.
\end{prop}
This seems acceptable given what I know. To make it acceptable, it seems that we need an accessibility parameter that holds fixed the actual makeup of Sally's first four cards and my five---otherwise, it's puzzling how my actual knowledge concerning the values of these cards would license the assertion of \ref{cards}.  However, if the accessible worlds must match actuality in these respects, inevitably one of the two counterfactuals will be vacuously true.  Given these precedents, it is not unreasonable to diagnose \ref{mcdermott} as another case of a similar sort.  After all, interpreted using the vacuity-friendly accessibility relation, it is just like \ref{innocent} in that the presuppositions of the two conjuncts are jointly inconsistent.  

% Given that the inferences about what speakers take for granted associated with presuppositions are in any case often fairly weak and defeasible, it is not surprising to suggest that they could be defeated in cases where following the usual rules would lead the conclusion that someone is taking contradictory things for granted.%
% \footnote{As noted earlier, ‘presupposition’ labels a rather heterogeneous class of phenomena, and there are big differences within this collection as regards how defeasible the relevant inferences are.  For example, parentheticals lead to a very strong inference---no co-operative speaker who doubts that John is a philosopher will assert ‘John, who is a philosopher, went home’, or assert any conjunction with this as a conjunct.}

An alternative approach to \ref{mcdermott} is to find some other values of the accessibility parameter---perhaps different for the two conjuncts---which gets the conjunction to be equivalent to ‘The die is showing 1 or 6’ in a way that doesn't require either conjunct to be vacuously true.  For example, we could take the accessibility relation for both conditionals to be the intersection of epistemic accessibility with an two-cell equivalence relation in which worlds where the die landed 1 or 6 form one cell, and worlds where it landed 2--5 form the other cell.  This also makes \ref{mcdermott} true when the die landed 1 or 6 and false otherwise, and guarantees that neither conditional is vacuously true.  For example, if it landed 1, the first conjunct is nonvacuously true (since the actual world is the closest accessible odd-number world), and the second conjunct is also nonvacuously true (since there are accessible worlds where an even number is showing, in all of which the number showing is 6).  While this approach gets the desired truth-condition, the accessibility constraints it requires will often be rather gerrymandered---e.g.\ to deal in a similar way with ‘If Mary goes to Paris she'll have a good time and if she doesn't go she'll have a bad time’ one will need to use a partition where worlds where Mary has a good time in Paris or a bad time elsewhere are in one cell, and worlds where Mary has a bad time in Paris or a good time elsewhere are in another.  Moreover, the approach in terms of “presupposition cancellation” explains something mysterious, namely that even when we are in a mood to count an actual result of 1 as vindicating \ref{mcdermott}, we are reluctant to just assert the first conjunct on its own (‘I was right on both counts---if the number showing was even it was 6’).  This might be explained by as a matter of the presupposition retaining its grip on us when we consider the conjunct individually.  By contrast, if the intended interpretation of \ref{mcdermott} involved an accessibility parameter making both conjunts non-vacuously true, it is harder to see what would be holding us back from following up by asserting them individually.  

% * what's the general strategy here - how does it extrapolate to the case of non-exhaustive and/or overlapping antecedents, or overlapping consequents?
%
% * is there something to say about why it struck as more artificial?

The phenomenon saliently brought out by \ref{mcdermott} can also be seen in conjunctions of conditionals whose antecedents are not exhaustive of the epistemically possible cases, such as \ref{nonexhaustive}:
\begin{prop}
	\nitem \label{nonexhaustive}
	If was 1 or 2 or 3 it was 1, and if it was 5 or 6 it was 6.
\end{prop}
Like \ref{mcdermott}, this looks straightforwardly true in a scenario where the die turns out to have landed on 1 or 6.  In those scenarios, we are not inclined to object that while one of the conjuncts has been established the other one remains dubious.  However, by contrast with \ref{mcdermott}, it is less clear whether the truth conditions we naturally assign to \ref{nonexhaustive} are those that would be predicted by a material conditional analysis: in particular, if it turns out that the die landed 4, it is unclear how we should be retrospectively evaluating the utterance of \ref{nonexhaustive}.  And if, before looking at the die, we are trying to figure out how much confidence to assign to the proposition expressed by an utterance of \ref{nonexhaustive}, it is unclear how the probability of the die landing 4 should enter into our calculations.%
\footnote{We could get a similar effect by changing the backstory for \ref{mcdermott} so that we aren't sure whether the die was rolled.}
In our framework, we can get different judgments by using different accessibility relations.  Perhaps the most obvious candidate is the one that partitions the epistemically live worlds into those where the die lands 1, 2, or 3, those where it lands 4, and those where it lands 5 or 6, so that \ref{nonexhaustive} is true when the die lands 4.  But even if a presupposition-cancellation account of \ref{mcdermott} is right, the fact that this interpretation makes the failure of the presuppositions of \emph{both} conjuncts a possibility may make it less natural than the corresponding interpretation of \ref{mcdermott}.  There are many other accessibility relations that guarantee truth when the die lands 1 or 6: it's enough that worlds where the die lands 5 are inaccessible from those where it lands 1, and worlds where it lands 2 or 3 inaccessible from those where it lands 6.  We could try to distinguish the possibilities by asking how confident we should be in the truth of \ref{nonexhaustive} (both in the envisaged scenario and in variants with a weighted die).  However we find it hard to muster clear confidence-theoretic verdicts about this case.  (It may well be vague which accessibility relations are in play.).%
\footnote{For example, we might use a partition where the worlds where it lands 1-4 are in one cell and those where it lands 5 or 6 are in the other; then the truth value in worlds where it lands 4 depends on whether it lands 1 in the closest world where it lands 1-3, which suggests (given the pattern of uncertainty about closeness that seems to be emerging) that our credence in the proposition expressed should be $2/5 + (1/3 × 1/6) = 41/90$.  We might also use different accessibility relations for the two conjuncts, e.g.\ using the $\set{1-4,5-6}$ partition for the first conditional and the $\set{1-3,4-6}$ partition for the second.  This makes it harder for the conjunction to be true in the cases where the die lands 4, since it has to both land 1 in the closest (epistemically possible) world where it lands 1-3 and land 6 in the closest world where it lands 5 or 6.}

% %
% \footnote{*** If we took it that the conjunction suffers from a presupposition-failure in a case where \emph{both} of its conjuncts would suffer from presupposition failure taken by themselves, we would also predict that \ref{nonexhaustive}, read in this way, presupposes that the die didn't land 3 or 4.  This doesn't seem right for the kind of utterance of \ref{nonexhaustive} we were imagining.
%
% 	*Note that ‘I don't know that it landed on 1 and I don't know that it landed on 6’ does not seem to presuppose ‘It landed on 1 or 6’.
% 	}

Let us turn next to disjunctions of indicative conditionals.  Here too there are some cases where a disjunct with a false antecedent seems to be simply ignored in ways that are surprising for standard accounts.  Consider:
\begin{prop}
	\nitem \label{bigorsmallprize}
	He either got a huge prize if he answered the final question correctly, or a small prize if he didn't.  
\end{prop}
Mysteriously, we seem to treat \ref{bigsmallprize} as if it were a disjunction of \emph{conjunctions} rather than conditionals:
\begin{prop}
	\nitem \label{prizedisjunction}
	He either got a huge prize and answered the final question correctly, or a small prize and didn't answer the final question correctly.  
\end{prop}
In particular, if we later find out that he did answer the final question correctly but didn't get a huge prize, we will feel that the person who uttered \ref{bigorsmallprize} simply turned out to be wrong.  We don't think to ourselves: the first disjunct of \ref{bigorsmallprize} is false, but that still doesn't settle whether the whole disjunction is false, since the truth of second disjunct is still an open possibility.  By contrast with the conjunction case, a material conditional account does particularly poorly here: understood as a disjunction of material conditionals, \ref{bigorsmallprize} is tautologous, since there is no way for both of the conditionals to have true antecedents.

We might contemplate trying to account for \ref{bigorsmallprize} by appealing to a suitably crafty resolution of the accessibility parameters, such that if one disjunct has a true antecedent, the truth value of the other disjunct is guaranteed to match the truth value of that disjunct.  For example, we could take the accessibility relation (for both disjuncts) to be the conjunction of epistemic possibility with a two-cell equivalence relation in which worlds where \ref{prizedisjunction} is true form one cell, and worlds where it is false form the other.  But this looks rather \emph{ad hoc}: the polar question corresponding to \ref{prizedisjunction} certainly doesn't feel intuitively salient in the relevant way.  Rather than appealing to such relations, we are tempted to appeal to the (intuitively salient) question whether he answered the final correctly, and lean on an analogy between \ref{prizedisjunction} and other examples where we seem to treat a disjunction of a false sentence with a true proposition and a true sentence with a false presupposition as if it were false.  Consider:
\begin{prop}
	\nitem
	\begin{prop}
		\aitem
		He didn't salute the king or he didn't salute the regent.
		\aitem
		I either won't regret doing it or won't regret not doing it.  
		\aitem \label{drinking}
		Mary either didn't stop drinking or didn't stop smoking.  
		% \aitem
		% Few people who received the will power supplement didn't stop drinking or didn't stop smoking.
		% \aitem
		% He either hasn't read Chomsky or hasn't read Kant.
	\end{prop}
\end{prop}
%We're imagining these sentences being uttered by speakers who are sure that exactly one of the two disjuncts has a true presupposition, and the question which one it is is salient.  
For example we might utter \ref{drinking} because we see Mary looking disappointed with herself and suspect it's because her New Year's resolution didn't work out---we know she either used to smoke or used to drink and had resolved to give up, but aren't sure which.  If it turns out that she never smoked, successfully stopped drinking, and was feeling disppointed for some completely different reason, we'll feel that something false was said by \ref{drinking} despite its literal truth (she really didn't stop smoking, since she never started).  There's lots to be said about this phenomenon, which may not have much specifically to do with presuppositions (for example \ref{drinking} might still feel false in the case where Mary used to both drink and smoke, successfully stopped drinking, didn't even try to stop smoking, and was disppointed for some completely different reason).  But we won't investigate the general phenomenon further here.%
\footnote{***Discuss McDermott's own three-valued approach to these data and its weird predictions}

%The general moral we want to draw is that the behaviour of indicative conditionals in these embedded environments looks like it










\section{Accessibility for counterfactuals}\label{sect:cf}
As many authors have observed, our standard for evaluating a counterfactual whose antecedent concerns a particular period of time involves helping ourselves to all kinds of facts about earlier times. For example, if we know John has had breakfast every day for the last year, we will unhesitatingly endorse
\begin{prop}
	\nitem \label{breakfast}
		If John had forgotten to have breakfast on Tuesday morning, that would have been the first time this year.
\end{prop}
The phenomenon is pervasive: even when the consequent of a counterfactual isn't about the past, in deciding what to make of it we will typically be \textbf{tacitly holding the past fixed}. For example, in thinking about
\begin{prop}
	\nitem 
		If I had run a four minute mile this morning, I would have been extremely surprised.
\end{prop}
we seem to be holding fixed the speaker's previous track record.

Our favoured account of this phenomenon appeals to fine-grained context-sensitivity in the accessibility parameter. When considering a counterfactual whose antecedent is about a certain period, it is natural to resolve its context-sensitivity in such a way that the accessible worlds are all required to match with respect to earlier times.%
\footnote{The match may not be perfect (cite), and it may need to be restricted to allow for a smooth transition into the kind of event described by the antecedent. Indeed for certain antecedents --- consider ‘If a giant comet had struck Washington DC yesterday afternoon\ldots{}.’ --- we may need quite a long stretch of preceding time where there is nothing like exact match in order to achieve the smooth transition.}
 Notice that the closeness relation plays no role in this account of why counterfactuals like \ref{breakfast} are true: \emph{all} the accessible worlds where John forgets to have breakfast on Tuesday are worlds where it his his first time doing so this year. So for example, while our account suggests that \ref{wednesdaybreakfast} is true in the context \emph{it} naturally evokes for the same reason as \ref{breakfast}, it provides no reason to expect \ref{wednesdaybreakfast} to be true relative to context naturally evoked by \ref{breakfast}:
\begin{prop}
	\sitem[\chisholm{breakfast}{'}] \label{wednesdaybreakfast} 
	If John had forgotten to have breakfast on Wednesday morning, that would have been the first time this year.
\end{prop}
After all, the context evoked by \ref{breakfast} is one where the accessible worlds are merely required to match with respect to history up to Tuesday; since some of them involve forgetting to have breakfast on Tuesday and on Wednesday, the accessibility facts provide no guarantee for the truth of \ref{wednesdaybreakfast} in this context. And indeed, the account of closeness that we will develop in chapter 2 will provide no grounds for confidence that the closest worlds where John forgets breakfast on Wednesday are not worlds where he forgets breakfast on Tuesday as well. Meanwhile, while \ref{mondaybreakfast} is non-vacuously true in the context it naturally evokes, it is vacuously true in the context evoked by \ref{breakfast}.
\begin{prop}
	\sitem[\chisholm{breakfast}{''}] \label{mondaybreakfast}
	If John had forgotten to have breakfast on Monday morning, that would have been the first time this year.
\end{prop}
By contrast, Lewis's influential treatment of counterfactuals tells a rather different story about why holding the past fixed is a reliable way of evaluating counterfactuals. For Lewis, what does the work is closeness, not accessibility---indeed, his account would work even on the assumption that all metaphysically possible worlds are accessible in all contexts.%
\footnote{Check if Lewis says this. Mention others who do.}
 Moreover, Lewis's account of the phenomenon does not rely in any essential way on context-sensitivity. For example, he thinks that there is a single context on which we can knowledgeably utter \ref{breakfast}, \ref{wednesdaybreakfast}, and \ref{mondaybreakfast}: given that we know that John in fact had breakfast every day this year, we know that worlds where John skips breakfast for the first time on Wednesday are closer than worlds where he skips breakfast for the first time on Tuesday, which are in turn closer than worlds where he skipped breakfast on Monday. More generally, the closeness relation that Lewis thinks is standard is one on which worlds whose history diverges from that of the actual world at later times are ipso facto closer than worlds where history diverges earlier (Lewis 1979).

This feature of Lewis's view leads to some well-known oddities that our view avoids. Pollock (***) noticed that Lewis's view seems to underwrite counterfactuals such as the following:
\begin{prop}
	\nitem \label{coat}
		If my coat had been stolen last year it would have been stolen on December 31st.
\end{prop}
Given that my coat was not in fact stolen, on Lewis's view worlds where it is stolen are closer to the extent that they diverge later from the actual world.%
\footnote{This is a slight oversimplification: if there are certain days on which it would have taken a large miracle for my coat to be stolen they will count as further away from actuality on Lewis's account. But this doesn't make the problem go away. Let's just make the plausible assumption that on any day last year, a small miracle (perhaps taking place in the brain of some errant youth) on that day would have been enough to take the world onto a trajectory where my coat gets stolen on that day.}
 But intuitively, unless I have some special reason to think that my coat was unusually vulnerable to theft on December 31st, I have no right to assert \ref{coat}.

It would really be quite bizarre if we had to start computing counterfactuals in the way seemingly recommended by Lewis. Just to take one more example: suppose that, sadly, someone fell from a ship and drowned while people stood and watched without anyone diving in to help. On Lewis's approach, there seems to be a quite strong reason to accept
\begin{prop}
	\nitem \label{drowning}
		If someone had dived in to try to help her, she would still have drowned.
\end{prop}
After all, if the later the divergence the closer, then the closest worlds will be ones where someone dives in too late, even if there are plenty of worlds where a more timely rescue attempt would have saved her.

The examples cry out for a treatment on which the only period of history that is completely held fixed is one that predates the period in question---last year in the case of \ref{coat}, the period during which the drowning victim was in the water in the case of \ref{drowning}. A contextualist approach like ours can readily accommodate this. Given the antecedent of \ref{coat}, it would be completely unmotivated to select some time after the beginning of last year as the basis for resolving accessibility: the natural constraint on accessibility will allow more or less indiscriminate importation of facts about the actual world only for times prior to last year.

One possible response to the problems for Lewis's account would be to maintain that for some reason \ref{coat} is computed as equivalent to
\begin{prop}
	\sitem[\chisholm{coat}{'}] 
	On any day last year, if my coat had been stolen on that day it would have been stolen on December 31st. 
\end{prop}
It's not obvious what would motivate this seemingly \emph{ad hoc} mechanism beyond a desire to avoid the counterexamples. But in any case, the proposal is clearly inadequate. Suppose that halfway through the year I moved from a dangerous neighbourhood to a much safer one; then I could naturally assert
\begin{prop}
	\nitem \label{probcoat}
		Probably if my coat had been stolen last year, it would have been stolen during the first half of the year.
\end{prop}
By contrast, \ref{probcoat'} is obviously unassertable:
\begin{prop}
	\sitem[\chisholm{probcoat}{'}] \label{probcoat'} 
	Probably on any day last year, if my coat had been stolen on that year it would have been stolen during the first half of the year.
\end{prop}

The problems raised by the coat and drowning examples are not peculiar to the details of Lewis's approach, or even to the possible worlds framework. Given certain minimal logical assumptions, problems of this sort will arise for any view that attempts to secure the reliability of ‘holding the past fixed’ for antecedents about different times without appealing to context-shift. Suppose that there was a thunderstorm all weekend in the area that destroyed many trees, but my tree luckily survived. Now consider
\begin{prop}
	\nitem \label{sundaysad}
		If this tree had been in pieces on Sunday morning, Sunday would have been an especially sad day for me.
\end{prop}
This is a typical example of the kind of sentence that is evaluated by holding the past fixed; so it's plausible that in the context naturally evoked by \ref{sundaysad}, \ref{sundaysaturday} is true:
\begin{prop}
	\nitem \label{sundaysaturday}
		If the tree had been in pieces on Sunday, it would have been intact on Saturday afternoon.
\end{prop}
But surely \ref{sundayweekend} and \ref{weekendsunday} are true as well (in the relevant context):
\begin{prop}
	\nitem \label{sundayweekend}
		If the tree had been in pieces on Sunday, it would have been destroyed during the weekend.
	\nitem \label{weekendsunday}
		If the tree had been destroyed during the weekend, it would have been in pieces on Sunday.
\end{prop}
After all, Sunday is part of the weekend, so denying \ref{sundayweekend} looks completely unpromising; meanwhile given that trees don't reform once they have split into pieces, \ref{weekendsunday} seems safe too. But given \ref{sundayweekend}, \ref{weekendsunday}, and \ref{sundaysaturday}, we are in a position to apply the inference-schema sometimes called ‘CSO’, according to which arguments of the following form are valid:
\begin{prop}
	\sitem[CSO]
	If $A$ then $B$; If $B$ then $A$; If $A$ then $C$ ⊦ If $B$ then $C$
\end{prop}
Plugging in ‘The tree was in pieces on Sunday’, ‘The tree was destroyed during the weekend’, and ‘The tree was intact on Saturday afternoon’ for $A$, $B$, and $C$, we can derive
\begin{prop}
	\nitem \label{weekendsaturday}
		If the tree had been destroyed during the weekend, it would have been intact on Saturday afternoon.
\end{prop}
But \ref{weekendsaturday} seems very problematic in the same way as the coat and drowning examples considered earlier.

CSO is valid according the best-known logics for counterfactual conditionals, namely those of Stalnaker and Lewis. And to our my minds its popularity is well-deserved. While it may not be immediately compelling when first encountered, there are several argumentative routes to it from principles whose appeal is more immediate. One route consists of
\begin{prop}  
	\sitem[RT (‘Cumulative Transitivity’)] 
	If $A$ then $B$; if $A$ and $B$ then $C$ ⊦ If $A$ then $C$ 
	\sitem[RCV (‘Very Limited Antecedent Strengthening’)] 
	If $A$ then $B$ and $C$ ⊦ If $A$ and $B$ then $C$
\end{prop}
For suppose the premises of CSO hold: if $A$, $B$; if $B$, $A$; and if $A$, $C$. Given the first and third, by RCV it follows that if $A$ and $B$ then $C$; and from this and the second premise (If $B$, $A$) RT yields that if $B$, $C$.%
\footnote{Also note that CSO entails CT and VLAS, and that they are equivalent modulo CEM.}


Reliance on RT seems to pervade a great deal of our ordinary reasoning using conditionals, both indicative and counterfactual. Consider how good the following piece of reasoning looks: ‘If they attacked they would use their cavalry; if they attacked using their cavalry, they would win; so if they attacked, they would win’. And consider how terrible the following speech sounds: ‘If he had gone to the party, he would have gone with Janet, and if he had gone to the party with Janet, he would have had a good time, but I doubt that if he had gone to the party he would have had a good time’. Clearly there is some general principle underlying the goodness of the good reasoning, and the badness of the bad speech, and it is hard to see what it could be if RT is invalid. Regarding RCV, similar remarks apply to the following excellent argument (which involves a contraposed application of RCV): ‘If they attacked they would use their mercenaries; but it's not the case that if they attacked using their mercenaries they would win; so it's not the case that if they attacked they would win’, and also to the following appalling speech: ‘If he had gone to the party he would have gone with Janet, and had a great time; however if he had gone to the party with Janet, he would have had a terrible time.’ (Note too that all of this looks equally compelling if we shift everything to an indicative form: ‘If they attacked they used their cavalry’, etc.)

Another route to CSO worth mentioning is to derive it from RCV together with the following extremely compelling principles:
\begin{prop}
	\litem[Cases]
	If $A$ and $B$ then $C$, If not-$A$ and $B$ then $C$ ⊦ If $B$ then $C$
\end{prop}
Unlike the previous derivation, this one relies on some background principles of the logic of conditionals, albeit ones that are almost universally accepted, namely Identity (‘If $A$ then $A$’), Finite Agglomeration (If $A$ then $B$, If $A$ then $C$ ⊦ If $A$ then $B$ and $C$) and closure of consequents under logical consequence. \textbf{Say more about these here if not earlier.} Suppose again that the premises of CSO hold: if $A$, $B$; if $B$, $A$; and if $A$, $C$. By RCV, if $A$ and $B$ then $C$. Since the material conditional $A⊃C$ is a logical consequence of $C$, we can infer from this by closure that if $A$ and $B$, then $A⊃C$. But by Identity, if not $A$ and $B$ then not $A$ and $B$, so by another appeal to closure, if not $A$ and $B$, then $A⊃C$. So by Cases, if $B$, then $A⊃C$, and by Finite Agglomeration, if $B$ then ($A$ and $A⊃C$). A final application of closure yields that if $B$ then $C$.%
\footnote{Cite/thank Andrew Bacon.}


Like RT, Cases also seems to underlie a good deal of ordinary conditional reasoning, both counterfactual and indicative. Consider how good the following piece of reasoning looks: ‘If they attacked using their cavalry they would win; if they attacked without using their cavalry they would also win; so if they attacked, they would win’. And consider how bad the following speech sounds: ‘If he had gone to the party with Janet, he would have had a good time; if he had gone to the party without Janet, he would have had a good time; but I doubt that he would have had a good time if he had gone to the party.’ Clearly there is some general principle underlying the goodness of reasoning and the badness of the speech, and it is hard to see what it could be if Cases is invalid.

(Both of the above paths required appeal to pairs drawn from the trio of RT, RCV, and Cases. But as we will later show, each member of the trio is by itself sufficient for CSO, given the principle of Conditional Excluded Middle, which we will be defending at length in Chapter 2.)

Those who would deny the validity of the above principles could try to fall back on the idea that they are at least quasi-valid, or have some other positive status of an epistemic sort, for example being such that if one were rationally certain of the premises, one would be entitled to be certain of the conclusion. But the challenges to such an approach are formidable. For one thing, the badness of the relevant speeches does not disappear under ‘Maybe’ or ‘Perhaps’ or ‘For all I know’. For example, the following speech is still terrible: ‘For all I know, he would have gone with Jane and had a good time if he had gone to the party, but wouldn't have had a good time if he had gone to the party with Jane’. We could also consider quantified speeches like ‘Two of those three people would have brought guests and had a great time, if they had gone to the party, but only one of them would have had a great time if they had brought a guest to the party’. \textbf{This isn't doing much for us right now---cut or expand?}

Given the general considerations for thinking that context-sensitivity is pervasive and the costs of giving up CSO, it seems to us much more appealing to explain the relevant data by appeal to context sensitivity rather than jettisoning CSO. On the picture we favour, \ref{weekendsaturday} is indeed straightforwardly true in the context naturally evoked by \ref{sundaysad}, but would be a very risky assertion in the context it naturally evokes. Those who endorse CSO but think that there is no relevant context-sensitivity in play will, like Lewis, be forced to accept the problematic \ref{weekendsaturday}.

The diagnosis of context-sensitivity enjoys substantial independent support, since there are many cases where counterfactuals with the same antecedent suggest different ways of holding the past fixed. For example, each of \ref{refreshed} and \ref{exhausted} seems to be true in the natural context they suggest:
\begin{prop}
	\nitem \label{refreshed}
		If I had been in the Caribbean this morning I would have been feeling refreshed
	\nitem \label{exhausted}
		If I had been in the Caribbean this morning I would have been exhausted from travelling
\end{prop}
In these cases it is the consequent rather than the antecedent that provides the linguistic cues as regards which portions of the past to hold fixed.%%
\footnote{Counterfactuals with stative antecedents especially prone to this.  **Also: counterfactuals with epistemic consequents-counterfactuals designed to display evidential connections-often behave in this way.  Consider: ‘If he had shaken my hand firmly, I would have known he was doing OK’, and contrast this with ‘If he had shaken my hand firmly, he would have been doing OK’.***}  
(Of course there may be non-linguistic cues as well.) On a view where \ref{refreshed} and \ref{exhausted} are both true in the same context, one would have to either live with bizarre counterfactuals like
\begin{prop}
	\nitem \label{refreshedexhausted}
		If I had been in the Caribbean this morning I would have been feeling refreshed and exhausted from travelling
\end{prop}
or else postulate widespread violations of an even more obviously compelling inference rule, namely \ref{conjunction}:
\begin{prop}  
	\litem[Conjunction] \label{conjunction}
	If $A$, $B$; If $A$, $C$ ⊦ If $A$, $B$ and $C$ 
\end{prop}
The appeal to context-sensitivity is thus pretty much inevitable when it comes to sentences like \ref{refreshed} and \ref{exhausted}. We see no reason to deny that the phenomenon manifested by the previously examples is any different: in both cases, contextual triggers determine how much of the past is held fixed in all accessible worlds.

Lewis, for his part, is willing to play the context-sensitivity card in a much more limited way. His picture is that there is a “standard” resolution of context-sensitivity under which worlds that diverge later are closer, and a separate category of “backtracking” resolutions of context-sensitivity where a different standard of closeness is in play. Lewis gives various examples where he thinks a backtracking resolution of context-sensitivity is natural, such as ‘If Jim were to ask Jack for help today, there would have to have been no quarrel yesterday’ (Lewis 1979, p.~33). He might say that one or both of our sentences \ref{refreshed} and \ref{exhausted} fall into this category as well.\footnote{While the examples Lewis focuses on are sentences where the consequent directly concerns times earlier than those mentioned in the antecedent, he is careful to leave open the possibility that the non-standard resolutions of context-sensitivity required to handle those are also natural for a range of other counterfactual utterances that don't have this structure.} Lewis never develops a theory about what plays the role of the closeness relation in backtracking contexts. But we think there is very little prospect for a theory that posits a \emph{single} closeness relation for backtracking contexts. Typically even evaluating a counterfactual in a “backtracking” way we still draw freely on quite a lot of facts about the past---for example, in evaluating \ref{refreshed} we probably draw on the fact that in the not-so-distant past the speaker was not feeling refreshed. There is no prospect of a general rule saying how much of the past to hold fixed for evaluating backtrackers. Given that fine-grained context-sensitivity is thus hard to avoid when dealing with the counterfactuals Lewis categorises as backtrackers, it would not at all surprising if, as we contend, it also extends to those he categorises as non-backtrackers.

------

As well as requiring some bullet-biting about cases like Pollock's coat, Lewis's approach also requires him to give up a prima facie compelling principle about the logic of counterfactuals. If time extends infinitely into the future, and worlds are closer the later they diverge from actuality, then for every non-actual world, there is a closer one. The closeness relation thus fails to obey the ‘Limit Assumption’, which says that every set of worlds contains some worlds that are as close to actuality as any world in the set. Lewis is well aware of this structural feature, and crafts his truth conditions accordingly: a counterfactual is true iff either there is no (accessible) world where the antecedent is true, or at least one world where antecedent is true is such that every world where the antecedent is true that is at least as close as it is one where the consequent is true. However, the upshot of this semantics is that counterfactuals do not generally obey the following principle:
\begin{prop}  
	\litem[Agglomeration] \label{agglomeration}
	If some propositions each would have been true if a certain condition had obtained, then if that condition had obtained, all of them would have been true.
\end{prop}
On Lewis's account AGGLOMERATION is not true in general, although it is true for finite collections of propositions. To see why it can fail in the infinite case, consider the collection of propositions which contains, for each time $t$, the proposition that history proceeds just as it actually does until $t$. Each conditional of the form
\begin{prop}
	\nitem \label{cfquant}
		If history had diverged from actuality at some point, then history would have proceeded just as it actually does until $t$.
\end{prop}
will be true, since for any $t$ worlds which diverge after $t$ are closer than worlds which diverge earlier. But clearly \ref{quantcf} is false:
\begin{prop}
	\nitem \label{quantcf}
		If history had diverged from actuality at some point, then each $t$ is such that history would have proceed just as it actually does until \emph{t.}
\end{prop}
As many authors have noted---starting with Pollock ***---the intuitive force of AGGLOMERATION carries over very smoothly from the finite to the infinite case. It is thus a uncomfortable feature of Lewis's view that it requires driving a wedge between the two kinds of cases.

\begin{itemize} 
	\item
	Add some more discussion of why AGGLOMERATION is so great. 
\end{itemize}

This feature is very hard to avoid on any account that attempts to accommodate the phenomena without making essential appeal to context-sensitivity. For on such an account, all sentences of the form
\begin{prop}
	\nitem \label{lateearly}
		If things had gone differently on day $n$ or later, things would have been the same up to day $n$
\end{prop}
will be true in a single context. Moreover, given the abundance of causal influence in the futurewards direction, there is strong pressure to accept all conditionals of the form
\begin{prop}
	\nitem \label{somelate}
		If things had gone differently on some day, things would have gone differently on some day later than day $n$.
\end{prop}
After all, a scenario where the only differences occur before day $n$ and then history then re-converges to that of the actual world seems quite far-fetched. But obviously \ref{latesome} is also true irrespective of context:
\begin{prop}
	\nitem \label{latesome}
		If things had gone differently on day $n$ or later, things would have gone differently on some day.
\end{prop}
So by CSO, we can conclude that all the conditionals of the following form are true in the same context:
\begin{prop}
	\nitem \label{someearly}
		If things had either gone differently on some day, things would have been the same up to day $n$.
\end{prop}
This is obviously inconsistent with AGGLOMERATION. The case for \ref{somelate} is perhaps not quite watertight: one might think that for all we know, if there had been any divergence at all it would have been a very slight and temporary divergence confined to some initial finite period. But we can forestall this response by substituting the property of being a day such that things go differently from how they actually go on it and all subsequent days for the property of being a day on which things go differently from how the actually go throughout the argument: then \ref{somelate} and \ref{latesome} both become trivial, while the case for \ref{lateearly} is in no way weakened.

Our view preserves AGGLOMERATION: if each of a certain class of propositions is true at the closest accessible world where $P$ is true, then at that world, all of those propositions are true. We think that the above argument against AGGLOMERATION fails because there is no single context in which all sentences of the form \ref{lateearly} are true. Each, nevertheless, is true in some contexts, and indeed has some tendency to evoke the kind of context in which it is true.%
\footnote{***Note that the relevant tendency seems considerably stronger for counterfactuals with eventive antecedents. [Is this a good place to show an awareness of the H? paper about ‘event’ conditionals?]}

Given the centrality of CSO to the arguments we have been considering, it is worth addressing a particular kind of putative counterexample to CSO that has been presented in some recent work by Kit Fine. Fine is concerned in the first instance not with CSO but with the principle that logically equivalent sentences can be substituted salva veritate in conditionals. On Fine's view, for example, \ref{simple} is true whereas \ref{disjunctive} is false, despite the fact that its antecedent is logically equivalent to that of \ref{simple}:
\begin{prop}
	\nitem \label{simple}
		If the tree had been dead on Sunday, it would have been alive on Saturday afternoon.
	\nitem \label{disjunctive}
		If the tree had been either dead on Sunday and not alive on Saturday afternoon, or dead on Sunday and alive on Saturday afternoon, it would have been alive on Saturday afternoon.
\end{prop}
These sentences also generate counterexample to CSO, since Fine accepts that the logically equivalent antecedents of \ref{simple} and \ref{disjunctive} are also “counterfactually equivalent” in the sense that \ref{sd} and \ref{ds} are both true:
\begin{prop}
	\nitem \label{sd}
		If the tree had been dead on Sunday, it would have been either dead on Sunday and not alive on Saturday afternoon or dead on Sunday and alive on Saturday afternoon
	\nitem \label{ds}
		If the tree had been either dead on Sunday and not alive on Saturday afternoon or dead on Sunday and alive on Saturday afternoon, it would have been dead on Sunday.
\end{prop}
Fine develops a new kind of model-theoretic semantics for counterfactuals which is capable of validating these judgments. While we don't propose to work through the details of this semantics here, we do wish to draw attention to two of its key features. The first feature is that ‘If $A$ or $B$, $C$’ turn out to be logically equivalent to the conjunction of ‘If $A$, $C$’ and ‘If $B$, $C$’. Thus according to Fine, \ref{disjunctive} entails the obviously false \ref{bad}, whereas \ref{simple} does not:
\begin{prop}
	\nitem \label{bad}
		If the tree had been dead on Sunday and not alive on Saturday afternoon, it would have been alive on Saturday afternoon.
\end{prop}
Now, as we mentioned in the introduction (*PAGEREF*), we are in agreement with Fine to this extent: sentences of the form ‘If $A$ or $B$, $C$’ really do admit quite natural readings on which they are equivalent to ‘If $A$, $C$ and if $B$, $C$’. (We be returning to this phenomenon in chapter 5: our treatment of it will involve giving up the idea that the relevant conditionals can be characterised as predicating a relation of two propositions, and will appeal to context-sensitivity in a way that leaves CSO intact.)

The second feature of Fine's view that we want to note is the fact that the semantics allows us to classify each sentence (relative to a context) as either disjunctive or non-disjunctive. Disjunctions are always disjunctive; atomic sentences may or may not be disjunctive. (Fine is amenable to a dose of context-dependence here.) Given that he has this ideology at his disposal, it would thus be natural for Fine to accommodate the seeming goodness of the inferences whose validity we wanted to explain by appeal to CSO (or by appeal to CT and/or VLAS) by maintaining that these inference-rules are valid when restricted to conditionals with non-disjunctive antecedents and consequents. In any case, Fine does not present any cases that make trouble for this fallback position. Insofar as Fine's new machinery provides a diagnosis of what goes wrong with the CSO-based arguments we have been considering, the diagnosis will thus have to turn that some of the relevant sentences are disjunctive (in the relevant contexts). For example, it would be natural for Fine to say that the antecedent of \ref{weekendsaturday} is disjunctive:
\begin{prop}
	\nitem \label{weekendsaturday}
		If the tree had died during the weekend, it would have been alive on Saturday afternoon
\end{prop}
Disjunctive sentences behave in the semantics like explicit disjunctions. Thus if, for example, Fine said that the antecedent of \ref{weekendsaturday} was tantamount to the disjunction ‘Either the tree died on Saturday or the tree died on Sunday’, \ref{weekendsaturday} would turn out to be equivalent to \ref{satsatsunsat}:
\begin{prop}
	\nitem \label{satsatsunsat}
		If the tree had died on Saturday, it would have been alive on Saturday afternoon, and if the tree had died on Sunday, it would have been alive on Saturday afternoon.
\end{prop}
which sounds straightforwardly false.

On big problem we see with this account of \ref{weekendsaturday} is that, in the context it naturally evokes, \ref{weekendsaturday} doesn't seem to be straightforwardly true or straightforwardly false. In that context, it would be natural to say
\begin{prop}
	\nitem \label{mightormightnot}
		If the tree had been in pieces at some point during the weekend, it might or might not have been intact all day Saturday
\end{prop}
Moreover, there are other contexts where a sentence such as \ref{weekendsaturday} seems more or less straightforwardly true, even though Fine seems to predict that it's false. For example if there was no storm on Saturday but there was one on Sunday it would be quite natural to assert \ref{weekendsunday}:
\begin{prop}
	\nitem \label{weekendsunday}
		If the tree had died during the weekend, it would have died on Sunday.
\end{prop}
The problem is not essentially due to the postulated semantic equivalence of ‘The tree died during the weekend’ to ‘The tree died on Saturday or Sunday’, since the same problematic issues arise even when we substitute that explicit disjunction into \ref{weekendsaturday} and \ref{weekendsunday}.

Fine does have one other resource that he invokes in somewhat related contexts. He notices that sentences like
\begin{prop}
	\nitem \label{eitherwest}
		If I had moved to the West coast or the East coast, I would have moved to the West coast.
\end{prop}
can be uttered felicitously even though his semantics predicts that it entails the absurd-sounding
\begin{prop}
	\nitem \label{eastwest}
		If I had moved to the East coast, I would have moved to the West coast.
\end{prop}
What he says is this: while \ref{eastwest} is indeed an absurd thing to utter because of the presupposition of non vacuity, relative to the context in which \ref{eitherwest} is uttered felicitously, \ref{eastwest} is in fact not absurd but instead vacuously true. Fine could say something similar about the problem sentence \ref{satsunsat},
\begin{prop}
	\nitem \label{satsunsat}
		If it had died on Saturday or on Sunday, it would have died on Sunday.
\end{prop}
But this does not look like a promising suggestion in settings where we are inclined to make remarks like
\begin{prop}
	\nitem \label{satsunmight}
		If it had died on Saturday or on Sunday, it might or might not have died on Saturday
\end{prop}
Since the truth of \ref{satsunsat} turns on whether \ref{bad} is vacuously true, the kind of ignorance we are avowing in uttering \ref{satsunmight} would have, for Fine, to turn on on ignorance as regards whether \ref{bad} is vacuously true. We have trouble seeing how a plausible theory of vacuous truth could fit the bill.

It is time to turn briefly to an objection that we anticipate being directed against our view. Consider the sequence of conditionals
\begin{prop}
	\nitem 
		\begin{prop}
			\aitem
			If I had skipped breakfast on Monday, that would have been the first time I did so all year 
			\aitem[b] \ldots{} 
			\aitem[g] 
			If I had skipped breakfast on Sunday, that would have been the first time I did so all year.
		\end{prop}
\end{prop}
On our account, each is true in the context it naturally suggests, but there is no context where we can be confident that all of them are nonvacuously true. But note that there appears to be a reading of the quantified sentence \ref{quantified} on which it is true:
\begin{prop}
	\nitem \label{quantified}
		Each day last week is such that if I had skipped breakfast on that day, it would have been the first time I did so all year.
\end{prop}
It is hard to deny that this quantified claim is true. Relevantly similar claims occur in many other settings.
\begin{prop}
	\nitem 
		If any member of my family had voted Republican she would have been he only one to do so
	\nitem 
		I bought lottery tickets on several occasions last year. On each of those occasions, if I had won I would have more than quadrupled my wealth.
\end{prop}
It is easy for someone like Lewis to explain how such sentences could get be true in a single context: a single closeness relation along the lines he suggests can vindicate each witness to the quantified claim. But such sentences pose a challenge to our account, on which the relevant explanatory work is done by accessibility rather than closeness. For example, if for each day last week the set of accessible worlds contains a world where I skip breakfast on that day, it is hard to see what would stop there from being an accessible world where I skip breakfast on two days last week. And unless we impose special Lewis-style constraints on the closeness relation (something which we want to resist) there is no obvious ground for confidence that the closest world where I skip on a certain day isn't one where I have already skipped on some earlier day.

Our response to this objection draws on some quite pervasive facts about the interaction of quantification with context-dependence. The paradigm on which we think most context-sensitive words should be modelled is ‘local’. Sometimes, the semantic contribution of ‘local’ is a particular property, say the property of being in Brooklyn, or perhaps the property of being within a certain distance of Barack Obama. But on other occasions, ‘local’ behaves semantically as if it contained a variable bound by some higher quantifier. For example
\begin{prop}
	\nitem 
		Each of my brothers went to a local tattoo parlour
\end{prop}
can mean that each of my brothers went to a tattoo parlour \emph{in the neighbourhood he lives in}. Such “bindability” seems to be a feature of pretty much all context-sensitive expressions (with a few possible exceptions, such as ‘I’). For example ‘Each of my brothers is angry that we don't spend enough time together’ can mean ‘Each of my brothers is angry that \emph{he and I} don't spend enough time together’. ‘Everyone in my family is tall’ can mean ‘Everyone in my family is tall \emph{for his or her age}’. ‘Everyone is scared because there might be a spider in the closet’ can mean, roughly, ‘Everyone is scared because it is \emph{an epistemic possibility for them} that there is a spider in the closet’. And so forth. We intend ‘accessible’ to fit this pattern. So \ref{quantified} could, for example, mean something like
\begin{prop}
	\nitem 
		Each day last week is such that either there is no world that matches the history of the actualised world up to that day where I skip breakfast on that day, or the closest such world is one where that day is the first day on which I skip breakfast all year.
\end{prop}
This provides, in effect, a different accessibility relation for each day.

\begin{itemize}
	\item
	TODO: add a discussion of Morgenbesser's coin: ‘If I had bet on heads I would have won’. Return to (46) once we have said this.
	\begin{itemize} 
		\item
		One moral: the standard accessibility relations for counterfactuals is not a straightforwardly historical one.
		\item
		Tricky cases: when $A$ enhances the probability of $B$ but $B$ still fails to happen it's sometimes tempting think that if $A$ hadn't happened $B$ still wouldn't have happened.
		\item
		We could say a little bit on the question what this means for counterfactuals analyses of causation. Don't be too down on the idea that one could characterise what's held fixed in non-causal (probabilistic?) terms.
		\item
		When characterising what Morgenbesser cases show about holding fixed it's tempting to say certain things about causation which would force you into the “big sudden miracle” view. This is bad, so be careful.
		\item
		Also consider cases where we have false beliefs about causal relevance and irrelevance. Is the accessibility relation characterised in causal terms, or is it a matter of non-causal properties that speakers believe to have certain causal profiles.
		\begin{itemize} 
			\item
			Remember about small gravitational impact of betting.
		\end{itemize}
		\item
		Another kind of case that might illustrates the complexity of accessibility for counterfactuals: the comet example from CD's miracles paper. 
	\end{itemize}
	\item
	Make sure that we properly address the idea that holding history fixed could just mean holding macro history fixed.
	\item
	Circumstantial modals which stand to counterfactuals as epistemics stand to indicatives: ‘It couldn't have happened that\ldots{}’
	\item
	It would be nice to say something about why holding history fixed is the canonical way of doing things. Connection to deliberation? (Look at Luke Glynn recent paper?)
	\item
	Play up the unity of our account.
	\item
	Notice that we didn't say that it always just is epistemic possibility (for some relevant person/group/etc\ldots{}) We think it can be stronger.
	\item
	Examples of modals where it is stronger: constrained uses.
	\item
	Examples of constrained conditionals --- maybe just announce some without giving the probability-theoretic argument\ldots{}
	\item
	To do in next chapter: Talk about Matt Bird's example.
	\item
	Add a little bit on general anti-contextualist arguments about eavesdroppers, disagreement and all that. Point out that the things that people like Bennett (and Gibbard?) say about Sly Pete cases are just a special case of this general theme.
\end{itemize}

\end{document}
.
