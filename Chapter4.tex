%!TEX root=If.tex
\documentclass[If.tex]{subfiles}

\begin{document}
\chapter{Worlds}
\label{chap:worlds}

\section{Hyperintensionality}
\label{sect:hyperintensionality}
It is time to face up to a serious worry about the account of conditionals we have spent the last three chapters defending: namely, that because it makes use of the semantic framework of possible worlds, it does not allow for so-called \emph{hyperintensionality}: cases where a true conditional can be turned into a false one by substituting for the main clause or the subordinate clause another clause true in exactly the same possible worlds.  Since the use of possible worlds is more fully entrenched in the theory of counterfactuals, it is not surprising that this kind of worry has been most fully discussed in connection with counterfactuals.  Given our emphasis on the context-sensitivity of conditionals, some putative examples of hyperintensionality are easy for us to explain away.  For example, consider 
\begin{prop}
	\nitem 
	\begin{prop}
		\aitem \label{taller}
		If Tim Williamson was taller Kobe Bryant, they would both be extremely tall
		\aitem \label{shorter}
		If Kobe Bryant was shorter than Tim Williamson, they would both be extremely tall
	\end{prop}
\end{prop}
The first seems obviously true, the second obviously false.  But this is naturally accounted for simply by saying that the natural interpretations of \ref{taller} and \ref{shorter} involve different accessibility parameter---one where Bryant's height is held fixed in the case of \ref{taller}, one where Williamson's height is held fixed in the case of \ref{shorter}.  Of course, fleshing out this response would require saying more about why the differences in wording naturally trigger these different resolutions of context-sensitivity, but we won't get into that here.%
\footnote{***Discuss literature on topic}

However there are other examples in the literature that aren't easily explained away by appeal to context-dependent accessibility (or for that matter some context-dependent account of closeness) on possible worlds.  These primarily involve counterfactuals whose antecedents look to be metaphysically impossible.  Here are some pairs of such counterfactuals that seem to differ in truth value:
\begin{prop}
	\nitem
	\begin{prop}
		\aitem
		If Hobbes had squared the circle in 1650, cancer would have been eradicated by 1660 (cf. Nolan).
		\aitem
		If Hobbes had squared the circle and discovered a perfect cure for cancer in 1650, cancer would have been eradicated by 1660.  
	\end{prop}
	\nitem
	\begin{prop}
		\aitem
		If the oceans had been filled with water but not H$_2$O, the oceans would have been filled with water
		\aitem
		If the oceans had been filled with water but not H$_2$O, the oceans would have been filled with H$_2$O
	\end{prop}
\end{prop}
Assuming with orthodoxy that the antecedents here are in fact metaphysically impossible, our theory as it stands judges all four of these sentences to be (vacuously) true no matter how the context-sensitivity of the conditional is resolved.  And switching to a theory where conditionals whose antecedent is true at no accessible world are automatically false rather than automatically true obviously doesn't help.  More generally, insofar as one goes for a theory according to which substitution of intensionally equivalent clauses doesn't affect truth at a context, it is hard to think of a natural way to account for the data using the kind of appeal to context-sensitivity that we suggested for \ref{taller} and \ref{shorter}.  

If the entire case for hyperintensionality had to rest simply on examples like these, we would not be particularly moved: the examples seem a bit tangential to the overall practice of counterfactual reasoning, and an error-theoretic approach on which the relevant judgments arise from our incautious extension of generally reliable heuristics, as suggested by \citet{WilliamsonCounterpossibls}, seems quite tenable.  However, once we turn from counterfactuals to indicatives, examples motivating hyperintensionality are readier to hand:  
\begin{prop}
	\nitem \label{ripper}
	\begin{prop}
		\aitem
		If Jack the Ripper was Randolph Churchill, Jack the Ripper wrote \emph{Alice in Wonderland}
		\aitem
		If Jack the Ripper was Lewis Carroll, Jack the Ripper wrote \emph{Alice in Wonderland}
	\end{prop}
	\nitem 
	\begin{prop}
		\aitem
		If Jack the Ripper fathered Winston Churchill, Jack the Ripper was Randolph Churchill
		\aitem
		If Jack the Ripper fathered Winston Churchill, Jack the Ripper was Lewis Carroll
	\end{prop}
\end{prop}
For the purposes of these examples, suppose that neither Randolph Churchill nor Lewis Carroll, two Ripper candidates from the literature, was in fact Jack the Ripper.

Given the intimate connection between indicative conditionals and epistemic modals, this pattern of hyperintensionality-friendly judgments is no surprise.  Claims of hyperintensionality have ben widely accepted for attitude verbs like ‘know’ and ‘believe’, on the basis of examples like \chisholm{lois}{a--b}:
\begin{prop}
	\nitem \label{lois}
	\begin{prop}
		\aitem
		Lois knows that Superman flies
		\aitem
		Lois knows that Clark flies
	\end{prop}
\end{prop}
Such claims seem equally well-motivated for epistemic modals.  For example, consider the following pairs of candidate speeches by Lois:
\begin{prop}
	\nitem
	\begin{prop}
		\aitem
		Superman might not work as a reporter
		\aitem
		Clark might not work as a reporter
	\end{prop}
	\nitem
	\begin{prop}
		\aitem
		There must be some interesting secrets to be learnt about Superman
		\aitem
		There must be some interesting secrets to be learnt about Clark
	\end{prop}
\end{prop}
Even without a specific analysis of ‘must’ and ‘might’ in terms of ‘knows’ to hand, it would be bizarre to deny hyperintensionality for ‘must’ and ‘might’ while accepting some form of hyperintensionality for ‘know’ and ‘believe’.  But similarly, it would be bizarre to deny hyperintensionality for indicative conditionals while accepting it for ‘must’ and ‘might’, at least once was recognises the kinds of connections between them that we discussed in \autoref{chap:accessibility}.



*** 






Putative examples of hyperintensionality 
	(including *false* conditionals with impossible antecedents, which turn true when you replace the antecedent with the conjunction of the consequent with its own negation).  
	- with counterfactuals
			examples where it's merely a word-choice-influencing-context thing
				If I resembled Madonna… versus If Madonna resembled me…
				If I lived close to Madonna…
			
			standard examples
			
			If it were to turn out that Hesperus wasn't Phosphorus, then…
			If we discovered tomorrow that Hesperus wasn't Phosphorus, that would [not] be very surprising
			If we knew Superman wasn't Clark, we wouldn't be bothering to investigate this
			If we knew Superman wasn't Clark, we would be much less busy than we actually are
			If [theorist T] were right, then it would be that [impossible consequence of T]
			
	- with indicatives
			If there's water in the glass but no hydrogen, nothing interesting is going on
			If vixens aren't foxes, they aren't Martian spy robots either
			If I've been completely confused and really ‘and’ denotes logical disjunction, snow is white and snow is not white.
			If I've been completely confused about ‘not’ and really it's synonymous with ‘possibly’, snow is white and snow is not white.
			If the extension of ‘leg’ includes tails, then horses have five legs
			If Graham Priest is right, an arrow in flight is both moving and stationary

	- note that nonvacuity (hence hyperintensionality) is motivated by the Credence Equation
		whereas the Chance Equation does nothing comparable
		So, dealing with the problem for indicatives is far more theoretically pressing!
		
	- Reasons for regarding ‘knows’ as hyperintensional seem to carry over to epistemic modals; if there's an intimate connection between them and indicatives, that also suggests we should expect hyperintensionality.  
	
\section{Impossible worlds}
A possible solution: keep the form of analysis but have impossible worlds, perhaps constructed as sets of propositions.  
	- Obviously lots of the details here will depend on what proposiitons are.
	- One salient choice point for hyperintensional theory of propositions: is the proposition that Hesperus is bright = the proposition that Phosphorus is bright, etc.?  
	- Use “Millianism” for the view that says yes to this much.  

Developing this approach
	- Two kinds of reasons for thinking that worlds aren't closed under conjunction, etc.
		- far-out examples like the one above with ‘and’
		- examples motivated by Millianism.  
			- If people use ‘Hesperus’ to talk about a planet that's more massive than the one they use ‘Phosphorus’ to talk about, then Hesperus is 

What this approach does to the logic:
		Things like CC are threatened by worlds not closed under conjunction
			Similarly CEM is threatened by worlds where negation is ill-behaved
		But CSO is still guaranteed
		
Argument that Millians should reject CSO (for indicatives, when multiple guises are in play)

If the distinctness theorists are right and we're going to Hesperus we're going to a lovely planet where people have green skin
If we're going to a lovely planet where people have green skin, the distinctness theorists are right and we're going to Hesperus.
If the distinctness theorists are right and we're going to Phosphorus, we're going to a planet populated by nine foot tall monsters.
If we're going to a lovely planet where people have green skin, we're going to a horrible planet populated by nine foot tall monsters.

Why this isn't really dependent on the Millian  that CSO can fail (when multiple guises are in play)

You know (under one guise) that Paderewski is rich and (other another) that he doesn't work. 
You know that Society S would never allow someone rich who doesn't work to be the guest of honour.

We want all of these to be true: 

If Paderewski is the guest of honor they are having a rich guest of honour
If they are having a rich guest of honour Paderewski is the guest of honour
If Paderewski is the guest of honor they are having a guest of honour who doesn't work

but not

If they are having a rich guest of honour they are having a guest of honour who doesn't work

given what we know about their policies


\section{Epistemic modals and indicative conditionals as guise-sensitive}
Existential quantification over guises in attitude reports 
	- choice point: occurrence-by-occurrence (Salmon, Soames) vs type-by-type (Kaplan, etc.)

Toy models: 
	“Descriptivist”/“Intensionalist” implementation
		for each psycholoogical verb we have an underlying relation to sets of worlds
		A “guise” of $x$ is a property necessarily instantiated by at most one thing, that $x$ instantiates
	Mentalese implementation
		A “guise” of $x$ is a property of Mentalese words

The obvious extension of this to ‘might’ and ‘must’

Why existential rather than universal?  

Extending to ‘if’

Intensionalist version: the closeness relation is on worlds - maybe just the same closeness relation we use for counterfactuals.  


\section{Why it's often OK to ignore all of this and just work with the closest-worlds analysis for indicatives and the quantification-over-worlds analysis of ‘might’ and ‘must’}



\section{Counterfactuals too?}
More examples (discussed on Friday) where counterfactuals seem similar to indicatives (on the score of hyperintensionality etc.)

(An alien who is allergic to H2O and has a detector...)
If there hadn't been H2O in the pool I'd have plunged right in
If there hadn't been water in the pool I'd have plunged right in

Also relevant: cfs that might just be categorised as ‘backtrackers’.  

If you had smoked, I'd have known immediately that you had the smoking gene

Of course we can still say that accessibility for counterfactuals is *often* limited to the metaphysically possible worlds (typically as part of being limited much more narrowly to nomically possible worlds matching history up to a time).  

Q: why isn't accessibility for indicatives limited like this?  One possible suggestion: we know that things are the way they actually are, so the only way to narrow in to a set of metaphysically possible worlds within the set of epistemically possible worlds is to narrow down to one (which is pointless).  (Or do the same thing with any contingent a priori knowledge.)



\section{What becomes of all the stuff from the previous chapter in this setting?}

A speech we'd like to have a salient true interpretation:

For any rational person x such that
	x has credence 50\% that Superman = Clark
	x has credence 1\% that the telescope is working and Superman = Clark
	x has credence 10\% that the telescope is working

then x has credence 10\% that if the telescope is working Superman = Clark





\end{document}