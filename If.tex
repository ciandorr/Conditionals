%!TEX TS-program = pdflatex
\documentclass[leqno, 11pt, a5paper, openany]{book}
\usepackage[a5paper]{mybiblatexstyle}
\usepackage{appendix}
\usepackage{aliascnt}
%\usepackage{tensor}
\usepackage{mathtools}
\usepackage{array}
\usepackage{subfiles}
\usepackage{longtable}
\usepackage{pfbreaks}
\usepackage{tikz}
\usepackage[linguistics]{forest}

\newcolumntype{N}{>{$}r<{$}}
\newcolumntype{E}{>{$}c<{$}}
\newcolumntype{M}{>{$}l<{$}}

\setlength{\proprightmargin}{0em}

\bibliography{Shoulders}

\newcommand{\argument}[1]{#1.}
\newcommand{\argbreak}{. }

\setlength{\bibitemsep}{0em}

% \let\OLDproof\proof
% \let\OLDendproof\endproof
% \renewenvironment{proof}[1][Proof]{\small\OLDproof[#1]}{\OLDendproof}

\newcommand{\page}[1]{p.~#1}
\newcommand{\pages}[1]{pp.~#1}

\newcommand{\negate}[1]{\overline{#1}}
\DeclareMathOperator{\prob}{Pr}
\DeclareMathOperator{\prior}{Cr}

\DeclareItemType{long}{left}{dummy}{\textsc{\MakeLowercase{\itemtext}}}{\textsc{\MakeLowercase{\reftext}}}
\DeclareNumProp[equation][\alph{myalph}.][\paren{\theequation\alph{myalph}}]{myalph}

\newcommand{\aitem}{\xitem{myalph}}

\newtheorem{theorem}{Theorem}
\newtheorem{lemma}[theorem]{Lemma}
\newtheorem{definition}[theorem]{Definition}


\newpagestyle{front}{
\sethead[\usepage][][]
{}{\textsl{\chaptertitle}}{\usepage}
}	
\newpagestyle{main}{
\sethead[\usepage][\textsl{\sectiontitle}][]
{}{\thechapter. \textsl{\chaptertitle}}{\usepage}
}
\newpagestyle{biblio}{
\sethead[\usepage][\textsl{Bibliography}][]
{}{\textsl{Bibliography}}{\usepage}
}


\renewcommand{\theequation}{\arabic{equation}}

\newcommand{\gap}{[\ldots]}%\begin{center}[\ldots]\end{center}}

\begin{document}
\frontmatter
\pagestyle{front}
\title{If…: A Theory of Conditionals}
\author{\textsc{Cian Dorr and John Hawthorne}}
\date{Draft of \today\\
$\quad$\\
Note for Mind and Language seminar participants: some apologies are in order, for (i) the fact that this is in pretty rough shape; (ii) the fact that it is not so directly about ‘semantic frameworks’, and (iii) the fact that there is so much of it.  We expect to focus on Chapter 1; parts of Chapter 2 are included for context.}

$\quad$\pagebreak

\maketitle

\tableofcontents

\mainmatter
\pagestyle{main}

\begin{comment}
\subfile{logic}
\end{comment}

\subfile{Chapter0}

\subfile{Chapter1}

\subfile{Chapter2}

\begin{comment}
\subfile{Chapter3}

\subfile{Chapter4}
\end{comment}

\printbibliography[heading=bibintoc]

\end{document}