\documentclass[If.tex]{subfiles}

\begin{document}

%\newcommand{\IF}[2]{\text{If }#1, #2}
\newcommand{\IF}[2]{#1\to#2}
\newcommand{\A}{A}
\newcommand{\B}{B}
\newcommand{\C}{C}
\newcommand{\D}{D}

\chapter*{Table of logical principles}\addcontentsline{toc}{chapter}{Table of logical principles} 

\setlength{\tabcolsep}{2pt}
\begin{longtable}{r@{\hskip 5pt}NEM}
\multicolumn{4}{l}{\emph{Metarules:}} \\
RCEA* %(Substitutivity in antecedents) 
	& \multicolumn{3}{l}{If  $⊦\A ≡ \B$ then $⊦(\IF{\A}{\C} ⊃ \IF{\B}{\C})$} \\
???*
	& \multicolumn{3}{l}{If $⊦\B ⊃ \C$ then $⊦(\IF{\A}{\B} ⊃ \IF{\A}{\C})$} \\
RN*
	& \multicolumn{3}{l}{If $⊦\B$ then $⊦\IF{\A}{\B}$} \\
???*
	& \multicolumn{3}{l}{If $⊦\A$ then $⊦\IF{¬\A}{\B}$} \\
\\
\multicolumn{4}{l}{\emph{Least controversial principles:}} \\
CC* %(‘Conditional Conjunction’) 
	& (\IF{\A}{\B}) ∧ (\IF{\A}{\C}) &⊃& (\IF{\A}{(\B∧\C)}) \\
ID** %(‘Identity’) 
	& A &\to& A \\ %\IF{\A}{\A} \\ 
MP** %(‘Modus Ponens’) 
	& (\IF{\A}{\B}) &⊃& (\A ⊃ \B) \\
CK* 
	& (\IF{\A}{(\B ⊃ \C)}) &⊃& (\IF{\A}{\B} ⊃ \IF{\A}{\C}) \\
\\
\multicolumn{4}{l}{\emph{Antecedent equivalence principles:}} \\
CSO** 
	& ((\IF{\A}{\B}) ∧ (\IF{\B}{\A}) ∧ (\IF{\A}{\C})) &⊃& (\IF{\B}{\C}) \\
RCV* %(‘Very Limited Antecedent Strengthening’) 
	& (\IF{\A}{\B}) ∧ (\IF{\A}{\C}) &⊃& (\IF{(\A ∧ \B)}{\C}) \\
RT* %(‘Cumulative Transitivity’) 
	& (\IF{\A}{\B}) ∧ (\IF{(\A∧\B)}{\C}) &⊃& (\IF{\A}{\C}) \\
CA* %(‘Disjunction’) 
	& (\IF{\A}{\C}) ∧ (\IF{\B}{\C}) &⊃& (\IF{(\A ∨ \B)}{\C}) \\
RCA* 
	& (\IF{(\A ∨ \B)}{\C}) &⊃& ((\IF{\A}{\C}) ∨ (\IF{\B}{\C})) \\
Cases* 
	& (\IF{(\A∧\B)}{\C}) ∧ (\IF{(\A ∧ ¬\B)}{\C}) &⊃& (\IF{\A}{\C})\\
CV %(‘Limited Antecedent Strengthening’) 
	& ¬(\IF{\A}{¬\B}) ∧ (\IF{\A}{\C}) &⊃& (\IF{(\A ∧ \B)}{\C}) \\
\\
\multicolumn{4}{l}{\emph{Principles valid for strict conditionals:}} \\
AS 
	& (\IF{\A}{\C}) &⊃& (\IF{(\A ∧ \B)}{\C}) \\
Contraposition 
	& (\IF{\A}{\B}) &⊃& (\IF{¬\B}{¬\A}) \\
Transitivity
	& (\IF{\A}{\B}) ∧ (\IF{\B}{\C}) &⊃& (\IF{\A}{\C}) \\
Simplification
	& (\IF{(\A ∨ \B)}{\C}) &⊃& (\IF{\A}{\C}) \\
\\
\multicolumn{4}{l}{\emph{Principles valid for material conditionals:}} \\
Or-to-if 
	& (\A ∨ \B) &⊃& (\IF{¬\A}{\B}) \\
Import-Export 
	& (\IF{\A}{(\IF{\B}{\C})}) &≡& (\IF{(\A ∧ \B)}{\C}) \\
First ‘paradox’
	& ¬\A &⊃& (\IF{\A}{\B}) \\
Second ‘paradox’ 
	& \B &⊃& (\IF{\A}{\B}) \\
Third ‘paradox’ 
	& (\IF{\A}{\B}) &∨& (\IF{\B}{\A}) \\
\\
\multicolumn{4}{l}{\emph{Other additions:}} \\
CEM* 
	& (\IF{\A}{\B}) &∨& (\IF{\A}{¬\B}) \\
And-to-if (CS)** 
	& (\A ∧ \B) &⊃& (\IF{\A}{\B}) \\
RIE 
	& (\IF{\A}{(\IF{(\A ∧ \B)}{\C})}) &≡& \IF{(\A ∧ \B)}{\C} \\
CNC
	& ¬((\IF{\A}{\B}) &∧& (\IF{\A}{¬\B}))
\end{longtable}

\emph{Principles about the interaction of conditionals with modals:}
\begin{center}
\begin{tabular}{r@{\hskip 5pt}NEM}
MOD* 
	& □\A &⊃& (\IF{\B}{\A}) \\
C0* 
	& □(\A ≡ \B) &⊃& ((\IF{A}{C}) ≡ (\IF{B}{C})) \\
CT*
	& (\IF{¬\A}{¬\B}) &⊃& (□\B ⊃ □\A) \\
CAS*
	& □(\A∧\B) &⊃& □\A \\
CRCA*
	& □(\A∧\B) &⊃& □\A ∨ □\B\\
CRCV
	& (\IF{¬\A}{¬\B}) &⊃& □\A ⊃ □(\A∨\B) \\
CRT
	& (\IF{¬\A}{¬\B}) &⊃& □(\A∨\B) ⊃ □\A \\
CCA
	& □\A ∧ □\B &⊃& □(\A ∧ \B)\\
4
	& □\A &⊃& □□\A \\
B 
	& ¬\A &⊃& □¬□\A
\end{tabular}
\end{center}
% MOD*
% 	& (\IF{\A}{⊥}) &⊃& (\IF{\B}{¬\A}) \\
% C0*
% 	& (\IF{A}{⊥}) &⊃& ((\IF{B}{C}) ≡ \IF{(A∨B)}{C}) \\
% CT*
% 	& (\IF{A}{B}) ∧ (\IF{B}{⊥}) &⊃& (\IF{A}{⊥}) \\
% CAS*
% 	& (\IF{A}{⊥}) &⊃& (\IF{(A∧B)}{⊥}) \\
% CRCA*
% 	& (\IF{(\A ∨ \B)}{⊥}) &⊃& ((\IF{\A}{⊥}) ∨ (\IF{\B}{⊥})) \\
% C4
% 	& (\IF{\A}{⊥}) &⊃& (\IF{\B}{(\IF{\A}{⊥})})\\
% CB
% 	& A &⊃& \IF{(\IF{\A}{⊥})}{⊥}
We can consider the above principles on all manner of different interpretations of $□$: we indicate which interpretation is in question by adding a subscript to the name of the principle.  Of particular interest is the operator $⊡$ defined in terms of the conditional as $⊡\A \equalsdef (\IF{¬\A}{⊥})$.  


Principles marked ** are guaranteed simply by the form of our favoured analysis, together with the claim that the worlds are well-ordered by ‘closer than’ with the actualised world in first place, and that $\A$ is true at the actualised world iff $\A$.  

Principles marked * are guaranteed by the form of our favoured analysis, together with the principles capturing the following standard role for truth at a world: $\A ∧ \B$ is true at a world iff $\A$ and $\B$ both are, $\A ∨ \B$ is true at a world iff either $\A$ or $\B$ is, and $¬\A$ is true at a world if $\A$ is not.  (Chapter 4 takes seriously the idea that there are impossible worlds that do not obey these principles.)

\end{document}