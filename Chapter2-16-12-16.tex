%!TEX TS-program = xelatex
\documentclass[leqno, 11pt, a5paper, openany]{article}
\usepackage[a5paper]{myxitsstyle}
\usepackage{appendix}
\usepackage{aliascnt}
%\usepackage{tensor}
\usepackage{mathtools}
\usepackage{subfiles}

\setlength{\proprightmargin}{0em}

\bibliography{Shoulders}

\setlength{\bibitemsep}{0em}

% \let\OLDproof\proof
% \let\OLDendproof\endproof
% \renewenvironment{proof}[1][Proof]{\small\OLDproof[#1]}{\OLDendproof}

\newcommand{\page}[1]{p.~#1}
\newcommand{\pages}[1]{pp.~#1}

\newcommand{\negate}[1]{\overline{#1}}
\DeclareMathOperator{\prob}{Pr}
\DeclareMathOperator{\prior}{Cr}


\begin{document}
\title{Conditionals book, Chapter 2: Closeness}
\author{Cian Dorr and John Hawthorne}
\date{Draft of \today}

\maketitle

\begin{enumerate}
\item
  Defence of CEM.

  \begin{itemize}
    \item
    Credence and probability judgments for subjunctives
  \item
    Credence and probability judgments for indicatives
  \end{itemize}
\end{enumerate}

\begin{itemize}
\item
  Other arguments for CEM

  \begin{itemize}
    \item
    ‘Everyone failed if he goofed off’
  \item
    Neg-raising and small differences?\\
  \item
    The argument from ‘only if’.
  \end{itemize}
\item
  Targets in this section: strictists and tie-lovers. Possible response
  from them to consider: presupposition of homogeneity. This seems crap:
  how would it even work for the coin cases?

  \begin{itemize}
    \item
    The super-contextualist response. It has to posit lots of
    mid-sentence context-shifts in cases where it seems really
    implausible.\\
  \item
    Just a footnote on the terrible Reverse Sobel Sequence argument for
    strictism - appeal to Moss.

    \begin{itemize}
        \item
      The deepest considerations in favour of strictism come from the
      desire to have a uniform treatment of propositional conditionals
      and conditionals with adverbs of quantification. We'll get to this
      in a later chapter.\\
    \item
      Does Kratzer have any arguments that the hidden operator in
      indicatives is Must-like? One could make an argument from the
      premise that we never see a \emph{lexicalised} modal that is
      epistemic but has the kind of ordering source feel that would be
      required by a CEM lover.\\
    \end{itemize}
  \end{itemize}
\end{itemize}

\begin{enumerate}
\item
  Response to worries about CEM.

  \begin{itemize}
    \item
    First worry: how could there fail to be ties of similarity?

    \begin{itemize}
        \item
      Response: Forget about similarity! A similarity based theory of
      closeness won't get the kinds of probabilistic judgments we want
    \end{itemize}
  \item
    Worries about “metaphysical magic”

    \begin{itemize}
        \item
      get on the table our methodological perspective about such
      worries.
    \end{itemize}
  \item
    Is the appeal to vagueness enough to defuse all the worries?

    \begin{itemize}
        \item
      No. On some views of vagueness, we shouldn't have the relevant
      intermediate credences when we are thinking that they are vague.\\
    \item
      Further worry: if there is all this vagueness it looks to infect
      even the very good conditionals which we seemingly can know and
      assert (in ways analogous to how we can know and assert some
      things about the future that have high chances short of 1.)
    \end{itemize}
  \item
    Mention Jeremy's argument from CEM to non-physicalism etc. (Perhaps
    just briefly discuss that here.)
  \item
    Duality arguments against CEM.
  \end{itemize}
\item
  Epistemological profile of closeness
\end{enumerate}

\begin{itemize}
\item
  Start with a picture thinking version which deals with zero-degree
  conditionals.

  \begin{itemize}
    \item
    Subtleties with embedded things.
  \end{itemize}
\end{itemize}

\begin{enumerate}
\item
  explain the van Fraasseny thing we are doing and say to what extent it
  gets you Adam's thesis.
\end{enumerate}

\begin{itemize}
\item
  Skyrms's thesis: the chance of the counterfactual = the conditional
  chance

  \begin{itemize}
    \item
    Explain why similarity can't have anything to do with it
  \end{itemize}
\end{itemize}

\begin{enumerate}
\item
  Explain how we deal with counterexamples to Adams: McGee, Kaufmann,
  \ldots{}
\item
  Why it's OK to only have the restricted version: conditionals in
  antecedents are normally constrained.
\item
  Why {[}instances of{]} CEM fall out of Adams thesis.\\
\item
  Opponents to address in this chapter: 5.1 People who do without
  closeness altogether---von Fintel, Gillies. 5.1.1 SM has already
  demolished their arguments about Sobel sequences.\\
  5.2 Real Hajek --- falsity all over the place 5.3 Epistemological
  version of Hajek --- lack of knowledge blocks assertability 5.3.1 for
  counterfactuals, let everything be exactly as it is in the case of the
  future. 5.4 Kratzer's vision where ‘if’ clause directly restricts the
  attitude verb. 5.5 5.6 Andrew's stuff where you keep the full version
  of Adam's thesis
\end{enumerate}

\begin{itemize}
\item
  Matt Bird's thing about connections. If (if you had flipped it it
  would have landed tails) then (if you had flipped it and bet on tails,
  you would have won).
\end{itemize}

\section{\texorpdfstring{Why
‘Closest’?}{Why Closest?}}\label{why-closest}

Recall that our bare-bones theory of conditionals is as follows:
\begin{prop}
\litem[CLOSEST] \label{closest}
	A conditional with antecedent p and consequent q is true iff either there is no accessible p-world, or the closest accessible p-world is a q-world.
\end{prop}
In saying this we are taking it for granted that where there are accessible p-worlds, there is a unique closest one. The aim of the present chapter is to say more about the relevant notion of closeness, and to develop and justify the required uniqueness assumption.

Some theories of conditionals use the ideology of worlds and closeness but make no assumption that there is always a unique closest accessible p-world provided there is any accessible p-world. The most influential such theorist is David Lewis \citeyearpar{LewisCounterfactuals}. Lewis's theory allows for two kinds of failures of uniqueness. First, there can be ties: there may be several maximally close accessible p-worlds (i.e.\ equally close p-worlds such that no p-worlds are closer than them). Second, there may be no maximally close accessible p-worlds: it could be that for every accessible p-world, there is a yet closer accessible p-world. Lewis's truth-conditions agree with ours in the case where there is a unique closest accessible p-world, but introduce a new element of universal quantification to deal with the other cases. When there are several maximally close accessible p-worlds, the conditional is true just in case all of them are q-worlds. When there are accessible p-worlds but no maximally close accessible p-worlds, the conditional is true just in case there is some accessible p-world such that every accessible p-world at least as close as it is a q-world. In fact the latter case also covers the case where there is one or more maximally close p-worlds; so Lewis's theory can be stated as follows:
\begin{prop}
\litem[LEWIS] \label{lewis}
	A conditional with antecedent p and consequent q is true iff either there is no accessible p-world, or there is an accessible p-world such that every accessible p-world that is at least as close as that world is a q-world.%
	\footnote{Lewis avoids the need to say ‘accessible’ all the time by setting things up in such a way any accessible world is closer than all inaccessible worlds (if there are any inaccessible worlds). But this is not essential to the aspect of Lewis's theory that we are presently concerned with.} 
\end{prop}
Lewis himself only applied this schema to the analysis of counterfactual conditionals, but it is certainly worth exploring whether a theory with this shape could also work for indicatives.

Another important family of rival theories of conditionals makes use of the ideology of worlds and accessibility, but does not appeal to closeness at all in the specification of truth-conditions. On these ‘strict conditional’ accounts, the truth-conditions of conditionals are straightforward:
\begin{prop}
\litem[STRICT] \label{strict}
	A conditional with antecedent p and consequent q is true iff every accessible p-world is a q-world.
\end{prop}
(One might be tempted to think that at the schematic level we are
presently working at, \ref{strict} is perfectly compatible with our account:
if we redefine ‘accessible world’ to mean what we previously meant by
‘closest accessible world’, won't \ref{strict} then be just a terminological
variant on \ref{closest}? No: in the expression ‘the closest accessible
p-world’, ‘closest’ is not playing the role of a monadic predicate of
worlds. Being a closest accessible p-world is not just being (a)
closest, (b) accessible, and (c) a p-world, just as the property of
being a shortest spy is not the conjunction of the property of being
shortest and the property of being a spy. Being a shortest spy is being
a spy who is at least as short as any spy; being a closest accessible
p-world is being a p-world that is at least as close as any accessible
p-world.)

Lewis's argument against \ref{strict} (for counterfactual conditionals) is
well-known and prima facie compelling. Lewis notes that \ref{strict} validates
the rule of antecedent strengthening: whenever ‘If P, Q’ is true, ‘If P
and R, Q’ for any R. But this rule does not look to be valid. For example, the inference
from
\begin{prop}
	\nitem \label{dog}
	If I bought my son a pet dog, he would be delighted
\end{prop}
to
\begin{prop}
	\nitem \label{strangle}
	If I bought my son a pet dog and strangled it, he would be delighted
\end{prop}
seems invalid. Or what comes to much the same thing: the conjunction of the first with the denial of the second seems perfectly consistent and felicitous. We note that exactly the same kind of argument can be given for indicative conditionals: for example, the inference from \ref{idog} to \ref{istrangle} seems just as terrible:
\begin{prop}
	\nitem \label{idog}
	If he bought his son a pet dog, his son was delighted
	\nitem \label{istrangle}
	If he bought his son a pet dog and strangled it, his son was delighted.
\end{prop}

To keep \ref{strict} going in the face of these prima facie compelling considerations against it, its proponents have posited widespread context-shift as regards what counts as accessible. Their thought (\emph{cite von Fintel}) is that in the context in which (1) is uttered truly, worlds where the speaker buys a pet dog and strangles it are inaccessible; but because of the presupposition of nonvacuity, uttering (2) triggers a new context in which at least one such world is accessible. The central mechanism that triggers contextual shifts is thus held to be that of presupposition accommodation. On this view, \emph{Antecedent Strengthening} is valid in the sense that it preserves truth when context is held fixed; but realistic cases where the putative counterexample sequences are uttered are not cases where context is held fixed.%
\footnote{Note that if one allowed context-shift to run completely rampant, one could even reconcile the truth-values delivered by \ref{strict} with those delivered by \ref{closest}. The idea would be that at the context where a conditional is uttered, the set of accessible worlds (in the sense relevant to \ref{strict}) are all and only those accessible worlds (in the sense relevant to \ref{closest}) that are at least as close as every accessible world where the antecedent is true. Unlike a version of \ref{strict} that tries to try its account of context-shift to the familiar phenomenon of presupposition accommodation, this kind of mad-dog contextualism seems to depart so much from the standard way of thinking about what it means for distinct utterances to belong to the same context that evaluating it would require getting a lot clearer about what theoretical role is being assigned to the new conception of ‘context’.} 
Since the views about accessibility we developed in chapter 1 also involve quite an extensive amount of context-shift, including context-shift driven by presupposition accommodation, we are in no position to discount \ref{strict} simply on account of the contextualism it requires.%
\footnote{--- One might think that the contextualist diagnosis of apparent failures of Antecedent Strengthening can't work, on the grounds that presupposition accomodation is a conversational phenomenon and we can consider the relevant arguments and evaluate their validity just in our own inner monologue, divorced from any conversational context. We think this is far too sanguine. \textbf{Say more here.}}

\begin{itemize}
\item
  Also mention Contraposition and Transitivity.
\end{itemize}

One feature that distinguishes \ref{closest} from both \ref{strict} and \ref{lewis} is
that status of the principle of Conditional Excluded Middle:
\begin{prop}
	\litem[CEM] \label{CEM}
	Either if P, Q or if P, not-Q.
\end{prop}
Given \ref{closest}, instances of this schema will be true independent of context, assuming that every world is either a Q-world or a not-Q world. (This assumption will be revisited in chapter 3, but we will here take it for granted.) By contrast, \ref{strict} obviously allows for interpretations of instances of CEM on which they fail to be true, because the set of accessible worlds includes both P-and-Q worlds and P-and-not-Q worlds. Likewise, \ref{lewis} allows for two kinds of failures of instances of CEM. In one kind of case, there are several P-worlds tied for maximal closeness, which differ with regard to the truth-value of Q; in the other kind of case, there are no maximally close P-worlds, and for every P-and-Q world there is a closer P-and-not-Q world, and for every P-and-not-Q world there is a closer P-and-Q world.

Proponents of \ref{strict} and \ref{lewis} are well aware of this feature of their views; indeed their views have been partly motivated by the desire for CEM to come out invalid. However, we think there are strong reasons to like CEM. We will consider several such reasons in the present section; in section 2 we will address some arguments against CEM.

The most central considerations for us turn on facts about the chances of conditionals, and the levels of confidence we should have in conditionals. We are going to look at some instances of CEM involving fair coins, which seem like good test cases: if there were there were a problem for CEM, one would expect it to show up for conditionals concerning the outcomes of fair coin tosses.

Let's begin with chance (understood as objective rather than epistemic). Consider one of the coins currently in your pocket. What's the chance that it would land Heads if it were tossed in the next minute? Around 50\%, surely (unless you are in the habit of carrying around trick coins). Similarly, the chance that it would fail to land Heads if it were tossed is around 50\%. But chance is a kind of probability, and it a basic theorem of the probability calculus that the sum of the probabilities of two propositions equals the sum of the probabilities of their disjunction and their conjunction. And in this case, the conjunction of the two conditionals---namely that if the coin were tossed it would land Heads, and if the coin were tossed it would fail to land Heads---is absurd, and deserves zero credence. So the chance of the disjunction of the conditionals---which is an instance of CEM---is roughly one. In fact, it would seem to be \emph{exactly} one, since any surprising factors that might elevate the chance of one disjunct above 50\% would presumably reduce the chance of the other disjunct below 50\% by the same amount. But given that the disjunction has chance one, it would be preposterous to deny that it is true.%
\footnote{We wouldn't quite want to assume that \emph{all} chance-one propositions are true: certain examples involving infinitely fine-grained outcomes make trouble for that simple generalisation. Similar points apply to arguments for the truth of a proposition from the premise that its epistemic chance is 1, or for the premise that we ought to assign it a credence of 1. But it seems hopeless to try to leverage these considerations into a defence of the denial of CEM.}

*** Think somewhere about cases like ‘What is the chance that what this piece of paper says is true?’ when it's a counterfactual.  It actually seems fine - and that's a good argument against NTV style theorists, Kratzerians, etc.  

We can generate a similar pattern of judgments by considering the chances at a certain past time of counterfactuals concerning times in the future of that time. For example, if you didn't toss a certain coin yesterday, it seems that the chance two days ago that it would land Heads if it were tossed yesterday was in the region of 50\%, as was the chance two days ago that it would fail to land Heads if it were tossed yesterday, so again, the chance of the disjunction of these counterfactuals would seem to have been 1.

Another way to motivate CEM for counterfactuals is to consider the levels of confidence that seem to be appropriate. For example, concerning a certain untossed coin which you have no special reason to suspect of being a trick coin, it seems that you should be about 50\% confident that it would have landed Heads if you had tossed it, and about 50\% confident that it would have failed to land Heads if you had tossed it. After all, discovering that the coin and the table it is being tossed on are magnetised in such a way as to favour Heads looks like it should make you more confident that the coin would have landed Heads if you had tossed it. Discovering a setup of magnets that favours Tails would make you less confident. Without evidence for any such setups, it seems obvious that your credence should be middling. And for exactly the same reasons, it looks like you should be about 50\% confident that the coin would have failed to land Heads if you had tossed it. Since you should be certain that the conjunction of these counterfactuals is false, and since rational levels of credence conform to the probability calculus, your level of confidence in the disjunction---an instance of CEM---had better be about 1.  Indeed, it seems that it should be exactly one, since any reasons for raising your credence in one of the disjuncts above 50\% seems like an equally good reason for lowering your credence in the other disjunct by the same amount.%
\footnote{A closely related argument for CEM appeals to judgments about epistemic probability rather than about appropriate levels of confidence, e.g.~that the epistemic probability that the coin would have landed Heads if it were tossed is about 50\%.}

These ways of assigning credences to counterfactuals can be further supported by considering the motivating role of counterfactuals in deliberation. Suppose that you face an uncomfortable choice between opening two boxes. You are 49\% confident that Box A contains a bomb primed to explode on opening the box, and 51\% confident that it contains nothing. You know that Box B contains a bomb linked up to a fair coin: if the box is opened, the coin will be tossed and the bomb will explode if it lands Heads. Obviously you should open Box A here. And the obvious explanation of this is that you are more confident that you would be killed if you opened Box B than that you would be killed if you opened Box A. But your credence that you would be killed if you opened Box A equals your credence that there is a bomb in Box A, namely 49\%. So your credence that you would be killed if you opened Box B, which is equal to your credence that the coin in Box B would land Heads if it were tossed, looks to be over 49\%--in fact, 50\%. Standard CEM-deniers, by contrast, will think that insofar as you are confident that you won't take Box B, and hence confident that the counterfactual ‘You would be killed if you opened Box B’ has a false antecedent, your credence in that counterfactual should be very low. They will need some other, more complicated and (we think) less natural way of relating credences in counterfactuals to good deliberation.

These natural confidence-theoretic judgments are very much out of line with the sorts of credences that seem to be recommended by standard versions of \ref{lewis} and \ref{strict}. For Lewis, cases involving chancy processes like coin-tossings are a primary motivation for allowing ties in the closeness ordering; given his actual theory of closeness, when we know that a coin is fair and was not tossed, we can be quite confident that the set of maximally close tossing worlds contain a mix of Heads and Tails worlds, and hence quite confident that each conditional are false. (Even if we aren't sure whether the coin was tossed or not, there will be some significant portion of our probability space devoted to the hypothesis that it was not tossed and that both conditionals are false because of ties.) Similarly, extant versions of \ref{strict} tend to say things about accessibility that encourage the idea that the accessible worlds in this case include both Heads-landing and Tails-landing worlds, in which case both conditionals will come out false according to \ref{strict}. We will certainly get this result if we interpret \ref{strict} using anything like the conception of accessibility described in chapter 1. One our claims in that chapter is that the same notion of accessibility relevant to counterfactuals can also be picked up by certain modals like ‘has to’ and ‘might have’; and obviously, given that the coin might have been tossed, it might have landed Heads, and might have landed Tails, and didn't have to land Heads, and didn't have to land Tails.%
\footnote{Of course, proponents of \ref{strict} might attempt to preserve our motivating confidence judgments by introducing some devious new conception of accessibility under which, in the case of an untossed fair coin, the accessible worlds where the coin is tossed are all alike as regards how it lands, and our 50\% confidence reflects our uncertainty about which worlds are accessible. We will briefly consider this view later. - One point to note: given the kind of contextual shiftiness that such a view would require, we can't be thinking that accessibility is ‘sticky’ in the way required by the von Fintel/Gillies account of Reverse Sobel Sequences. We also can't assimilate accessibility restriction to the broad model of quantifier domain restriction in general, where the stickiness phenomenon is clearly a real thing.}

We find a rather similar pattern of confidence-theoretic judgments when we turn from counterfactuals to indicative conditionals. Here the prima facie force of the relevant confidence-theoretic judgments has been very widely acknowledged. For example, if you are not sure whether a certain coin was tossed yesterday, and have no special evidence favouring the hypothesis that it was tossed and landed Heads over the hypothesis that it was tossed and landed Tails, it seems that you should be about 50\% confident that it landed Heads if it was tossed and about 50\% confident that it landed Tails if was tossed, and hence---since you should be about 0\% confident that it both landed Heads and landed Tails if it was tossed---you should be about 100\% confident that either it landed Heads if it was tossed, or it landed Tails if it was tossed. So the confidence-theoretic case for CEM looks at least as forceful for
indicatives as for counterfactuals.%
\footnote{We haven't included an argument based on the objective chances of indicative conditionals, because it's not so easy to find sentences where it's clear both the relevant notion of chance is objective and that the conditionals are genuine indicatives---recall that we have suspended judgment on the status of ‘does-will’ conditionals. One could however try to make something of examples like `There's a fifty-fifty chance that if he tosses this coin during the next hour, he wins a prize'.}

While many opponents of CEM seem by our lights to have let their theory ride roughshod over their natural confidence-assignments, some of them have shown enough awareness of the ordinary practice to want to explain away data of the sort we have been relying on. Here is Jonathan Bennett:
\begin{quote}
	Admittedly, we often find it natural to say things like `There's only a small chance that if he had entered the lottery he would have won', and `It's 50\% likely that if he had tossed the coin it would have come down heads'. In remarks like these, the speaker means something of the form $A>(P(C)=n)$---if the antecedent were true, the consequent would have a certain probability; yet the sentence he utters means something of the form $P(A>C)=n$\ldots{}. When we use one to mean the other, we employ a usage that is idiomatic but not strictly correct. (Bennett \page{251}.) 
\end{quote}
Bennett is not very clear about whether the relevant notion of chance here is epistemic or objective. \textbf{check} But whichever way we go, we don't think the diagnosis is very promising. First of all, Bennett seems to be accusing us of conflating things that we seem in fact to quite good at distinguishing. This is particularly clear if we replace likelihood claims with claims about particular people's degrees of confidence, as in
\begin{prop}
	\nitem \label{confcf}
	I am pretty confident that if this dice had been rolled without my knowing that it was rolled, it would have landed on some number other than 6.
\end{prop}
The analogue of Bennett's move in this case would be to say that \ref{confcf} is conflated with \ref{cfconf}:
\begin{prop}
	\nitem \label{cfconf}
	If the die had been rolled without my knowing that it was rolled, I
	would have been pretty confident that it had landed on some number
	other than 6.
\end{prop}
But it is hard to believe that we could conflate such obviously different claims. Moreover, even if such a conflation were plausible, there would be no prospect of using it to explain away our temptation to regard \ref{confcf} as true, since \ref{cfconf} is manifestly false.

Even confining our attention to claims of chance, as Bennett does, the “conflation” strategy strategy delivers terrible results. The problem is especially clear when the time-index of the chance ascription is made explicit:
\begin{prop}
	\nitem \label{nowlikely}
	It is likely right now that if Jim drank arsenic tomorrow, he would be dead by the weekend.
\end{prop}
Bennett's transformation would turn this into
\begin{prop}
	\nitem \label{wouldbelikely}
	If Jim drank arsenic tomorrow, it would be likely right now that he would be dead by the weekend.
\end{prop}
Assuming that it is not in fact likely right now that Jim will be dead by the weekend, \ref{wouldbelikely} seems false: it's not true that if he drank arsenic tomorrow, he would have all along been likely to have been dead by the weekend. And this is so whether the relevant notion of likelihood is understood as objective or epistemic.  In neither case are the probabilities today plausibly regarded as counterfactually dependent on how things play out tomorrow.

A desparate fix for this problem would be to say that we conflate \ref{nowlikely} with some counterfactual concerning probabilities at later times, such as
\begin{prop}
\nitem \label{wouldbelikelythen}
  If Jim drank arsenic tomorrow, it would be likely then that he would be dead by the weekend.
\end{prop}
Quite apart from the intrinsic implausibility of such a conflation, there are cases where a shift to narrow scope coupled with a rewriting of the time won't deliver acceptable truth conditions, no matter what time is used in the rewrite. Consider:
\begin{prop}
\nitem \label{leftpoison}
  There is a 50\% chance if Jim drank poison tomorrow, he would drink poison with his left hand.
\end{prop}
On the natural way of filling out the story, if Jim drank poison tomorrow, it's not clear that there would any time at all such that the chance at that time of his drinking poison with his left hand was 50\%. Once the drinking happened, the chance would be one or zero; beforehand, probably, the chance would be much lower than 50\%, since there would still be a chance of Jim not drinking poison at all.%
\footnote{Perhaps given the continuity of actual-world physics we should admit that if Jim had ended up drinking it with his left hand, there would have been a moment when the chance of his doing so was 50\%, en route to 1 from its initial low value. But this does nothing to help explain the seeming assertability of the sentence, since for one thing few of us are apprised of such physics, and for another thing we could only have this reason for believing that there would have been such an instant if we believed that if Jim drank poison tomorrow he would in fact have done so with his left hand.}
Similar remarks apply to the epistemic interpretation of \ref{leftpoison}. Suppose the situation is one in which if he drank poison, it would be completely obvious to all relevant parties which hand was being used. \ref{leftpoison} still sounds fine in this case, although there is no reason to think if he drank poison, there would be any time at which the epistemic probability of him drinking with his left hand was 50\%.

*** Insert here a cross-reference forward to future discussion of the Kratzerian treatment of the chance- and confidence-theoretic judgments.  

Another group of considerations in favour of CEM have to do with the interaction of negation and denials with conditionals. The validity of CEM is equivalent to that of the inference from ‘It is not the case that if P, Q’ to ‘If P, it is not the case that Q’. However such inferences are hard to evaluate directly: after all, explicit negations of the form ‘It is not the case that if P, Q’ and ‘It is false that if P, Q’ are not common in natural language, and so it would be unwise to place too much weight on any instincts regarding sentences of that form. But there are forms of denial that are much more natural. For example, we can consider negative answers to questions concerning a conditional:
\begin{prop}
\nitem \label{coindialogue}
  \begin{prop}
      \item
		 Would this coin have landed Heads if it had been tossed?
		 \item
		 --- No.
  \end{prop}
\end{prop}
This answer is clearly quite inappropriate unless you have some very unusual knowledge about the characteristics of the coin or the tossing setup. Furthermore, in any setting where that answer is felicitous, one is also in a position to assert
\begin{prop}
\nitem \label{wouldtails}
  If the coin had been tossed, it would have landed Tails
\end{prop}
assuming that one's insider information does not disrupt the standard assumption that if the coin had been tossed, it would have either landed Heads or Tails and not both. But the standard versions of \ref{strict} and \ref{lewis} just described seem to entail that merely knowing that the coin was fair would be enough to justify the negative answer to \ref{coindialogue}, and that the inference from this negative answer to \ref{wouldtails} is completely unjustified.

This kind of consideration also supports CEM in the case of indicatives. For example, the reasoning of the client in the following dialogue looks cogent:
\begin{prop}
	\nitem \label{taxdialogue}
		\emph{Client:} Will I have a big tax bill if I have won the lottery?

		\emph{Accountant:} No.

		\emph{Client:} Great: so if I have won the lottery my financial
		troubles are completely over and done with.
\end{prop}
In saying ‘No’, the accountant is clearly committed to `You won't have a
big tax bill if you have won the lottery'.

It might be suggested that the inappropriateness of the negative answer to \ref{coindialogue} should be assimilated to the phenomenon of ‘Neg-raising’ that applies to words like ‘believe’ and ‘want’, wherein sentences in which negation takes wide scope over some other operator fail to entail, but seem in some other way to suggest, the truth of the corresponding sentences in which negation takes narrow scope. For example, saying `I don't believe it is raining outside', or answering ‘No’ to the question ‘Do you believe that it is raining outside?’, tends to suggest that you believe it isn't raining outside. Likewise `I don't want to go to London' suggests ‘I want not to go to London’. Could this mechanism be enough to explain the seeming goodness of the inference to \ref{wouldtails} from the negative answer to \ref{coindialogue}, and the infelicity of that negative answer? We doubt it. The suggestions associated with Neg-raising are easily cancelled: for example, if my answer to the question ‘Do you want to go to London?’ is `No, I don't care either way', no-one would be tempted to infer that I want not to go to London. By contrast, there is nothing which one could to ‘No’ in answering \ref{coindialogue} which would make it felicitous and block the inference to \ref{wouldtails}. For example, saying `No, it's a fair coin' does nothing to make the answer any better.

Some have argued that when we put focal stress on the word ‘would’ we get judgments more in line with those of CEM-deniers. Hajek (2007), for example, claims that \ref{WOULD} sounds true:
\begin{prop}
\nitem \label{WOULD}
  It is not the case that the coin WOULD land tails if it were tossed.
\end{prop}
We have no clear judgment about what is going on in this sentence; we suspect that sentences generated by prefixing conditionals with ‘It is not the case that’ are so unnatural that our views about them are especially likely to be theory driven. One might try to avoid this by using the question-answer format. But it doesn't seem that focusing ‘would’ in \ref{coindialogue} makes the answer ‘No’ much more acceptable. (The most obvious reason for focusing this ‘would’ is to signal one's challenge to a previous assertion of ‘The coin would have landed heads if it had been tossed’: this context certainly does nothing to improve the negative answer.) Perhaps, however, we can find some more natural examples where focusing ‘would’ provides evidence against CEM, at least for whatever reading of the conditional is activated by such focus. Suppose for example that have a pile of coins, some normal, some double-headed and some double-tailed. It is not unnatural to say things like \ref{threekinds} in describing this situation
\begin{prop}
	\nitem \label{threekinds}
	There are three kinds of coins in this pile: those that WOULD land heads if tossed, those that WOULDN'T land heads if tossed, and the rest.
\end{prop}
And once you have got into this mood, you can for example, issue the instruction
\begin{prop}
\nitem \label{WOULDorder}
	Just give me the coins that WOULD land heads if tossed
\end{prop}
where one expects obedience to consist in handing over the double-headed coins.

One point to make about these focus-based examples is that they carry across to ‘will’, including uses of ‘will’ that are not embedded in conditionals. Compare \ref{threekinds} and \ref{WOULDorder} with:
\begin{prop}
	\nitem
	There are three kinds of students: there are the ones that WILL pass, there are the one's that WON'T pass, and there are the students that might go either way.
	\nitem
	Don't give a student a lot of time unless they WILL pass
\end{prop}
This provides a reason for caution about the putative anti-CEM data from focus, since even those philosophers who reject CEM for counterfactuals will likely still want to preserve sentences like ‘Either you will pass this exam or you will not pass this exam’. Granted, there is the radical option of adopting a theory of ‘will’ according to which it involves universal quantification over a range of possible futures, and hence fails to commute with negation even when only one future time is relevant. But we suspect that the kind of focus-driven effect on display is something much more general, that does nothing special to illuminate the semantics of ‘would’ or ‘will’. For example, we are not seeing a big difference between the foregoing examples and the following:
\begin{prop}
	\nitem \label{sick}
	There are three kinds of patients in this ward: those that ARE sick, those that AREN'T sick, and those that might or might not be sick.
	\nitem
	Don't waste any more time doing tests on the patients who ARE sick.
\end{prop}
A tentative hypothesis: putting focal stress on bland small words like ‘are’ and ‘would’ can make the relevant clause behave as if it were prefixed by some epistemic operator like ‘It is known that\ldots{}’ or ‘It is knowable that\ldots{}’.%
\footnote{\textbf{Add a vagueness-theoretic note here? Some might suggest that the relevant operator is along the lines of ‘It is definitely the case that\ldots{}’. This might be OK for us, but examples like \ref{sick} suggest that that's not always what's going on.}}

Let us briefly report on some other arguments for CEM that can be found in the literature, having to do with a range of embedded constructions. The first argument, due to \textbf{Barker and von Fintel}, turns on the behaviour of ‘only if’. In a wide range of cases involving both indicatives and counterfactuals, ‘P only if Q’ seems to entail ‘If not Q, not P’. Given the standard account of ‘only’ (\textbf{cite Rooth etc.}), the semantic effect of adding ‘only’ to a sentence is to negate a certain range of relevant alternatives to that sentence, which differ with respect to the constituent acted on by ‘only’. Using this machinery it is easy to account for the validity of an inference from ‘P only if Q’ to ‘Not: P if not Q’: we need only say that in context, the relevant alternative to ‘if Q’ is ‘if not Q’. (It would also suffice if we posited a range of relevant alternatives whose disjunction is equivalent to ‘not Q’.) But without CEM, there is no way to bridge the gap from ‘Not: P if not Q’ to ‘If not Q, not P’.%
\footnote{Things get more complicated when the range of relevant alternatives to Q do not exhaust the not-Q possibilities. Suppose that we know that Jim was thinking about going to the museum today. We can say ‘He is having a good time only if he is getting one of the GUIDED TOURS around the museum’. This speech is fine even if we are open to the possibility that in fact, Jim didn't go to the museum and is having a good time in the pub. The standard semantics for ‘only’ predicts this, since it may be that in context, the only relevant alternative for the ‘only’ is tantamount to ‘Jim is in the museum but not getting a guided tour’. By contrast, assuming modus ponens, `If Jim is not getting one of the guided tours around the museum, he isn't having a good time' has to be false if Jim is having a good time in the pub, given that it has a true antecedent and a false consequent. So it seems that the inference from ‘P only if Q’ to ‘If not Q, not P’ can fail in cases where the relevant alternatives do not exhaust the accessible worlds. There are similar complications for the standard logic 101 idea that ‘Only Fs are Gs’ entails ‘All Non-Fs are not Gs’: ‘Only MALE ostriches eat meat’ does not unproblematically entail `Everything that isn't a male ostrich doesn't eat meat'. **However there is a complication here: with focus, ‘If Jim is not getting one of the GUIDED TOURS around the museum, he is not having a good time’ can sound alright even when we think he might be having a good time in the pub. {[}\emph{Can it?}{]} In context, the antecedent seems to be strengthened by the disjunction of the relevant alternatives. This kind of behaviour is of course familiar from unembedded utterance of negated sentences with focus\ldots{}\ldots{} Similarly, sentences like ‘Nothing other than a MALE ostrich eats meat’; `No-one who doesn't take a taxi to this museum enjoys the visit', etc. One approach to this data is to say ‘so much the worse for the reflexivity of accessibility’, i.e.~so much the worse for modus ponens - but we think this is too costly. What we prefer here is a mechanism of ‘embedded scalar implicature’. Just as ‘some’ in ‘Those who ate SOME of their pizza were less ill afterwards than those who ate ALL of it’ can be strengthened to ‘some but not all’, it seems we can strengthen `Those who didn't eat ALL of their pizza were still not as happy as those who didn't eat ANY of it' to `Those who didn't eat all, but did eat some of their pizza\ldots{}'. (\emph{Note that the whole point of these sentences is to contrast two groups among those to whom the relevant restrictor literally applies, so it's hopeless to think that the mechanism is that of domain restriction to exclude those who don't fall under the strengthened meaning.}) Similarly: `If you don't know ALL of your alphabet you will still get some credit for knowing some of it'.}

The second argument, due to von Fintel and Iatridou (2002), turns on a range of examples due to Higginbotham (???) in which ‘if’-clauses seem to occur in the scope of determiner quantifiers like ‘Every’ and ‘No’. Higginbotham's observation is that \ref{everygoof} is equivalent to \ref{nogoof}, assuming that failing can be equated with not passing:
\begin{prop}
	\nitem \label{everygoof}
	Every student will fail if he goofs off
	\nitem \label{nogoof}
	No student will pass if he goofs off
\end{prop}
Von Fintel and Iatridou observe that if we give these sentences the form they seem to have, then if we allow that there might be some students for whom neither ‘x will fail if x goofs off’ nor ‘x will pass if x goofs off’ are true, \ref{nogoof} will fail to entail \ref{everygoof}. Similarly, if we allow that there might be students relative to whom \emph{both} ‘x will fail if x goofs off’ and ‘x will pass if x goofs off’ are true---as we would if we were equating ‘if’ here with the material conditional---\ref{everygoof} won't entail \ref{nogoof}. What's needed to get the equivalence is thus the relevant instances of CEM---namely ‘Either x will fail if x goofs off, or x will pass if x goofs off’---plus the relevant instances of “Conditional Non-Contradiction”, namely ‘It is not the case that both x will fail if x goofs off and x will pass if x goofs off’. (The material conditional view secures the relevant instances of CEM but not CNC; on our favoured view, CEM is unproblematic; while every instance of CNC that doesn't involve violating the presupposition of non-vacuity will be true.)

Note that while Higginbotham's examples belong to the ‘does-will’ category of conditionals that we have been trying to avoid, exactly the same phenomenon arises with standard indicatives and counterfactuals:
\begin{prop}
	\nitem
	Every student would have failed if he had goofed off
	\nitem
	No student would have passed if he had goofed off
	\nitem \label{everygoofpast}
	Every student failed if he goofed off
	\nitem \label{nogoofpast}
	No student passed if he goofed off
\end{prop}

Against this diagnosis of the equivalences, someone might complain that there is a further set of equivalences that need explaining and are not explained by an account like ours, namely the equivalence of some of the above sentences with sentences that don't involve conditionals at all. For example, \ref{everygoof} and \ref{nogoof} have been claimed to be equivalent, respectively, to \ref{everygoofwho} and \ref{nogoofwho}:
\begin{prop}
	\nitem \label{everygoofwho}
	Every student who goofs off will fail
	\nitem \label{nogoofwho}
	No student who goofs off will pass
	\end{prop}
Similarly, \ref{everygoofpast} and \ref{nogoofpast} are arguably equivalent to \ref{everygoofpastwho} and \ref{nogoofpastwho}:
\begin{prop}
	\nitem \label{everygoofpastwho}
	Every student who goofed off failed
	\nitem \label{nogoofpastwho}
	No student who goofed off passed.
\end{prop}
Our account does not seem to predict these equivalences. For example, we would analyse \ref{everygoofpast} as ‘Every student is such that either there is no accessible world where he goofed off, or the closest accessible world where he goofed off is one where he failed’, which is strictly stronger than \ref{everygoofpastwho} (given that the actual world is accessible and closer than all other worlds). Such considerations might push one towards the view that---despite surface appearances---the relevant sentences don't have a logical form whereby a wide-scope quantifier controls a variable within the scope of a conditional, in which case it would be hard to learn anything about the logical behaviour of conditionals from these sentences. However such accounts are quite problematic---see von Fintel and Iatridou (???).

Kratzer (???) suggests a way of recovering the equivalence of the Higginbotham sentences with conditional-free sentences like \ref{everygoofwho}--\ref{nogoofpastwho} while retaining their apparent logical forms. Her idea is that in the quantifiers in \ref{everygoofpast} and \ref{nogoofpast} are by default read as implicitly restricted, so that the sentences are tantamount to:
\begin{prop}
\nitem \label{everygoofrestricted}
  Every student who goofed off failed if he goofed off
\nitem \label{nogoofrestricted}
  No student who goofed off passed if he goofed off
\end{prop}
The idea that the antecedent of the conditional plays a dual role: as well as its standard semantic role, it also serves as a signal that helps to resolve the context-sensitivity of quantifier domain restriction. Given that the conditional validates modus ponens and and-to-if, these sentences are logically equivalent to \ref{everygoofpastwho}
and \ref{nogoofpastwho} respectively.%
\footnote{Kratzer also makes the further suggestion that the relevant accessibility relation for the conditionals in \ref{everygoofpast} and \ref{nogoofpast} is the trivial one in which every world is accessible only to itself, so that the conditionals become equivalent to material conditionals. But her way of recovering the equivalences does not depend on this suggestion: all that matters is that the conditional is one that respects modus ponens and and-to-if.}
This proposal also disrupts the use of the Higginbotham data to argue for CEM, since insofar as the domains are restricted in line with \ref{everygoofrestricted} and \ref{nogoofrestricted}, the only substitution instances of the conditionals we will ever need to consider will be ones with true antecedents, leaving us free to reject CEM for conditionals with false antecedents.

Kratzer in fact applies the domain restriction technique only to the past-tense sentences \ref{everygoofpast} and \ref{nogoofpast}. Her view of the original future-tense sentences \ref{everygoof} and \ref{nogoof} is different: she thinks that the most natural reading of these sentences is one on which they are not equivalent to the conditional-free \ref{everygoofwho} and \ref{nogoofwho}, and do not involve the restriction of the domain to students who will goof off. In support of this, she argues that \ref{everygoof} is false in a case where where one the relevant students is a ``teacher's pet'' who will pass come what may but doesn't in fact goof off, even though \ref{everygoofwho} could still be true. She concludes that the argument for CEM is sound for the ‘will’ conditionals, and indeed presents a CEM-friendly semantics for those conditionals.

We agree with Kratzer about the non-equivalence of \ref{everygoof}/\ref{nogoof} with \ref{everygoofwho}/\ref{nogoofwho}; moreover, we think that there are analogous worries about the claimed equivalence of \ref{everygoofpast}/\ref{nogoofpast} with \ref{everygoofpastwho}/\ref{nogoofpastwho}. Suppose that two busloads of students went to take an exam. The excellent students went in one bus, while the struggling students went in the other. You suspect that the bus containing the struggling students did not make it to the exam on time, since you know that that bus got held up in bad traffic. You thus suspect that only excellent students took the exam. In this setting, you have good reason to suspect that \ref{everytestwho} is true:
\begin{prop}
	\nitem \label{everytestwho}
	Every student who took the exam passed
	\end{prop}
	However you have far less reason to suspect that \ref{everytest} is true:
	\begin{prop}
	\nitem \label{everytest}
	Every student passed if he or she took the exam
\end{prop}
This difference in your levels of confidence strongly suggests that \ref{everytest} and \ref{everytestwho} aren't being interpreted as equivalent. (We are open to there being contexts where sentences like \ref{everytest} are restricted in the way Kratzer suggests; but as the above example illustrates we doubt that there is any general pressure to do so.) So we think that there is still a strong argument for the validity of CEM both for the ‘will’ conditionals and for the past-tense indicatives.

We agree that if one simply looked at \ref{everytestwho} and \ref{everytest} in abstraction from any particular narrative there is some temptation to think that they are equivalent. One possible diagnosis as follows: first, \ref{everytest} straightforwardly entails \ref{everytestwho} by modus ponens. And while the inference in the other direction from \ref{everytestwho} to \ref{everytest} is not valid, it is \emph{quasi-valid} in the sense explained in section ???: the result of strengthening the premise by adding ‘must’ entails the conclusion. The argument thus has the same status as the arguments from ‘P or Q’ to ‘If not P, Q’ and the argument from ‘P or Q and not-P’ to ‘Must Q’. As we saw in section ???, arguments with this status tend to induce positive feelings akin to those produced by actually valid arguments; thus it is no surprise that there is we get a feeling of equivalence when we are confronted with such pairs as \ref{everytestwho} and \ref{everytest}.%
\footnote{\textbf{However, this diagnosis does not apply to ‘Most students who took the exam passed’ and ‘Most students passed if they took the exam’. Come back to this after we have written the chapter dealing with ‘usually if’ and so forth.}}



Some authors who reject the claim that instances of CEM are invariably true have claimed that they nevertheless have an alternative kind of positive status, namely that of being \emph{true whenever their presuppositions are satisfied}.  The idea is that when one says ‘If P, Q’, one presupposes the corresponding instance of CEM, namely ‘Either if P, Q or if P, not Q’.  Most proponents of this idea have been advocates of \ref{strict}, so that for them, the content of this presupposition is that all accessible P-worlds are alike with respect to whether or not they are Q-worlds: the presupposition is thus in their hands a “presupposition of homogeneity”.  A favoured analogy is with plural definites: ‘The philosophers left the room’ is held to presuppose that either all or none of the relevant philosophers left the room, so that ‘Either the philosophers left the room or the philosophers failed to leave the room’ is true whenever its presuppositions are satisfied.%
\footnote{Citations: von Fintel, Klinedinst, ***.  *** We'd better here have a discussion of the presuppositions of disjunction: the only cases that matter for us are relatively uncontoversial, namely when both have the same presupposition, or when the second disjunct entails the presupposition of the first.  In each case it is uncontroversial that the relevant presupposition carries forward to the disjunction.  }

Let's call sentences which have to be true so long as their semantic presuppositions are satisfied “Strawson-valid” sentences.%
\footnote{More generally, Strawson-validity is a property of \emph{arguments}: an argument is Strawson-valid if its conclusion has to be true whenever all of its premises are true, and the presuppositions of all the premises and the conclusion are satisfied.  A Strawson-valid sentence is the conclusion of zero-premise Strawson-valid argument.}
It is worth being clear that Strawson-validity confers nothing like the same kind of security as validity in a more standard conception.  For example, the following would all be Strawson-valid given standard tenets about the presuppositions of ‘the’ and ‘stopped’:
\begin{prop}
	\nitem
	The elephant wearing a hat in my bedroom is wearing a hat.
	\nitem
	Either John has stopped shouting, or John is shouting right now.
	\nitem
	Either I regret eating the moon, or I ate the moon and do not regret it.  
	\nitem
	She is female.
\end{prop}
It is thus important not to assimilate the kind of thorough embrace of CEM that we have been advocating to the more lukewarm endorsement that goes along with Strawson-validity.%
\footnote{The contrast between Strawson-validity and the stronger form of validity tends to be obscured by the popular device of treating presupposition failure as inducing a truth-value gap.  If one holds this, then not even the law of non-contradiction is sacrosanct, since it will have instances that fail to be true because of presupposition-failure, such as ‘It is not the case that (John regrets eating the moon and it is not the case that John regrets eating the moon)’.  Within this framework, then, one might think that Strawson-validity was the strongest status that could reasonably be claimed for any schema.  However, even within this framework, one can still draw a contrast between schemas which are guaranteed to be true whenever non-gappy expressions are substituted for the schematic letters and other schemas whose Strawson-validity is due to the presupposition-theoretic properties of the non-schematic expressions in the schema.}
Nevertheless, we agree that there are some cases where a diagnosis of Strawson-validity provides the best all-things-considered explanation of the positive felt status of some schema.  Indeed, we have provided such a diagnosis in the case of the following good-looking schemas:
\begin{prop}
	\sitem[CNC]
	Not ((if P, Q) and (if P, not Q))
	\sitem[CNC*]
	Not (if P, Q and not Q)
\end{prop}
On our account, this will be false when there are no accessible P-worlds; however, in such cases all the ingredient conditionals---and hence also their conjunctions and negations---will suffer from presupposition failure thanks to the presupposition of non-vacuity discussed in \autoref{sect:nonvacuity}.%
\footnote{As the fact that we are willing to speak of false sentences with false presuppositions indicates, we are not working with a gap-theoretic approach to presupposition; however our remarks could easily be adapted to such a setting.}  

The thesis that CEM is Strawson-valid gives proponents of \ref{strict} or \ref{lewis} to account for the goodness of some of the inferences that feature in certain of our subsidiary arguments against those views.  For example, the fact that answering ‘No’ to the question ‘Would this coin have landed Tails if it had been tossed?’ is inappropriate if one takes the coin to be fair (and not to have been tossed) could be accounted for by saying that such an answer would signal aquiescence in the presupposition that either the coin would have landed Tails if it had been tossed or the coin would have failed to land tails if it had been tossed, a presupposition which is false if the coin is fair and untossed.  However, the Strawson-validity of CEM does not even begin to make sense of the data concerning chances and credences that drive our central argument against \ref{strict} and \ref{lewis}.  For example, on those approaches, ‘I am pretty confident that if he had rolled the die it would have come up between 1 and 5’, as uttered in a setting where one is pretty confident that the die is fair, will be like ‘I am pretty confident that the elephant in the room is more than five feet tall’, as uttered in a setting where one is pretty confident that there is no elephant in the room.  

We can also probe more directly the question whether conditionals carry the presuppositions that would be required for CEM to be Strawson-valid given \ref{strict} or \ref{lewis}.  In general, when a sentence φ carries a presupposition, ‘S doesn't know whether φ’ carries the same presupposition: for example, ‘Bert don't know whether Alice stopped smoking’ presupposes that Alice used to smoke.%
\footnote{It also seems to presuppose that Bert knows that Alice used to smoke, though that won't be important here.}
Given this presuppositional profile, then, the Strawson-validity approach ought to predict that that claims of the form ‘S doesn't know whether the coin would have landed Heads if tossed’ would be infelicitous when the coin in question is known to be fair, or indeed even when it is not known not to be fair.  In fact, however, such claims seem completely fine under such circumstances.  In our view, such attributions of ignorance are acceptable even in the case of a coin that's known to be fair; but their felicity may be even more evident when fairness is only of the epistemic possibilities.   Suppose for example that Sally picked her coin from the bucket with a mix of fair and double-Headed coins---‘Sally doesn't know whether her coin would have landed Heads if tossed’ seems like an excellent description of this situation.  Or consider a self-attribution of ignorance like ‘I don't know whether he would have said yes if she had asked’.  Here it is fine to follow up with ‘He is so unpredictable: he might have said yes and he might not have’.  There is a contrast here between conditionals and the case of the plural definites that forms the inspiration for the view: ‘I don't know whether the philosophers left the room’ does seem to carry a suggestion that the philosophers moved or stayed as a bloc.  Note also that these tests for presuppositionality give a much more favourable verdict in the case of CNC and CNC*: ‘I don't know whether cannons would both work and not work if Aristotelian physics were true’ is quite bizarre.%
\footnote{Note too that following up this speech with ‘That's because Aristotelian physics is impossible’ is absolutely terrible, although the presuppositional view of CEM suggests that this should be the analogue of ‘He is so unpredictable’ followup mentioned above.}


\begin{itemize}
	\item
	Of course one can consider hybrid views: for example, Kratzer (???Conditionals) develops a view that conforms to \ref{strict} for indicatives and \ref{lewis} for counterfactuals.
	\item
	Like \ref{closest}, \ref{lewis} and \ref{strict} can be fleshed out in different ways,
	according to one's favoured gloss on accessibility, worlds, and (in
	the case of \ref{lewis}) closeness.
	\item
	Add: pointing out some nice logical features of CEM, e.g.~the way that
	it makes various things entail CSO and one another.
\end{itemize}

\section{Objections to CEM} \label{objections-to-cem} In the literature---especially thanks to the influence of Lewis (???)---closeness is standardly understood as some kind of similarity. For one world to be closer than another is for the former to be more similar to the actual world than the latter in the relevant respects. What the relevant respects are and what weight they carry is supposed to be up for grabs: Lewis emphasises that the respects and the weighting might not be recoverable simply from general pre-theoretic ideas about overall similarity between worlds. Lewis also thinks that there is plenty of context-sensitivity as regards respects and weightings. Nevertheless, the use of the word ‘similar’ isn't meant to be utterly divorced from its home in ordinary language. There are various structural expectations which are triggered by the ideology of similarity which are preserved on Lewis's view: the kind of relation we are supposed to be thinking about is one specifiable by a scale that comes in degrees, where the degrees in question are determined by some aggregation procedure whose inputs are degrees of resemblance in a range of specified respects, where resemblance comparisons in the particular respects are supposed to be much more straightforward.%
\footnote{As Goodman (???) points out, some of the structural expectations invoked by Lewis's use of the word ‘similar’ are actually ones which he disavows in his most careful moments in a way that many expositors have overlooked, including Lewis himself in some of his less careful moments. In particular, if one pronounces the semantically relevant three-place relation as ‘$w_2$ is more similar to $w_1$ than $w_3$ is’, one would expect that the following pattern can never arise:
	\begin{itemize}
		\item
		$w_2$ is more similar to $w_1$ than
		$w_3$ is
		\item
		$w_3$ is more similar to $w_2$ than
		$w_1$ is
		\item
		$w_1$ is more similar to $w_3$ than
		$w_2$ is
	\end{itemize}
	For, letting $|ww'|$ be the degree of similarity between the worlds $w$ and $w'$, these claims would seem respectively to entail the jointly unsatisfiable $|w_1w_2|<|w_1w_3|$, $|w_2w_3|<|w_1w_2|$, and $|w_1w_3|<|w_2w_3|$. But in fact, Lewis is open to the idea that the relevant three-place relation does contain triples with this cyclical structure, and indeed, as Goodman argues, such cases can arguably be constructed for the particular similarity relation described by Lewis in LewisCDTA. The key to resolving the mystery is that for Lewis, the key three-place similarity-theoretic relation should really be pronounced something like ‘$w_2$ is more similar to $w_1$ than $w_3$ is in the respects that matter at $w_1$’; differences in which respects “matter” at different worlds can then create the surprising cycles.}
And even this very schematic notion of similarity would lead one to expect there to be many cases where two worlds are equally similar to a third world. After all, the relations of resemblance in particular respects that enter into the final aggregation procedure are generally things that do allow for ties: for example, two worlds could be equally similar to a third in respect of \emph{mass} even when one was greater than it in mass and the other less. Moreover, if the aggregation procedure ever allows tradeoffs between different respects of resemblance without one trumping the other, one would expect there to be cases where the tradeoff results in an exact balance. In view of the richness of the space of possible worlds, it is thus hard to see how anything we could think of as a relation of overall resemblance could fail to generate many ties.

If the closeness relation is understood as one of similarity, there is also considerable pressure to allow for failures of the Limit Assumption. Certainly for particular respects of resemblance where comparisons are straightforward, there can be propositions such that for any world where that proposition is true, there is another world where that proposition is true that resembles that actual world more in the relevant respect. Take total mass as the relevant respect, and consider the set of worlds whose mass is greater than $m_1$, where $m_1$ is greater than the mass of the actual world. Clearly for any world in this set, there is another world that resembles actuality more in respect of total mass, e.g.~one whose total mass is halfway between $m_1$ and that world's mass. And given standard assumptions about the continuity of various fundamental magnitudes it is hard to see how this sort of phenomenon could fail to be replicated at the level of overall similarity. Thus, even prior to its details being filled in, a similarity-driven conception of closeness strongly suggests both kinds of Lewisian counterexamples to CEM.

Our response is to completely jettison the similarity-driven conception of closeness. We have various reasons for going this route. Two of them are broadly logical. First, as discussed in section ???, infinite agglomeration is just as compelling as finite agglomeration, but as we have just seen, identifying the closeness ordering with a similarity ordering makes the Limit Assumption and thus infinite agglomeration look untenable. Second, as discussed in the previous section, we think there are good reasons to accept CEM, which is again threatened by identifying the closeness ordering with a similarity ordering. 

Even if one accepted these logical principles, one might think it was an overreaction to abandon wholesale the connection between similarity and closeness: one might might propose a picture where similarity facts constrain but do not determine closeness facts, or more generally, a picture where there is a general tendency for the similarity and closeness orderings to line up. (If we only had to deal with the problem of ties, we could entertain the obvious constraint that whenever one world is is more similar to actuality than another it is closer, treating the closeness ordering as a mere tie-breaking refinement of the similarity ordering---cf.~StalnakerIDCEM. It is far less clear how we could understand the constraining role of similarity if we think that the closeness ordering does, while the similarity ordering does not, respect the Limit Assumption.) 

But what really pushes us to reject even this moderate attitude towards the relevance of similarity to closeness is the the pattern of confidence judgments we find for both counterfactuals and indicatives. As a warmup, let us begin with a familiar kind of concern. Suppose first that we tossed a fair coin ten times yesterday and saw that it landed Heads on four of those times. Consider the counterfactual
\begin{prop}
\nitem \label{ninecoins}
  If the coin had landed Heads nine times, it would have landed Heads all ten times.
\end{prop}
Our judgment is that the right way to assign confidence to the proposition expressed by \ref{ninecoins} is to consult what you know about the conditional chances: of the $2^{10}$ equiprobable ways the coins could have landed, ten involve exactly nine Heads outcomes while one involves ten Heads outcomes, so your credence in the proposition should be $1/11$. But if the kind of similarity that constrains closeness is understood in a pre-theoretically natural way, one would expect that any world where the coin comes up Heads ten times will be less similar to actuality than at least one world where it comes up Heads only nine times, so that \ref{ninecoins} would get a vanishingly low credence. For essentially similar reasons there will be a mismatch in the case where you don't know how the coin landed. So long as the coin is fair we think the appropriate credence is still $1/11$; but the similarity-constrained approach threatens to entail that the only way \ref{ninecoins} could be true would be for ten Heads to in fact have been tossed, a hypothesis to which we assign a much lower credence (namely $1/2^{10}$). (Unless the coin actually landed Heads every time, then for every ten-Heads world we should be able to find a nine-Heads world that is more similar to actuality.)

The general problem is that once we connect closeness to similarity---even in the more modest, constraining way---we will be left with odd confidence distributions. In particular, when there are various ways of a thing happening some of which are more similar to actuality than others, the similarity will push us to be unreasonably confident that if that thing had happened it would have happened in one of the more similar ways. Essentially this objection was made in an early review of Lewis's \emph{Counterfactuals} by Fine (???): Fine considered the counterfactual ‘If Nixon had pressed the button there would have been a nuclear holocaust’, and objected that Lewis's theory yields to the problematic judgment that this is false, since worlds where the button is pressed and some subsequent misfire blocks the expected global nuclear war are more similar overall to the actual, nuclear-war-free world. In response, Lewis makes the point that the relevant similarity metric does not have to coincide with the one elicited by pre-theoretic similarity judgements about worlds, and denies that the relevant peaceful worlds are any more similar to actuality than the war worlds on his \emph{intended} similarity metric.%
\footnote{Lewis's final similarity ranking is intended not merely to prevent the peace worlds from counting as more similar to actuality than the war worlds, but further to get some of war worlds to be more similar than any of the peace worlds. We have further concerns about whether his official specification actually achieves this goal. Lewis's trick for promoting the war-worlds above peace worlds---assuming determinism---is to say that all of the latter contain at least two small ‘miracles’, i.e.~localised exceptions to the actual laws, whereas some of the former contain only one small miracle. We are not sure why Lewis is so confident that a delicately adjusted small miracle couldn't do the job of simultaneously ensuring a button-pushing and its failure. For example, Nixon could trip and fall in such a way that his finger hits the button just after his teeth sever the wire connecting it to the nuclear arsenal. (The problem is even more obvious in a case where there are various ways of pressing the button, a few of which are ineffective.) One could however tinker further to address this worry, e.g.~by saying that the aforementioned tripping-and-falling small miracle is more “remarkable” than certain miracles that lead to war, and for this reason makes for more dissimilarity to the actual world. (This suggestion would be in the spirit of some of Lewis's own ideas about the generalisation of the account to the case of indeterminism which we will discuss below.) By contrast, the problem for Lewis's account which we focus on in the main text is not one that could plausibly be evaded by small adjustments to the similarity ranking.} 
But structurally the same problem recurs if we look at the respects of similarity that really do matter according to Lewis. For Lewis, the most important consideration (besides the avoidance of large, widespread exceptions to the actual laws of nature) is the extent of the spatiotemporal region of perfect match. When the region of perfect match between $w_3$ and $w_1$ is a proper part of the region of perfect match between $w_2$ and $w_1$, and neither $w_2$ nor $w_3$ contains ‘big miracles’ with respect to the laws of $w_1$, $w_2$ is more similar to $w_1$ than $w_3$ is. If closeness is constrained by a similarity ranking that works in this way, the upshot is that we can be confident, generally speaking, that if things had gone otherwise than they actually go, the departure would have been later rather than earlier (since the later the departure, the larger the initial chunk of perfectly matching spacetime). But this still leads to problems in the coin example. Suppose that the coin landed Tails the first time it was tossed. Then on Lewis's account some worlds where it lands Heads exactly nine times---namely, those which match actuality throughout a period including the first toss---are more similar to actuality than any world where it lands Heads all ten times. So if we know that it landed Tails the first time, we should be practically certain that \ref{ninecoins} is false. Moreover, if we don't know anything about how it landed but know that it was fair, given that we should still be at least 50\% confident that it landed Tails the first time, the only way we could generate the intuitively correct credence of 1/11 in \ref{ninecoins} would be to assign \ref{ninecoins} a credence of 2/11 conditional on the hypothesis that the coin landed Heads the first time, which seems completely crazy. (After all, conditional on the hypothesis that the coin landed Heads the first time, we are certain that \ref{ninecoins} is true just in case the coin would have landed Heads on all of the final nine flips if it had landed Heads on at least eight of those flips. Intuitively, we ought to assign the latter proposition a credence of 1/10, both unconditionally and conditional on the hypothesis that the coin landed Heads the first time.)

The problem here is essentially the same as the problem of Pollock's Coat which we discussed in chapter 1. There, we saw that a similarity metric that favours late divergence will license us to be very confident in the truth of
\begin{prop}
\nitem \label{coat}
  If my coat had been stolen last year it would have been stolen on
  December 31st.
\end{prop}
when we know that our coat was not in fact stolen last year. And for reasons similar to those discussed above, an approach that licenses such confidence will also generate unreasonable-looking credence profiles in propositions such as \ref{coat} in cases where we are uncertain whether our coat was stolen. In fact, we think that the appropriate way to assign credence to the proposition expressed by \ref{coat} on its most obvious interpretation is to set this credence equal to our expectation for the conditional chance of the coat being stolen on December 31st conditional on its being stolen some time during the year.%
\footnote{Calculating this number is not straightforward given that the coat being stolen on one day requires it not to have been stolen earlier. In the case where you have no relevant discriminating evidence it will be less than 1/365 but certainly not zero.}

Another respect of similarity that proponents of similarity constraints on closeness have actually regarded as important is similarity with respect of the \emph{absence of remarkableness}, particularly in respect of the outcomes of chance processes. The task these theorists set themselves is to craft a notion of similarity on which, assuming that a certain plate is not in fact dropped, some worlds where it is dropped and hits the floor are more similar to the actual world than any world where it is dropped and quantum-tunnels right through the floor. On the (realistic) assumption that such events of quantum-tunnelling always have some miniscule but nonzero chance of happening, the idea that the disruption of \emph{laws} is a count against closeness does not achieve the desired result, since no actual laws need be disrupted by either kind of world. Clearly the worlds in question need not differ with respect to the size of the region of exact match, so Lewis's idea about the primary importance of exact match doesn't help either. Saying that \emph{low-probability} events count against similarity is a non-starter: whatever happens, the precise trajectory of the plate will be a low-probability event, maybe even a zero-probability event. Lewis's suggestion is that among the various low-chance events we should distinguish a subclass of “remarkable” ones, and to count only these as making for dissimilarity. There are various ways of making this suggestion more precise. One possibility worth mentioning is to understand the remarkable outcomes within a given partition of possible outcomes to be those low-probability outcomes that are much more \emph{natural} than most of the low-probability outcomes in the partition (using something like the notion of relative naturalness for properties developed in \cite{LewisNWTU}). For example, the property of being a sequence of ten coin-tosses all of which land Heads looks considerably more natural than the property of being a sequence of ten coin-tosses which land in the pattern HTTHTHHTHH.%
\footnote{Of course both of these properties are a long way from being perfectly natural; but this need not disrupt the judgment of comparative naturalness.} 
Similarly, consider any specification of a way for the plate to hit the floor and break that is detailed for the probability of the plate doing \emph{that} to be roughly as low as the probability of it quantum-tunnelling: plausibly, each of the aforementioned specifications defines a property far less natural than the property of being a quantum-tunnelling through a potential barrier of such-and-such strength.%
\footnote{The naturalness-theoretic gloss on “remarkableness” brings it close to the notion of “atypicality” invoked by Williams (???), drawing on earlier work by Elga (???). The notion of atypicality invoked by these authors is rooted in a mathematical characterisation of the contrast between “random” and “non-random” infinite sequences of coin-tosses due to Gaifman and Snir (???). Williams hopes that Gaifman and Snir's insight can be generalised to finite sequences, and even to particular localised events, and uses expressions like ‘simplicity’ in characterising this generalisation. But Williams mainly focuses on sequences of coin-tosses; it is not straightforward to extract a particular treatment of the quantum-tunnelling case from his remarks.}

Any attempt to specify a similarity relation that turns on considerations of remarkableness will have to reckon with two facts: (a) Assuming that the actual world is quite extensive, we can be very confident that many remarkable events actually occur; and (b) if things had gone differently in some specified way, it would still have been very likely for the ensuing history of the world to contain many remarkable events that don't actually occur. There is plenty to say here (see Hawthorne ???), but we think the best bet for the remarkableness lover is to add a dose of contextualism to the story. For example, we could allow context to contribute a “reference property”, such as the property of being a flip of a certain coin by a certain person during a certain period, or a pattern of motions of a certain plate during a certain period. The extension of this property gives us a reference class which varies from world to world, and may be sometimes empty. We could then articulate a remarkableness-driven respect of similarity such as the following: $w_2$ is more similar to $w_1$ than $w_3$ is iff the reference class at $w_3$ has more remarkable properties than the reference class at $w_1$ or the reference class at $w_2$.%
\footnote{Note that this toy theory counts $w_2$ as more similar to $w_1$ than $w_3$ in the relevant respect even if the reference class has ten remarkable properties at $w_1$, eleven at $w_3$, and only one at $w_2$. A more mechanical similarity measure based on counting remarkable properties would not give this result. But the failure to demote $w_2$ seems desirable. Suppose that in the actual world, I threw ten plates at one wall and they all quantum-tunneled. We don't want to be able to say ‘If I had thrown the plates at the opposite wall, then at least one of them would have quantum-tunnelled’.} 
(Of course a final story will have to say something about how this respect of similarity is to be aggregated with others.)

The structural problem we have identified for similarity-based constraints on closeness recurs for the remarkableness-based similarity measures. The basic problem, as always, is that similarity constraints mandate giving more credence to counterfactuals whose consequents characterise non-actual outcomes that are more similar to actuality, in cases where considerations of chance push in a different direction. Suppose for example that a fair coin landing heads or tails a hundred times in a row count as remarkable outcomes, while a certain specific sequence S of heads and tails outcomes counts as unremarkable. In the actual world we know that the coin was never tossed. Before the coin-tossing was called off, Fred bet that the coin was going to either land all Heads, all Tails, or in sequence S, and George bet that it was going to land in sequence S. Given a remarkableness-based similarity constraint, we should then be able to know that worlds where the coin is tossed a hundred times and the sequence of outcomes is S are closer than worlds with a hundred heads or with a hundred tails.%
\footnote{Of course what matters according to similarity-lovers is overall similarity, not just the particular remarkableness-based respect of similarity that we are currently considering. But in this case, none of the other respects of similarity that have been thought to play a role---perfect match, approximate match, lack of miracles, typicality of the world as a whole, and so on---do anything to favour the all-heads or all-tails worlds over S-worlds, so the S-worlds will plausibly be counted as more similar overall as well as in respect of remarkableness.} 
And we should thus be in a position to assign very high credence to the proposition expressed by \ref{hundred}:
\begin{prop}
\nitem \label{hundred}
  If Fred had won his bet, George would have won his bet too.
\end{prop}
But intuitively this proposition deserves a credence equal to the conditional chance of George's winning given Fred's winning, namely 1/3.%
\footnote{Could we avoid the undesirable result that (37) is true by appeal to some further contextualist trick, according to which (37) for some reason evokes a context in which the all-heads and all-tails outcomes don't count as remarkable in the way that detracts from their similarity? In principle yes, but it's hard to see how such an account would go. For example, saying that the mere \emph{salience} of a particular low-probability outcome makes that outcome stop counting as remarkable will disrupt the proposed explanation of the truth of `If the plate had been dropped it wouldn't have quantum-tunnelled through the floor'.}

\begin{itemize}
\item
  Add note about how the case relates to the arguments in Hawthorne 2003 and Williams's responses to that. (That paper primarily uses assertion rather than credence.)
\end{itemize}

The problem with similarity-constraints on closeness is thus quite general. But without a general theory of closeness that guarantees that the worlds where the plate hits the floor and breaks are closer than worlds where it quantum-tunnels through the floor, how are we to explain the assertability of \ref{break} and \ref{notunnel}?
\begin{prop}
\nitem \label{break}
  If the plate had been dropped, it would have broken
\nitem \label{notunnel}
  If the plate had been dropped, it would not have gone right through the floor
\end{prop}
The first point to note about these sentences is that, according to \textbf{the way of assigning degrees of confidence to counterfactuals that was endorsed in CEM section}, we should have a high degree of confidence in the propositions expressed by \ref{break} and \ref{notunnel}, assuming that we know that the antecedents are false and that the relevant conditional objective chances are high.%
\footnote{Even if we don't know that the antecedent is false, the counterfactuals will still deserve high credence so long as we don't have evidence relevant to their consequents that is “inadmissible” with respect to the relevant time.} 
Of course, high credence isn't in general sufficient to explain assertability---consider ‘This is a losing lottery ticket’. A very standard view is that assertability requires knowledge. How, without some general analysis of closeness in terms of similarity, or some analysis of accessibility that guarantees that there aren't any accessible worlds where the plate is dropped and tunnels through the floor, could we ever know whether the closest accessible world where the plate is dropped is a tunnelling world? A good starting point for thinking about this question is the apparent assertability, despite small objective chance of error, of simple claims about the future: for example, our background knowledge about quantum tunnelling does not block us from asserting ‘This plate will soon be broken’, or from self-ascribing knowledge that the plate will soon be broken. It is not an easy task to come up with a workable theory that pinpoints the relevant difference between future lack of quantum tunnelling and future lottery-losing; for some ideas about this, including ideas according to which our reactions betray deep-seated errors, see
\cite{HawthorneKL}.%
\footnote{Our talk of objective chance here need not be understood as building in a commitment to indeterminism: for accounts of chance-talk that make non-trivial chance-ascriptions compatible with determinism, see ***. Nor do the epistemological issues turn essentially on the assumption of indeterminism: even if we knew that determinism was true, ‘I will lose the lottery’ would still be unassertable (in ordinary circumstances), and it would still be hard to find an explanation of this unassertability that didn't overgeneralise to all manner of ordinary assertions about the future.} 
We can remain neutral about this here: the main suggestion we want to make is that one should approach the knowledge and assertability of counterfactuals in the same way as the knowledge and assertability of claims about the future. Whatever explains the unassertability of ‘I will lose the lottery’ will also explain the unassertability of ‘If I had bought a ticket in that lottery I would have lost’; whatever explains why the unassertability of the first claim doesn't carry over to all sorts of other claims about the future which have small chances of being false will explain why the unassertability of the second claim doesn't carry over to all sorts of other counterfactuals whose consequents have a small chance of being false conditional on their antecedents. Of course whatever defensive manoeuvres we make, we should concede that in the future case, if you are unlucky and the plate does in fact tunnel through the floor, then the earlier self-ascription of knowledge was false after all; similarly, if you are unlucky and the closest accessible world where the plate falls is one where it tunnels, your self-ascription of knowledge that the plate would break if it fell is false.%
\footnote{In the case of future-tense claims, the prima facie attractive idea that knowledge is closed under many-premise deductive consequences leads to further puzzles: by performing a conjunction introduction based on many intuitively knowable high-chance premises, one can deduce a conclusion that one knows to have very low chance. This is both puzzling in its own right, and makes further puzzles for the project of formulating an account of the rational connection between the proposition that P and the proposition that P has a certain objective chance. Given the best-known such principle (Lewis's ‘Principal Principle’), we will have a choice between saying either that the conjunction in question is known despite the fact that one rationally ought to have assigned it a very low credence, or appealing to the standard get-out clause of ‘inadmissible information’. All of this structure carries over to the case of counterfactuals: given multi-premise closure and (finite) agglomeration for counterfactuals, we will be in a position to know certain counterfactuals for which SKYRMS* recommends very low confidence. Possible reactions include giving up multi-premise closure; tolerating the idea that one can know things to which one ought to have assigned low confidence; or modifying SKYRMS* further with an ‘inadmissible counterfactual information’ clause.}

(The challenge of connecting up facts about the appropriate degrees of confidence in conditionals with facts about assertability and knowledge arises for indicatives as well as for counterfactuals. However, for indicatives, there is more scope for explaining assertability by appeal to the accessibility parameter. Suppose that I know that the plate won't be dropped without breaking, and suppose that what it takes for a world to be accessible is that be compatible with my knowledge. Then the aforementioned knowledge about the future will guarantee the truth of ‘If the plate is going to be dropped, it is going to break’; and if I know that I know that the plate won't be dropped without breaking, I will be in a position to know the proposition expressed in context by the indicative conditional without drawing at all on thoughts about closeness. On our view this kind of explanation is often apt, but is not the only path to knowledge of indicative conditionals: one can also bring in whatever resources one used to make room for knowledge of counterfactuals like \ref{break} and \ref{notunnel}. By contrast, it is much harder to see how a proponent of \ref{strict} can make room for knowledge of indicative conditionals in a case where iterated knowledge is unavailable.)

We have rejected similarity-based analyses of closeness. And as will be becoming increasingly clear to the reader, we are not going to put anything their place: we will be “treating the concept of closeness as primitive”. At least, we will not be offering the kind of analysis of closeness that would allow us to break out of the circle of concepts that includes closeness as well as conditionals: on our account, so long as we are in a context where both $w_1$ and $w_2$ are accessible, ‘$w_1$ is closer than $w_2$’ is equivalent to ‘If one of $w_1$ or $w_2$ were actualised, $w_1$ would be actualised’ and to ‘If one of $w_1$ or $w_2$ is actualised, $w_1$ is actualised’. We are a little wary of calling this an ‘analysis’, in part because we are a bit unclear about what it takes for something to count as an analysis. Claims of analysis are sometimes understood to carry implications about what grounds what, or what is more metaphysically fundamental or natural than what, or what is more conceptually basic or “explanatorily prior” to what: we are not endorsing any such claim or priority for closeness over conditionals.

So, in a sense we are less ambitious in our goals than many other theorists of conditionals have been. Nevertheless, the claims we are making are by no means trivial - indeed as we have seen they are inconsistent with the views of many other writers, so we feel no need to defend the substantiveness or interest of the views we are putting forward.

“Treating closeness as primitive” is merely refraining from offering any definitions or equivalences that break out of the circle just noted. It is not any kind of claim about closeness. We are thus not making any claim to the effect that closeness is fundamental or perfectly natural or unanalysable or anything else of the sort. Nevertheless, the structural claims we are making about closeness invite a certain inchoate metaphysical worry. Take for example the claim that a certain untossed coin is either such that some world where it lands Heads is closer than any world where it lands Tails, or such that some world where it lands Tails is closer than any world where it lands Heads. What, people want to know, could conceivably \emph{make it be the case} that one of these closeness profiles obtains?

\begin{itemize}
\item
  \emph{mention the tradition of thinking that middle knowledge is weird}
\end{itemize}

\section{Conditionals, probabilities, and conditional probabilities}
\label{conditionals-probabilities-and-conditional-probabilities}
Our central arguments for CEM and against similarity-based accounts of closeness have turned on judgments about the chances of conditionals, and the appropriate levels of confidence in conditionals, in simple toy examples involving coin-tosses and the like. It seems plausible that should be some general principles under which these judgments can be subsumed. In this section we will introduce some natural-looking principles, trace the connections between them, and see how they might be derived from underlying assumptions about the relevant notions of accessibility and the probabilistic features of the closeness ordering. The general principles we will isolate will then be further refined in section ***.

Let's begin with the judgment that when a coin is fair, the objective chance that it would land Heads if it were tossed (once) during the next hour is 50\%. Plausibly, this is driven by a certain judgment about the chances of propositions that are not conditionals, namely that the chance that the coin will be tossed during the next hour is twice as big as the chance that it will be tossed and land Heads during that hour. In general, it's natural to think that the chance that Q would be true if P were true is equal to the chance that P and Q are true, divided by the chance that P is true: what is standardly known as the “conditional chance” of Q given P. Making the time relativity explicit, we get the following:
\begin{prop}
	\litem[CHANCE EQUATION] \label{chanceq}
	When the chance at $t$ that $P$ is positive, the chance at $t$ that if $P$ it would be that $Q$ equals the conditional chance at $t$ of $Q$ given~$P$.%
	\footnote{\textbf{If needs be, restrict to the case where the chance of $Q$ is defined.}}
\end{prop}

Of course, on our account the counterfactual here involves a context-sensitive “accessibility” parameter, and there is no temptation to think that the CHANCE EQUATION is true no matter how that is resolved. For example, consider the counterfactual ‘If I had finished the book in 2016, that would have been because I worked hard on it all through 2015’, which cries out for an interpretation where accessibility does not require match with respect to 2015. Suppose that I didn't in fact work on the book in 2015, with the result that at the beginning of 2016 the chance of my finishing it in 2016 was miniscule (although not zero). Since the truth about 2015 has chance 1 at the beginning of 2016, the conditional chance then of my having worked hard on the book in 2015, given that I finish it in 2016, was zero. But on its natural interpretation, the counterfactual in question seems plausibly true; and while it's not very clear what its chance exactly was at the beginning of 2016, the judgment that it's plausibly true does not fit well with the claim that its chance at the beginning of 2016 was zero. However, these considerations do not count against the \ref{chanceq} when we interpret the accessibility parameter for the counterfactual in what seems like the most natural way, where the accessible worlds are all and only those that match actuality with respect to the course of history up to $t$.%
\footnote{\textbf{Note that since $t$ is really a bound variable here, the relevant interpretation of an instance of the \ref{chanceq} involves the kind of ‘binding into context sensitivity’ we discussed earlier with reference to the example of ‘local’.}}

*** First have a ‘CONDITIONAL CHANCE EQUATION’ analogous to 

Many have found it natural to analyse the time-relative notion of chance that features in the \ref{chanceq} in terms of a notion of chance not relativised to times, what is sometimes called ‘ur-chance’: the idea is that for the chance of $P$ at $t$ to be $x$ is just for the ur-chance of $P$ conditional on the truth about history up to $t$ to be $x$.
\footnote{The question what the ur-chances are is supposed to be a contingent, empirical matter just like questions about chances at times---the idea is that a great deal of science can be reconstructed as an investigation of the character of the ur-chance distribution.} 
(One nice feature of this analysis is that it readily explains the fact that the chance of a proposition at a later time is always equal to its chance at an earlier time conditional on the complete truth about intervening history.) Given this analysis, the \ref{chanceq}, on our favoured interpretation of the counterfactuals, is equivalent to the following generalisation about ur-chances:
\begin{prop}
	\litem[CONDITIONAL UR-CHANCE EQUATION] \label{condurchanceq}
	When $H$ is any true proposition completely describing an initial stretch of history, and $P$ is any proposition such that the ur-chance of $P$-and-$H$ is positive, the ur-chance, conditional on $H$, that the closest nomically possible $P$-and-$H$ world is a $Q$-world, equals the ur-chance of $Q$ conditional on $P$-and-$H$. 
\end{prop}
It's natural to allow for the trivial case where $H$ is simply a tautology---this gives an analogue of the \ref{chanceq} for the ur-chances. However, there is no uncontroversial way to derive the above more general principle from that special case.%
\footnote{One could produce such a derivation using the principle “RIE”, according to which ‘If A and B, then C’ is always equivalent to ‘If A, then if A and B, then C’. For the special case yields that the ur-chance of $Q$ conditional on $P$ and $H$ equals the ur-chance of ‘If $P$ and $H$, then $Q$’, which by RIE equals the ur-chance of ‘If $H$ then if $P$ and $H$ then $Q$’, which by the special case again equals the ur-chance of ‘If $P$ and $H$ then $Q$’ conditional on $H$. Although RIE is not part of any of the standard logics of conditionals, as we will see in the next section it has some nice features which make it attractive.}

Note that there is an analogue of the CONDITIONAL UR-CHANCE EQUATION for time-relative chance: the chance at $t$, conditional on any complete true description $H$ of history up to any subsequent time $t'$, that if $P$-and-$H$ it would be that $Q$ equals the chance at $t$ of $Q$ conditional on $P$-and-$H$. This follows from the CONDITIONAL UR-CHANCE EQUATION given the analysis of time-relative chance. (Note that for the same reason as before, the obtaining of this analogue principle at a given time cannot be uncontroversially derived from the obtaining of the CHANCE EQUATION at that time.) Moreover, even if you don't work with the concept of ur-chance, you will be committed to this principle insofar as you are committed to the claim that every time satisfies the CHANCE EQUATION, given that the chances at later times derive from those at earlier times by conditionalisation.

\begin{itemize}
\item
  Should we say somewhere that if the ur-chances are introspective and
  the UR-CHANCE EQUATION is itself a necessary (chance 1) truth, the
  restriction to true $H$ can be dropped from it?
\end{itemize}

So far we have been talking about interpretations of counterfactuals on which accessibility amounts to nomic possibility together with match of some initial segment of history. Back in chapter 1, we suggested that this kind of interpretation is widespread, but by no means universal, for counterfactuals. The example of Morgenbesser's Coin served as a paradigm case of a context governed by a different kind of accessibility restriction: plausibly ‘If I had bet on Heads, I would have won’ favours an interpretation where the accessible worlds are required not only to match with respect to history prior to the time of betting, but also to match with respect to the (later) outcome of the coin toss (and perhaps also with respect to various other facts conceived of as causally isolated from the bet on Heads). What is the chance of the conditional, before the decision whether to bet, under this interpretation? The mere fact that the coin was \emph{in fact} going to land Heads, so that all of the accessible worlds were in fact Heads worlds, is neither here nor there: since there was a 50\% chance that the coin would land Tails, there was a 50\% chance that all of the accessible worlds ---and so in particular the closest accessible world where I bet on Heads---would instead be Tails worlds. If we conceive of “winning” in such a way that there are no nomically possible worlds where I bet on Heads and the coin lands Heads but I fail to win, and no nomically possible worlds where I bet on Heads and the coin doesn't land Heads and I still win, this is all we need to derive the intuitive result that the chance of the conditional, under the relevant interpretation, is 50\%. However, things become more subtle if we switch instead to something like ‘If I had bet on Heads, I would be in a good mood’. Suppose that at a certain pre-betting time, the conditional chance of my being in a good mood given that I bet on Heads and the coin lands Heads is 90\%, and the conditional chance of my being in a good mood given that I bet on Heads and the coin lands Tails is 30\%. It seems quite intuitive that the chance of the conditional, in that case, was 60\%. But on the operative interpretation the conditional is equivalent to the following disjunction:
\begin{quote} 
	Either Heads and the closest (history-and-law-matching) Heads world where I bet on Heads is one where I am in a good mood, or not-Heads and the closest not-Heads world where I bet on Heads is one where I am in a good mood. 
\end{quote}
It doesn't follow from the CHANCE EQUATION alone, or from the CONDITIONAL UR-CHANCE EQUATION, that the chance of this disjunction is 60\%: the two closeness claims have chances 90\% and 30\% respectively, but there is no guarantee that they are independent of Heads and Tails, and thus no guarantee that the chances of the two disjuncts are 45\% and 15\% respectively. For example, there is nothing to rule out the possibility that conditional on my not betting on Heads and the coin not landing Heads, there is no chance at all that I'm in a good mood in the closest not-Heads world where I bet on Heads, in which case the second disjunct has chance 0. Satisfying the CHANCE EQUATION merely requires that the overall chance that I'm in a good mood in the closest not-Heads world where I bet on Heads, conditional on my not betting on Heads, is 30\%, but does not constrain how this 30\% gets distributed between the Heads and not-Heads possibilities.

\emph{Work up to the following:}

\begin{prop}
	\litem[UR-CHANCE CONJUNCT INDEPENDENCE] \label{urconjunctindependence}
	If the ur-chance of $P$-and-$Q$ is positive, the ur-chance, conditional on $P$, that the closest nomically possible $P$-and-$Q$ world is an $R$-world, equals the ur-chance of $R$ conditional on $P$-and-$Q$. 
\end{prop}
Like all the generalisations we are considering in this section, UR-CHANCE CONJUNCT INDEPENDENCE will turn out to need some restriction. But we hope that the preceding discussion has helped to make it plausible that the appropriately restricted version of UR-CHANCE CONJUNCT INDEPENDENCE is the most basic principle about the chances of conditionals, which not only explains the appropriately restricted version of the CHANCE EQUATION, but also explains the more complicated facts about the chances of counterfactuals under interpretations where accessibility relation is something other than simple historical match.

\begin{itemize}
\item
  Add parenthetical paragraph about chance-zero issues for the standard
  picture.
\item
  Add something about the question whether, in conditionalising on
  “history up to $t$”, we are merely conditionalising on the
  categorical facts, or also on facts about the closeness ordering.
\end{itemize}

The foregoing account of the chances of counterfactuals provides a very natural justification for, and explanation of, the claims we have been relying on this chapter about the \emph{credences} we ought to assign counterfactuals. Consider, for example, the judgment that when you are certain that a coin is fair, your degree of confidence that it would land Heads if it were tossed (once, during the next hour) should be 50\%. If you can reasonably be certain of the relevant instance of the CHANCE EQUATION, you should be certain that the chance that it would land Heads if it were tossed equals the conditional chance that it will land Heads given that it will be tossed, which you are certain is 50\%. The appropriateness of assigning 50\% credence to the counterfactual can then be explained by appeal to the following principle: when you should be certain that the current chance of a proposition is $x$, your credence in that proposition should be $x$.%
\footnote{\textbf{Two worries about this principle: (a) Lucky; (b) if you think you can know non-chance-1 things about the future and that you may be certain of what you know, you may have inadmissible evidence about the future, e.g.~in the cases from the coins paper.}} 
This is a special case of the following equally attractive general principle linking chance and credence, which applies even when you are not certain about the current chances:
\begin{prop}
	\litem[CURRENT-CHANCE PRINCIPAL PRINCIPLE] \label{currentpp}
	Your credence in a proposition $P$, conditional on the proposition that the current objective chance of $P$ is $x$, should be $x$.
\end{prop}
Together with the assumption that you should be certain of the relevant
instances of the CHANCE EQUATION (concerning the present time), this
yields the following generalisation about the appropriate credences for
counterfactuals:
\begin{prop}
	\litem[CURRENT SKYRMS PRINCIPLE] \label{currentskyrms}
	When your credence that the current conditional chance of $Q$ given $P$ is $x$ is positive, your credence that Q would be true if P were true, conditional on the proposition that the current conditional chance of Q given P is $x$, should be $x$. 
\end{prop}
If we bracket worries about infinity, this constraint on your
conditional credences allows us to calculate what your
\emph{unconditional} credence in a counterfactual ought to be:
\begin{prop}
	\item
	Your credence that Q would be true if P were true should equal the expected value, according to your credences, of the current conditional objective chance of Q given P.
\end{prop}

\textbf{EXPLAIN}

What about cases where the relevant propositions about objective chance concern the chances at some time other than the present, for example when I assign a credence of 50\% to the proposition that a certain coin (which was not in fact tossed yesterday) would have landed Heads if it had been tossed, on the basis of my knowledge of the pattern of conditional chances two days ago? One might be tempted to simply generalise CURRENT SKYRMS to arbitrary times:
\begin{prop}
	\litem[SKYRMS] \label{skyrms}
	Conditional on the proposition that at $t$, the conditional objective chance of Q on P is x, your credence that Q would be true if P were true should equal x. 
\end{prop}
But this is no good. The most obvious problem concerns counterfactuals
whose antecedents deserve substantial credence. Suppose I am very
confident that a certain coin was tossed and landed Heads; then given
\emph{Modus Ponens} and \emph{And-to-if}, I should also be confident
that it would have landed Heads if it had been tossed---and this is true
even if I am sure that the coin is fair, and hence that before it was
tossed, the chance of its landing Heads conditional on its being tossed
was only a half. The obvious way to get around this problem for SKYRMS
as written is to revise it so that it is only taken as a guide to
credence in counterfactuals conditional on the falsity of their
antecedents:
\begin{prop}
	\litem[SKYRMS*] \label{false_antecedent_skyrms}
	Your credence that Q would be true if P were true, conditional on the hypothesis that \emph{P is false and} the conditional objective chance at t of Q on P is x, should equal x.%
	\footnote{\textbf{Something here about inadmissibility???}}
\end{prop}
But even SKYRMS* seems quite tendentious. Suppose you are now certain that you didn't go to the movies yesterday, and also that two days ago the chance of the proposition that you enjoy your trip to the movies, conditional on the proposition that you do go to the movies yesterday, was high. SKYRMS* says you should now be confident that you would have had a good time if you had gone to the movies yesterday. This prescription applies even if you discover that the projector broke halfway through the showing of the film that you would have gone to see, where this breakdown was a low chance eventuality two days ago. It's tempting to think that this discovery should lead you to decrease your level of confidence that you would have enjoyed the movie if you had gone to it below the level recommended by SKYRMS*. It is not clear that this refutes SKYRMS* on the intended interpretation where the accessibility constraint for the counterfactual is simply match with respect to history up to $t$: perhaps what we are doing is evaluating the counterfactual using one of the more complex accessibility constraints discussed in section \textbf{Morgenbesser}, where being accessible amounts to matching actuality both with respect to history up to yesterday, and with respect to certain matters causally independent of whether you go to the movies or not. That said, it remains quite natural to suppose that the discovery about the projector is evidentially relevant even on in the intended interpretation---certainly on many ways of thinking about closeness, the discovery would be strong evidence that the closest worlds where one goes to the movies are also projector-breakdown worlds.

Even if the foregoing concerns made us suspicious of SKYRMS*, we can at least fall back on the following more guarded generalisation of CURRENT SKYRMS:
\begin{prop}
	\litem[ADMISSIBLE SKYRMS] \label{admissible_skyrms}
	If none of your evidence concerning history subsequent to $t$ is “inadmissible” with respect to the proposition that Q would have been true if P had been true, then conditional on the hypothesis that the conditional chance at $t$ of $Q$ given $P$ was $x$, your credence that Q would have been true if P had been true should equal x.
\end{prop}
Here “inadmissible” evidence just means whatever kinds of evidence about subsequent history \emph{do} have a bearing on the counterfactual. In the case of a counterfactual with a true antecedent we have a clear sense of what this is, while as we have just explained, questions about the evidential bearing of facts about post-$t$ history (such as the projector breakdown) on counterfactuals with false antecedents entirely about history prior to $t$ are much harder to adjudicate. Just as CURRENT SKYRMS can be derived from the CHANCE EQUATION together with CURRENT PP, so ADMISSIBLE SKYRMS can be derived from the CHANCE EQUATION together with
\begin{prop}
	\litem[PRINCIPAL PRINCIPLE] \label{pp}
	If none of your evidence concerning history subsequent to $t$ is inadmissible with respect to the proposition that $P$, then conditional on the hypothesis that the chance at $t$ of $P$ is $x$, your credence in $P$ should be $x$. 
\end{prop}

The CURRENT PP follows from PP given the further premise that we never have, at any time, evidence about later history that is inadmissible with respect to any proposition $P$. Depending on how one thinks of the relevant notion of evidence, this may in fact be a problematic premise. In the literature, people sometimes raise this worry by appeal to exotic possibilities involving crystal balls. But there may be more mundane versions of the worry. Some have the view that anything that one knows counts as a piece of evidence. Suppose you know that you had dinner last night and that you will have dinner tonight, despite the fact that there is currently a small chance that you will die before tonight. In this case, on the relevant view of evidence, your evidence logically entails the counterfactual ‘If you had had dinner yesterday, you would have dinner tonight’. One might take this to show that it's rationally permissible for your credence in the counterfactual to be 1, rather than somewhat less than 1 as recommended by CURRENT SKYRMS. Even more obviously, if one simply simply knew a certain counterfactual about the future (\textbf{refer to discussion of knowledge of counterfactuals not based on knowledge of categorical stuff}), that might be taken to license a credence out of kilter with CURRENT SKYRMS. Insofar as we don't want to pre-judge these issues, ADMISSIBLE SKYRMS provides a relatively safe fallback.

Let's now turn to the case of indicative conditionals. We have been relying on judgments such as the following: when you are sure that a coin is fair but unsure whether it was tossed, you should be about 50\% confident that it landed Heads if it was tossed. There is obviously a general pattern here, and the obvious way of capturing it is as follows:
\begin{prop}
	\litem[CREDENCE EQUATION] \label{credenceq}
	If you have positive credence in Q, then your credence that if P, Q should equal your conditional credence in P given Q. 
\end{prop}

Recall that as we are using ‘conditional credence’, your conditional credence in P given Q is just your credence in P-and-Q divided by your credence in not-Q, at least when the latter is positive.%
\footnote{mention Edgington on conditional credence as a \emph{sui generis} mental state}

Of course, given the context-sensitivity of conditionals, principles like CREDENCE EQUATION need to be treated with some care. The most we could reasonably expect is that CREDENCE EQUATION will be true when its context-sensitivity is resolved in a certain particular, salient way---presumably, one in which the condition for accessibility has something to do with what is epistemically live from the perspective of the person whose credences are in question.

\begin{prop}
	\litem[CONDITIONAL PRIOR EQUATION]
	If $\prior$ is a rational prior credence function, then for any proposition E which could be one's total evidence, and any $P$ such that C(P and E) is positive, C(the closest P-and-E world is a Q world|E) = C(Q|P-and-E). 
\end{prop}

\begin{itemize}
\item
  restrict to ‘a priori possible’ worlds?
\item
  motivate the following, in part, by considering Kaufmann-style
  examples.
\end{itemize}

\begin{prop}
	\litem[PRIOR CONJUNCT INDEPENDENCE]
	If $\prior$ is a rational prior credence function, then for any $P$ and $Q$ such that C(P and Q) is positive, C(the closest P-and-Q world is an R world|P) = C(R|P-and-Q). 
\end{prop}

We will have more to say on ways in which the context-sensitivity of indicatives can be resolved in ways that are unfriendly to THE EQUATION in section ???. And as we shall discuss in section ???, there are obstacles, whatever the context, to endorsing the the pattern captured by THE EQUATION in full generality. But as we see things, these worries about the principle in full generality does nothing much to overturn the plausibility of judgments about what credences are appropriate in particular simple cases, or the problems that such judgments pose for CEM-unfriendly approaches to indicatives.

\begin{itemize}
\item
  Outline of this whole section

  \begin{itemize}
  \item
    Chance Equation
  \item
    Deriving the Chance Equation from an Ur-Chance Equation

    \begin{itemize}
        \item
      make sure to be clear that the concept of ur-chance is physics
      driven, and that we are still imposing a substantive accessibility
      constraint, namely to physically possible words.
    \end{itemize}
  \item
    In cases where you have an additional matching constraint, you don't
    get the Chance Equation. Chancy versions of the Kaufmann cases here?
  \item
    If you have a view where counterfactuals entirely about pre-t
    matters have chance 1 or 0 at t, then (given our general strategy of
    letting accessibility rather than closeness do this kind of work)
    you're going to need also to say that accessibility involves a
    matching constraint on such counterfactuals. And similarly, if you
    have a view where counterfactuals whose antecedents have chance 0 at
    t all have chance 1 or 0 at t (the ‘freezing’ view), you'll want a
    correspondingly demanding accessibility restriction. (Note: the van
    Fraassen thing still seems to work even in this case!)
  \item
    Credences of counterfactuals: the Principal Principle, and the
    Skyrms-type principles that it gets you.

    \begin{itemize}
        \item
      A natural way of doing it has the most basic PP-style principle be
      about priors and ur-chances. Principles about posteriors that have
      a ‘no inadmissible evidence’ clause are pretty much tantamount to
      principles about priors.
    \end{itemize}
  \item
    The Credence Equation for indicatives. Which notions of
    accessibility are friendly to it and which are unfriendly?

    \begin{itemize}
        \item
      Note that even if unembedded indicatives and ‘might’s are often
      evaluated with respect to some relevant group, it's still
      plausible that in the context of a quantified attitude report,
      like ‘For any rational person x, if x is confident to such and
            such degree that if P then Q that\ldots{} then\ldots{}’, there is
      a tendency to resolve it so that just that person's evidence is
      what matters.
    \end{itemize}
  \item
    How failures of introspection can make for failures of the Credence
    Equation, and why we shouldn't worry about this.
  \end{itemize}
\end{itemize}

\section{The threat of triviality} \label{chances-and-credences-for-conditionals}
At least in the case of indicative conditionals, many readers will be nervous about the kinds of credence-theoretic judgments on which we have been relying in arguing for CEM and against similarity-theoretic accounts of closeness, at least when combined with the view that such conditionals express truth-evaluable propositions. The best-known worry here stems from certain “triviality theorems” due to Lewis (**). When $\prob$ is any probability function and $→$ is any binary operator (function from pairs of propositions to propositions), say that $→$ bears the CCCP-relation to $\prob$ iff for all propositions A and B, $\prob(A→B) = \prob(B|A)$. Using this definition, we can state Lewis's central result as follows:
\begin{quote}
	If $→$ bears the CCCP-relation to $\prob$, then $→$ does not bear the CCCP-relation to any other probability function derived from $\prob$ by conditionalisation, except for “trivial” probability functions that assign 1 and 0 to every proposition. 
\end{quote}
To prove this, assume for \emph{reductio} that $→$ bears the CCCP-relation both to $\prob$ and to $\prob_C$, where $\prob_C$ is distinct from $\prob$, non-trivial, and derived from $\prob$ by conditionalisation on some proposition $C$. Note that for these conditions to be met $\prob(C)$ must be strictly between 0 and 1, since if it is 0 the operation of conditionalisation is not defined, while if it is 1, $\prob_C$ is not distinct from $\prob$. Let $D$ be some proposition such that $\prob_C(D)$ is neither 0 nor 1; there must be some such proposition by the assumption that $\prob_C$ is nontrivial. Let $E$ be the disjunction of $D$ with the negation of $C$. Since $\prob_C(¬C) = 0$, $\prob_C(E) = \prob_C(D)$; thus $\prob_C(E)$ is not 0 or 1, and so we know $\prob(E)$ cannot be 0 or 1 either. Now consider the proposition $E→¬C$ derived from $E$ and $¬C$ by our operator $→$. We have that $\prob(E→¬C|C) = \prob_C(E→¬C) = \prob_C(¬C|E)$ (since $→$ bears the CCCP-relation to $\prob_C$ and $\prob_C(E)>0$) = $\prob(¬C|EC) = 0$. So $\prob(¬C) = \prob(E)\prob(¬C|E)$ (since $E$ entails $¬C$) = $\prob(E)\prob(E→¬C)$ (since $→$ bears the CCCP-relation to $\prob$)= $\prob(E)(\prob(E→¬C|C)\prob(C) + \prob(E→¬C|¬C)\prob(¬C)) = \prob(E)(0 + \prob(E→¬C|¬C)\prob(¬C))$. But then $\prob(E)\prob(E→¬C|¬C)$ must be 1, which is impossible given that $\prob(E)$ is not 1.

Lewis's result will pose a challenge to anyone who thinks that there is a particular interpretation of the conditional as expressing a binary operator $→$ that bears the CCCP-relation to every reasonable credence function. For given the result, this claim commits one to the view that there are no two nontrivial reasonable credence functions both of which obey the probability axioms, such that one of them can be derived from the other by conditionalisation. This is inconsistent with the classical Bayesian position according to which reasonable people's credences always obey the probability axioms and evolve by conditionalisation.

Of course, this classical Bayesian position is controversial in many ways. The claim that credences evolve by conditionalisation, a process that always involves becoming \emph{absolutely certain} of some propositions of which one was previously not absolutely certain, is particularly tendentious, given that without a Cartesian philosophy of mind it seems hard to identify a sufficiently rich range of candidates to be the propositions deserving credence 1; starting with \citet{JeffreySILP}, many authors have wanted to allow for the possibility that one's credences might evolve reasonably without any change in the set of propositions to which one assigns credence 1. But further generalisations of the theorem due to Lewis, Hajek and Hall (\textbf{cite}) suggest that worries about strict conditionalisation are not really to the point. One important generalisation is the ‘orthogonality’ result proved in Hall ???: if two nontrivial probability functions $\prob$ and $\prob'$ are both CCCP-related to single binary operator $→$, they must be “orthogonal” in the sense that for some proposition $A$, $\prob(A)=1$ and $\prob'(A)=0$. If one wanted to maintain that credences should evolve in such a way as to stay CCCP-related to a single binary operator, one would thus have to not only \emph{agree} with proponents of strict conditionalisation that rational changes of credence always involve coming to have credence 1 in some proposition in which one previously had credence less than 1, but maintain, even more surprisingly, that such changes always involve coming to have credence 1 in some proposition in which one previously had credence 0. This kind of ban on moderate revisions of one's credential state is hard to make palatable.

\begin{itemize}
\item
  Lewis's result is also problematic for those who merely think that
  credences ought to \emph{approximately} satisfy the probability
  axioms, or that the relevant operator → is one that
  \emph{approximately} obeys the Equation\ldots{}
\end{itemize}

Since Lewis's result is just about the space of probability functions to which a particular binary operator bears the CCCP-relation, it does not turn on a credence-theoretic gloss on the relevant notion of probability. The result is thus, for example, also problematic for those who think that there is a particular interpretation of the conditional as expressing a binary operator → with the property that for any time $t$, the chance at $t$ of $A→B$ equals the conditional chance at $t$ of $B$ on $A$. For it is plausible that chances obey the probability axioms, and that the chance function at a later time $t'$ is always equal to the result of conditionalising the chance function at an earlier time $t$ on the complete truth about history between $t$ and $t'$ (at least in cases where that complete truth had a nonzero chance of obtaining at $t$).%
\footnote{Note on determinism-friendly notions of chance\ldots{}}

Lewis's result does not, however, pose any particular problem for our highly contextualist approach to both indicative and counterfactual conditionals. In the case of counterfactuals, our view is that the accessibility parameter can naturally be filled in by properties of the form \emph{being a world at which everything which has no objective chance of being false at $t$}, where $t$ is a contextually relevant time. To the extent that we think that there is a connection between conditional chances at $t$ and the chances of counterfactuals at $t$, it is only when the relevant counterfactuals are assigned this particular interpretation. When $t=t'$, we think that---for the most part, with some exceptions that we will discuss later---the chance at $t$ of the proposition \emph{either A has no chance of being true at $t'$ or the closest B-world where everything that has no objective chance at $t'$ of being false is true is an A-world} is (at least approximately) equal to the conditional chance at $t$ of B on A. \textbf{Fix to fit new material in previous section.}  But when $t≠t'$, there is no general reason to expect this to be the case. Context-sensitivity plays a similarly crucial role for indicative conditionals. The accessibility parameter for an indicative conditional can naturally be filled in by properties of the form \emph{being a world compatible with $x$'s total evidence at $t$}. When $x = x'$ and $t = t'$, and $x$ is ideally reasonable at $t$, we think that (with various exceptions to be discussed later) $x$'s credence at $t$ in the proposition \emph{either no A-world is compatible with $x'$'s total evidence at $t'$, or the closest $A$-world compatible with $x'$'s total evidence at $t'$ is a B-world} is at least approximately equal to $x$'s conditional credence at $t$ in $B$ given $A$. But when $x≠x'$ or $t≠t'$, there is no general reason to expect this to be the case. In particular, if $x$'s credence function at $t$ is the result of conditionalising $x'$'s credence function at $t'$ on some new evidence, Lewis's result provides a way of generating pairs of propositions for which $x$s conditional credences at $t$ come apart from $x$'s unconditional credence at $t$ in the proposition expressed by the indicative conditional interpreted using the original accessibility parameter.

In thinking about the status of general principles like \textbf{***the Credence and Chance Equations} in our framework, it is important to remember that that we allow for the context-sensitivity of the accessibility parameter to be ‘bindable’ (see section ????). Even on a particular occasion of utterance, we don't have to think of the conditional connective as contributing a single binary operator (function from pairs of propositions to propositions): when there is a higher quantifier in the sentence, a conditional can contribute different binary operators relative to different assignments to the variable bound by the higher quantifier. This means, for example, that `Any rational person's credence that if Oswald didn't shoot Kennedy, someone else did equals their conditional credence that someone else shot Kennedy, conditional on the proposition that Oswald didn't shoot him' does not, on its most natural reading, imply that there is any one proposition such that any rational person's credence in that proposition equals their conditional credence that someone else shot Kennedy, conditional on the proposition that Oswald didn't shoot him. (Similarly, ‘Everyone asserted that the local bars are great’ does not, on its most natural reading, imply that there is any one proposition which everyone asserted.) The precisifications of the Credence and Chance Equations that are left untouched by Lewis's results can then be spelled out along the following lines:
\begin{quote}
	For any propositions P and Q and time t such that the chance of P at $t$ is positive, the chance at t of the proposition \emph{either there is no chance that P at $t$, or the closest P-world where everything that has no chance of being false at t is true is a Q-world} equals the conditional chance at t of Q given P. 
\end{quote}
\begin{quote}
	For any propositions P and Q, time t, and person x who is ideally rational at t and has a positive credence in P at $t$, x's credence at t in the proposition \emph{either $x$'s evidence at $t$ is inconsistent with P, or the closest P-world where all of $x$'s evidence at $t$ is true is a Q-world} equals $x$'s conditional credence at $t$ in Q given P. 
\end{quote}

(It's worth noting that there are several other expressions whose context-sensitivity tends to go hand in hand with that of the accessibility parameter for indicative conditionals: this includes the epistemic uses of ‘might’, ‘must’, ‘possible’, ‘have to’, ‘probably’, ‘likely’, and so on. These expressions also enter into generalisations that are at least as tempting as the Credence Equation. For example:
\begin{prop}
	\item
	Conditional on the proposition that is probable that P, you should have high credence that P.
	\item
	Conditional on the proposition that it might be the case that P, you should have positive credence that P.
\end{prop}
But there are results similar in structure to Lewis's result (Hawthorne and Russell 20??) that threaten to reduce these principles to absurdity on any interpretation that assigns a single operator to ‘it is probable that’ or ‘it might be the case that’. Here again, we can give the principles a much better run for their money by appealing to the bindable context-sensitivity of ‘probable’, ‘might’, etc.)

While Lewis's results do not threaten the Credence and Chance Equations on the above interpretations, there is another result, due to Stalnaker (???), that shows that they cannot be true in full generality. Stalnaker's result shows that there is a very wide range of probability functions $\prob$---a range that certainly includes many probability functions that could characterise reasonable credence functions, or chance functions---are such that no binary operator having certain logical properties bears the CCCP-relation to them. Since the logical properties in question are properties that characterise conditionals on any resolution of context-sensitivity, according to our closeness-theoretic analysis, this result shows that the Credence and Chance Equations are untenable even on the bound readings that we have isolated.

To understand Stalnaker's result, we can begin by observing that given that a certain binary operator $→$ obeys Modus Ponens and And-To-If, it bears the CCCP relation to a certain probability function $\prob$ if and only if, for any propositions $A$ and $B$ for which $\prob(A)>0$, $\prob(A→B)=\prob(A→B|¬A)$. To put it in a standard jargon, a conditional is probabilistically independent of the negation of its antecedent.
\footnote{This is equivalent to its being probabilistically independent of its antecedent.} 
The reason is that
\begin{align*}
\prob(A→B) & = \prob(A→B|A)\prob(A) + \prob(A→B|¬A)\prob(¬A) \intertext{(probability theory)}
&= \prob(B|A)\prob(A) + \prob(A→B|¬A)\prob(¬A) \\\intertext{MP, And-to-If}
&= \prob(A→B)\prob(A) + \prob(A→B|¬A)\prob(¬A) \\\intertext{since $→$ bears the CCCP-relation to $\prob$}
\intertext{and so}
\prob(A→B)(1-\prob(A)) &= \prob(A→B|¬A)\prob(¬A)\\
\intertext{and since $\prob(¬A) = 1-\prob(A) > 0$}
\prob(A→B) &= \prob(A→B|¬A)
\end{align*}
Given this result, there can't be any case where $\prob(A)$ and $\prob(AB)$ are both strictly between 0 and 1 and where $A→B$ logically implies $A$. For in that case, the logical implication will force $\prob(A→B|¬A)$ to be 0, so $\prob(A→B)=0$, so $\prob(B|A)=0$, contradicting the assumption that $\prob(A∧B)$ is positive. So, all Stalnaker needs to do is to exhibit some conditional that logically entails its own antecedent, and is also such that it would implausible to claim that every reasonable credence function or chance function assigns probability 0 or 1 either to the antecedent or to the conjunction of the antecedent and the consequent.

Stalnaker's example of this pattern concerns conditionals of the form \emph{If (either $A$ or if $A$ then $B$), then $A$ and $B$}. Given the closeness-theoretic analysis, we can reduce to absurdity the assumption this conditional is true but has a false antecedent, as follows. For the antecedent to be false, the actual world must be a not-$A$ world, and also there must be an accessible $A$-world, and the closest accessible $A$-world---call it $w$---must be a not-$B$ world. Since this entails that there is an accessible world where the antecedent is true (since its first disjunct is true), the conditional can only be true in these circumstances if the closest world where the antecedent is true---call it $w'$---is an $A$-and-$B$ world. But since the antecedent is logically weaker than $A$, $w'$ is either $w$, or a world that is closer than $w$. But the former case is incompatible with the truth of the conditional since $w$ is a not-$B$ world, while the latter case is similarly incompatible, since $w$ is the closest $A$-world.

Someone might hope to block the implication from \emph{If (either $A$ or if $A$ then $B$), then $A$ and $B$} to its antecedent by somehow tinkering with the structure of our closeness-theoretic model. However, if one retains VLAS (“Very Limited Antecedent Strengthening”), together with the principle that logically equivalent sentences can be substituted salva veritate in the antecedents of conditionals, there will be no way to deny this implication: for VLAS gives us \emph{If $A$ and (either $A$ or if $A$ then $B$), then $B$}, which is equivalent to \emph{if $A$ then $B$} given the substitution principle, which entails \emph{either $A$ or if $A$ then $B$} by disjunction introduction. Anyone who, like us, finds VLAS and the substitution principle compelling, will thus have to restrict the scope of the Credence and Chance Equations. Note also that VLAS is a straightforward consequence of CSO, while the substiution principle follows from CSO together with the principle that if P logically entails Q, ‘If P, Q’ is true. CSO will thus almost certainly have to be sacrificed in order to preserve the Equations in full generality.

We will have more to say below about views that attempt to maintain the fully general versions of one or both Equations by denying CSO. But first, we would like to show how it is possible to maintain restricted but still very powerful versions of the Equations---principles that can explain and systematise the various case-by-case judgments about credences and chances that we have appealed to---within a CSO-friendly logical framework.

Fortunately, the literature has provided us with what we need, in the form of a paper that Bas van Fraassen published in 1975. Our picture of how one ought to assign credences to hypotheses about the closeness ordering is inspired by one of the constructions that plays a starring role in that paper. To run with the construction, we will need to suppose that we can somehow make sense of a contrast between “categorical” and “hypothetical” propositions, in such a way that the truth values of hypothetical propositions do not supervene on the truth values of categorical propositions. Let us suppose further that this can be done in such a way that typical sentences not involving conditionals express categorical propositions. This is going to be important. Van Fraassen's results provide a way of vindicate the restriction of the Equations to conditionals with categorical antecedents; this won't provide much of a vindication of our judgments about the chances and credences of ordinary conditionals unless such conditionals typically have categorical antecedents.

The easiest way to get the picture on the table involves a somewhat colourful metaphysical just-so story. Given that not everything supervenes on the categorical, God's creative work wasn't over when he had determined which categorical propositions were going to be true. He also needed to generate a countably infinite list of complete categorical profiles, headed by the actually instantiated categorical profile. The character of this list of profiles will turn out to play an important role in determining the closeness relation among worlds, and hence the truth values of conditionals---if a profile including the proposition that A and B occurs earlier on the list than any profile including the proposition that A and not B, then some A-and-B world is closer than any A-and-not-B world. (Later on we will explain how to flesh this out into a complete account of closeness.) The key idea is that in advance of any evidence, we should think about the composition of the list in such a way that (a) our confidence that any given categorical proposition is part of the $n$th member of the list equals our confidence that that proposition is true, and (b) we treat treat questions about the nature of the profile that occurs in position $j$ as independent of questions about the nature of the profile that occurs in positions other than $j$. These conditions can be precisified using the language of priors: any ideally reasonable prior credence function $\prior$ should (a) assign the proposition \emph{$P$ is entailed by the profile at position $n$} the same probability as $P$ itself, and (b) for any two propositions $P$ and $Q$ such that $P$ is entirely about a certain position $j$ and $Q$ is entirely about positions other than $j$, $\prior(P) = \prior(P|Q)$.  We also require (c) that countable additivity obtain for propositions about different positions in the sequence (even if it does not obtain in general): if $P_1, P_2, \ldots$ are all entirely about different positions, $\prior(P_1 ∨ P_2 ∨ \cdots)$ = $\lim_{n→∞}\prior(P_1 ∨ \cdots ∨ P_n)$.  

\begin{itemize}
\item
  note: check what's actually required for a ‘Bernouilli distribution’,
  including ‘local’ countable additivity. + we don't necessarily want to
  to follow vF in just leaving the non-measurable things out of the
  algebra altogether - express it in a way consistent with a more
  ‘permissivist’ approach to these cases.
\end{itemize}

%One upshot of this account of rational priors is that if $\prior(P)$ is positive, $\prior$($P$ occurs at some position in the sequence) is 1. Let us use ‘$@nP$’ to express the proposition that $P$ is entailed by the $n$th profile in the sequence. The proposition that $P$ occurs at some position in the sequence is thus equivalent to the infinite disjunction $@1P ∨ @2P ∨ @3P ∨ \cdots$. This is logically equivalent to the following infinite disjunction in which any two disjuncts are jointly inconsistent: $@1P ∨ (¬@1P ∧ @2P) ∨ (¬@1P ∧ ¬@2P ∧ @3P) ∨ \cdots$: here the $n$th disjunct is $¬@1P ∧ \cdots ∧ ¬@(n-1)P ∧ @nP$, the proposition that $P$ occurs for the first time in the $n$the position in the series. By the countable additivity requirement, the probability of this disjunction should equal the sum of the probabilities of its disjuncts: $\prior(@1P ∨ (¬@1P ∧ @2P) ∨ (¬@1P ∧ ¬@2P ∧ @3P) ∨ \cdots) = \prior(@1P) + \prior(¬@1P ∧ @2P) + \prior(¬@1P ∧ ¬@2P ∧ @3P) + \cdots$. By the independence requirement, the probability of each of these conjunctions equals the product of the probabilities of its conjuncts: $\prior(¬@1P ∧ \cdots ∧ ¬@(n-1)P ∧ @nP)$ = $\prior(¬@1P)\cdots\prior\prior(¬@(n-1)P)\prior(@nP)$. And by condition (a), $\prior(@nP) = \prior(P)$, so this equals $x(1-x)^n$, where $x = \prior(P)$. So we have $\prior$($P$ occurs at some position in the sequence) = $(x + (1-x)x + (1-x)^2x + (1-x)^3x + \ldots)$ $= x\sum_{n=0}^∞ (1-x)^n = (x/1-(1-x)) = 1$ (using the formula for the sum of a geometric series). Meanwhile, if $\prior(P)$ is zero, $\prior$($P$ occurs at some position in the sequence) is also zero, since it is the sum of an infinite sequence all of whose members are 0. Note that in thinking about what it means for a proposition to be such that all rational priors assign it probability 0 or 1, we should beware of equating this status with some kind of “epistemic impossibility” or “epistemic necessity”, since there is good reason to take seriously the hypothesis that some true propositions---e.g.~about the precise landing point of a dart---deserve prior probability of 0 simply because of their extreme specificity.% \footnote{Some philosophers have proposed thinking of credences as nonstandard real numbers as a way of avoiding this results\ldots.}

Let's not for now get into the question what deeper explanation, if any, might be given for the idea that rational prior probability functions treat questions about the composition of the sequence in this distinctive way. We will return to that kind of issue in the concluding chapter. For now, what's important is to see how these constraints get us results in the vicinity of the Equations. The most basic such consequence is as follows:
\begin{prop}
	\litem[SEQUENCE CONJUNCT INDEPENDENCE]
	If $A$, $B$, $C$ are categorical propositions and $\prior$ is a rational prior credence function with $\prior(AC)>0$, then $\prior$(some $ABC$-profile occurs earlier in the Sequence than any $A\negate{B}C$-profile|$C$) = $\prior(B|AC)$. 
\end{prop}
\begin{proof}
	Let us use ‘$@_nP$’ to express the proposition that $P$ is part of the $n$th profile in the Sequence, and ‘$P≺Q$’ to express that $P$ occurs earlier than $Q$ in the Sequence. 
	We begin by calculating $\prior(ABC ≺ A\negate{B}C|\negate{A}C)$, i.e.
	\[
	\frac{\prior((ABC≺A\negate{B}C) ∧ \negate{A}C)}{\prior(\negate{A}C)}
	\]
	The numerator is equivalent to the following infinite disjunction:
	\begin{equation*}
		(\negate{A}C ∧ @_2ABC) ∨ (\negate{A}C ∧ @_2\negate{AC} ∧ @_3ABC) ∨ 
		(\negate{A}C ∧ @_2\negate{AC} ∧  @_3\negate{AC} ∧ @_4ABC) ∨ \cdots
	\end{equation*}
	Since the disjuncts are pairwise inconsistent and ***WHAT FURTHER CONDITION EXACTLY?***, the countable additivity requirement (c) implies that the probability of this disjunction equals the sum of the probabilities of the disjuncts.  So let us consider the probability of the $n$th disjunct,
	\[
	\prior(\negate{A}C∧@_2\negate{AC}∧\cdots∧@_{n}\negate{AC}∧@_{n+1}ABC)
	\]
	By the independence requirement (b), since the conjuncts are about different positions in the sequence, the probability of the conjunction equals the product of the probabilities of the conjuncts, 
	\[
	\prior(\negate{A}C)\prior(@_2\negate{AC})\cdots\prior(@_n\negate{AC})\prior(@_{n+1}ABC).
	\]  
	And by condition (a), $\prior(@_nP) = \prior(P)$ for any categorical $P$, so this equals
	$\prior(\negate{A}C)\prior(\negate{AC})^{n-1}\prior(ABC)$.  The probability of the disjunction is thus
	\[
		\prior(ABC)\prior(\negate{A}C) (1 + \prior(\negate{AC})^ + \prior(\negate{AC})^2 + \cdots)
	\]
	which by the standard formula for the sum of an infinite geometric series equals 
	\begin{equation*}
	\frac{\prior(ABC)\prior(\negate{A}C)}{1 - \prior(\negate{AC})}
	 = \frac{\prior(ABC)\prior(\negate{A}C)}{\prior(AC)}
	 = \prior(B|AC)\prior(\negate{A}C)
	\end{equation*}
	So we can conclude that
	\[
	\prior(ABC ≺ A\negate{B}C|\negate{A}C) = \prior(B|AC).
	\]
	Meanwhile, we also have that
	\[
	\prior(ABC ≺ A\negate{B}C|AC) = \prior(B|AC),
	\]
	since $AC$ implies that $ABC ≺ A\negate{B}C$ just in case $B$ is true.  So, finally, we can use the total probability formula to conclude that
	\begin{align*}
		\prior(ABC≺A\negate{B}C|C) 
			&= \prior(ABC≺A\negate{B}C|AC)\prior(A|C) + 	\prior(ABC≺A\negate{B}C|\negate{A}C)\prior(\negate{A}|C) \\
			&= \prior(B|AC)\prior(A|C) + \prior(B|AC)\prior(\negate{A}|C) \\
		&= \prior(B|AC)(\prior(A|C)+\prior(\negate{A}|C)) \\
		&= \prior(B|AC) \qedhere
	\end{align*}
\end{proof}


Meanwhile, if $\prior(P)$ is zero, $\prior$($P$ occurs at some position in the sequence) is also zero, since it is the sum of an infinite sequence all of whose members are 0. Note that in thinking about what it means for a proposition to be such that all rational priors assign it probability 0 or 1, we should beware of equating this status with some kind of “epistemic impossibility” or “epistemic necessity”, since there is good reason to take seriously the hypothesis that some true propositions---e.g.~about the precise landing point of a dart---deserve prior probability of 0 simply because of their extreme specificity.% \footnote{Some philosophers have proposed thinking of credences as nonstandard real numbers as a way of avoiding this results\ldots.}

What does this claim, which concerns a certain proposition about the Sequence---that some $A$-and-$B$ profile occurs earlier in it than any $A$-and-not-$B$ profile---have to do with conditionals? It's at this point that we need to appeal to the aforementioned bridge principle, according to which, whenever some $P$-profile occurs earlier in the Sequence than any $Q$-profile, some $P$-world is closer than any $Q$-world. Given our closeness-theoretic account of conditionals, this means that the proposition expressed by ‘if $A$, $B$’ on an interpretation where the accessibility restriction does not rule out any worlds deserves the same prior credence as the proposition that either there is no A-world, or some A-and-B world is closer than any A-and-not-B world.  But when $\prior(A) > 0$, the bridge principle entails that $\prior$(there is at least one $A$-world) is 1, since by the SEQUENCE EQUATION, $\prior$(some $A$-and-$A$ profile occurs earlier in the Sequence than any $A$-and-$¬A$ profile) = $\prior(A|A)$ = 1 whenever $\prior(A)>0$.  So we get that when $A$ and $B$ are categorical and $\prior(A)>0$, $\prior(\text{if A then B}) = \prior(B|A)$.  



Of course, the Credence Equation is not just supposed to hold for ordinary people with rich bodies of evidence, not just for hypothetical beings with nothing but their priors to guide them. The most straightforward route from the Prior Equation to the Credence Equation would involve the following somewhat tendentious assumptions about the relation between rational credence and priors.
\begin{prop}
\litem[Conditionalisation]
	If $x$ is ideally rational at $t$, then for some rational prior credence function $\prior$, $x$'s credence function is the result of conditionalising $\prior$ on the proposition that is $x$'s total evidence at $t$. 
\litem[Introspection]
	If $P$ is $x$'s total evidence at $t$, then $P$ entails that $P$ is $x$'s total evidence at $t$. 
\end{prop}
In this setting, the Prior Equation will yield the following
\begin{prop}
	\litem[Posterior Equation]
	If $x$ is ideally rational at $t$, \textbf{P} is $x$'s credence function at $t$, and \textbf{P}(A)>0, then \textbf{P}(\emph{some $A$ world is consistent with $x$'s total evidence at $t$, and the closest $A$-world consistent with $x$'s total evidence at $t$ is a $B$-world}) = \textbf{P}(B|A). 
\end{prop}

(Proof\ldots{})

\begin{itemize}
\item
  in this case there is a serious worry about the fact that the priors
  will assign credence 0 to some things that have positive chance at t,
  so you maybe can't get by straightforward conditionalisation from the
  priors to the chances at t. Deal with this in the first instance by
  confining attention to “macro” chances?\\
\item
  remember that the guise business will eventually make this more
  complicated\ldots{}
\item
  An alternative construction: the Epistemic Sequence is a sequence of
  Metaphysical Sequences, each of which comes from an epistemically
  possible ur-chance die roll gizmo\ldots{}
\item
  Still, we are in need of a positive result to reassure us that we can
  have quite a lot of Equation/Skyrms style judgments without getting
  into contradiction.
\item
  The basic vF picture (God, dice, etc.), and what it gets you:
\end{itemize}

\begin{prop}
\item
  Categorical Equation for ur-chances;
\item
  Categorical Equation for chances at t, assuming that these are derived
  from ur-chances by conditionalisation on the categorical truth about
  times up to t, assuming that the chance of the chance facts
\item
  Categorical Equation for the priors on indicatives.\\
\item
  Categorical Equation for posteriors that are got by classical
  conditionalisation, where the accessibility parameter matches what was
  conditionalised on.
\end{prop}

\begin{itemize}
\item
  Embedded conditionals.\\
\end{itemize}

\begin{prop}
\item
  vF's way of extending
\item
  how this generalises the Equation to conditional consequents, and even
  some not-too-complex antecedents involving conditionals.

  \begin{itemize}
    \item
    Define a ‘safe’ proposition $P$ as one such that if, given a
    certain sequence, $P$ holds at the $k$th world but not at the
    $j$th world, where $k>j$, then every other sequence that agrees
    with that sequence up to $k$ and is such that $P$ holds at the
    $k$th world is one where it doesn't hold at the $j$th world.\\
  \item
    Show that when $P$ is safe, the conditional probability that $Q$
    is true at the $k$th world given that $P$ is true at the $k$th
    world equals the conditional probability that $Q$ is true at the
    $k$th world given that $P$ is true at the $k$th world and not
    at any earlier world. So long as the probability that $P$ is true
    at some world is 1, it follows that the conditional probability that
    $Q$ is true given that $P$ is true equals the unconditional
    probability that $Q$ is true at the first $P$-world.\\
  \item
    And more generally: when $P$ and $Q$ are both safe, the
    conditional probability that $R$ is true at world $k$ given that
    $P$ and $Q$ are equals the conditional probability that $R$ is
    true at world $k$ given that $P$-and-$Q$ is true at world
    $k$, $P$-and-$Q$ is not true at any world $<k$, and $P$ is
    not true (i.e.~not true at world 0). So long as the probability that
    $P$-and-$Q$ is true at some world is 1, it follows that the
    conditional probability of $R$ given $P$-and-$Q$ is equal to
    the conditional probability that $R$ is true at the first
    $P$-and-$Q$ world given not-$P$, and hence also (given the
    previous result) to the conditional probability that $R$ is true
    at the first $P$-and-$Q$ world given $P$.\\
  \item
    Show too that:

    \begin{itemize}
        \item
      everything categorical is safe (obviously)
    \item
      every conditional with a safe antecedent and consequent is safe.
      (Suppose that $P→Q$ is true at world $k$ but not at world
      $j$. Then it must be that for some $n$ with $j≤n<k$, world
      $n$ is a $P$-and-not-$Q$ world, and world $m$ is a
      not-$P$ world for every $j≤m<n$. But all of this will still be
      true no matter how we modify the sequence at position $n$ and
      later.)
    \item
      every conjunction of safe propositions is safe (obviously).
    \item
      every proposition that, for some categorical question, says what
      the answer to that question is at each of the first $n$ worlds
      is safe. (Suppose that $P$ is like this, true at $k$ but not
      at $j$. There are two possibilities: either the categorical
      profile of worlds earlier than $k$ is already inconsistent with
      $P$, or $P$ is consistent with this but inconsistent with the
      categorical profiles of worlds $≥k$). In the former case, $P$
      will still be false at $j$ no matter how we change things
      $≥k$; in the latter case, the falsity of $P$ at $j$ follows
      from its truth at $k$.)
    \end{itemize}
  \item
    On the other hand:

    \begin{itemize}
        \item
      The negation of a safe proposition need not be safe
    \item
      The disjunction of safe propositions need not be safe
    \end{itemize}
  \end{itemize}
\item
  It validates the surprising new logical inference rules “RIE”: If C,
  D; If B or C, B; so if B (if C, D).\\
\end{prop}

\begin{itemize}
\item
  If P and Q, R; so if P (if P and Q, R). (And backwards. )
\end{itemize}

\begin{enumerate}
\setcounter{enumi}{3}
\item
  In the case of counterfactuals, this is defensible but not
  particularly compelling. Suppose that as a matter of fact, if I had
  tossed the coin twice, it would have come up Tails both times; but if
  it had come up Heads the first time, it would have come up Heads the
  second time. Does it follow that if I had tossed it and it came up
  Tails both times, it would still have been the case that it would have
  come up Heads the second time if it had come up Heads the first time?
  The idea suggests a surprisingly fatalistic conception of the
  counterfactuals in question.\\
\item
  If you don't like it, you can adapt the vF construction to no longer
  yield it: instead of assigning a categorical profile to each natural
  number, God makes a tree of categorical profiles by assigning a
  categorical profile to each $n$-tuple of positive numbers. We make
  this tree of categorical profiles into a tree of trees of categorical
  profiles by assigning to each $n$-tuple $s$ the tree we get by
  throwing all the elements that don't begin with $s$ and pruning
  $s$ off the beginning of all the elements that do begin with
  $s$.\\
\item
  But actually the vF rule has the great property that it makes ‘if A,
  then if B, C’ logically equivalent to ‘if A and B, C’ when the
  embedded conditional is given the natural constrained reading on which
  ‘If A, then if B, C’ is tantamount to ‘If A, then if A and B, C’. This
  is a nice and desirable feature! Let's tentatively endorse it. In
  terms of closeness, it's equivalent to: if $w_1$ is closer than both $w_2$
  and $w_3$, then at $w_1$ ($w_2$ is closer than $w_3$) iff $w_2$ is closer than $w_3$.
\end{enumerate}

\begin{itemize}
\item
  Say something about indicatives that embed counterfactuals - idea that
  counterfactuals get to count as honorary categoricals in this setting.
\item
  Talk about Andrew's CSO-denying way of getting the Equation in greater
  generality.
\item
  Why it requires introspection, and strict conditionalisation,
  and\ldots{}
\item
  Note: go back and revisit CSO discussion either here or in the
  previous chapter, taking note of the quasi-validity type moves that
  CSO-rejecters could appeal to to defeat the seeming evidence for
  CSO.\\
\item
  ‘If he had been in England, he would have been in London, but if he
  had been in the South of England, he would have been in Oxford’. And
  the indicative version thereof. The CSO rejecter can say that the
  indicative version is a bad speech, since it implies ‘He is not in
  England’, thus violating the presupposition of nonvacuity; but this
  does not explain the problem with the counterfactual version. The
  counterfactual speech does however have the deficiency that,
  plausibly, you couldn't know it. Is this an adequate account of its
  badness?\\
\item
  More argument forms that look good but aren't valid according to CSO
  rejecters:
\item
  ‘If A and B, C; if A and not B, C; so if A, C’. Not even quasi-valid.
  (It may have some ‘preserving high credence’ status; but there is no
  reason to think that this status is something we have trouble
  distinguishing from validity proper.)

  \begin{itemize}
    \item
    Consider quantified speeches to block knowledge-based explanations
    of apparent goodness of inferences/badness of conjunctions. ‘At
    least two of these 10 people would have been in London if they had
    been in England\ldots{}’.
  \end{itemize}
\item
  Other “triviality” results (Hajek): infinitely many worlds; Jeffrey
  conditionalisation\ldots{}
\item
  Think more about things we could say about the importance of
  counterfactuals, and whether these things can help our case for
  (counterfactual) CSO. Use in deliberation - seems like obviously good
  deliberation. But there are moves that can be made by anti-CSO people
  in this context, where counterfactual CSO has some positive
  credence-preserving status (figure out exactly what this would be?).
\item
  Cian: look again at Andrew's paper to see how things break down if you
  imagine someone with super strong evidence that leaves only 4
  epistemically possible worlds.
\end{itemize}

\begin{itemize}
	\item 
	This chapter is going to end with a discussion of why the limitations our approach requires aren't tragic.  One important point to consider is to what extent considerations of probabilistic independence might help when we are dealing with the unsafe.  
	
	Example: two very far apart coins, one of which has a very high chance of being tossed, the other of which only has a tiny chance of being tossed.  Sentence: If one of the two had been such that it would have landed Heads if tossed, then both of them would have been.  Given independence, chance of consequent is 1/4 and chance of antecedent is 3/4, so Equation style reasoning says chance of 1/3.  But for the vF version of us, it looks like chance will be near 1/4, since conditional on falsity of antecedent, closest world where both are tossed is probably the same as the closest world where the hard-to-toss one is tossed.  
\end{itemize}

\section{Context-dependence (Kaufmann,
McGee\ldots{})}\label{context-dependence-kaufmann-mcgee}

\begin{itemize}
\item
  somewhere: discuss the derivation of And-to-if from CEM, and praise
  this as demonstrating the virtues of simplicity.
\end{itemize}

\end{document}
