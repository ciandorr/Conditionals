\documentclass[If.tex]{subfiles}

\begin{document} 
\chapter*{Introduction} \addcontentsline{toc}{chapter}{Introduction} 
There is a long tradition of treating ‘if’ (or perhaps ‘if\ldots then\ldots’) as a binary connective, expressing a function from ordered pairs of propositions to propositions. The proposition expressed by a particular ‘if’ sentence on a given occasion is the result of applying the function contributed by ‘if’ to two other propositions, the antecedent and the consequent, which are expressed on that occasion by the  subordinate clause and the main clause respectively. For example, the proposition expressed by
\begin{prop}
	\nitem \label{indicative} 
	If Jack is in the park now, Jill is in the park now
\end{prop}
on an occasion of utterance is the result of applying the function contributed on that occasion by ‘if’ to the propositions contributed on that occasion by ‘Jack is in the park now’ and ‘Jill is in the park now’. The antecedent and consequent needn't be sentences that would serve as standalone declarative utterances. For example, in
\begin{prop}
	\nitem \label{counterfactual} 
	If I resigned tomorrow, I would be hired by Google the day after tomorrow
\end{prop}
‘I resigned tomorrow’ would not make sense to assert on its own: nevertheless, it is natural to think of it as contributing a proposition, namely the same one that would be expressed by a standalone utterance of ‘I will resign tomorrow’.%
\footnote{We use ‘antecedent’ and ‘consequent’ for propositions. Others use these words for linguistic items, either what we call the ‘subordinate clause’ and ‘main clause’ of the ‘if’-sentence (e.g. ‘I resigned tomorrow’ and ‘I would be hired by Google the day after tomorrow’ in the case of \ref{counterfactual}), or certain standalone declarative sentences derived by applying certain transformations to these clauses (e.g. ‘I will resign tomorrow’ and ‘I will be hired by Google the day after tomorrow’ in the case of \ref{counterfactual}).  Note that expressions like ‘if Jack is in the park now’ and ‘if I resigned tomorrow’, that include the word ‘if’, are also grammatically clauses---we will call these expressions ‘if’-clauses, and reserve the word ‘subordinate clause’ for the material following ‘if’.}

This model is prima facie plausible for many ‘if’-sentences. But there are some ‘if’ sentences for which it is completely hopeless. Consider
\begin{prop}
	\nitem \label{llamas} 
	If a farmer owns llamas, the farmer is rich
\end{prop}
What two propositions would be the inputs to the function putatively contributed by ‘if’ on this occasion? Certainly ‘the farmer is rich’ does not contribute the proposition that the one and only farmer is rich, or the proposition that the one and only farmer with such-and-such feature is rich, or a proposition concerning the richness of any particular farmer. If we were determined to force this sentence into the two-proposition model, the best option would be to take the consequent to be the proposition that all the llama-owning farmers are rich, and the antecedent the proposition that there is a llama-owning farmer. But this seems a bit of a stretch; and it is far from obvious what linguistic mechanisms would be responsible for associating those two propositions with the subordinate and main clauses of \ref{llamas}.

The dominant strand in the philosophical literature deals with sentences like \ref{llamas} by silently ignoring them. But there are other cases which are arguably like \ref{llamas} but which sometimes have been shoehorned into the two-proposition model, such as \ref{doeswill}:
\begin{prop}
	\nitem \label{doeswill} 
	If Jack sees Jill next week, he will wave
\end{prop}
Here there is a natural candidate to be the antecedent, namely the proposition that Jack will see Jill next week. But what would be the consequent? The proposition that Jack will wave sometime in the future? The proposition that Jack will wave sometime next week? More plausible candidates are the proposition that every time Jack sees Jill next week he will wave at Jill, or the proposition that on at least one occasion next week Jack will see and wave at Jill. But these \emph{ad hoc} reconstructions are not much use to systematic theorising; and it would be a mistake to take it for granted that a systematic theory will conform to the two-proposition model at all.

Another kind of case where the application of the two-proposition model is questionable is where certain kinds of modals occur in the main clause of a conditional:
\begin{prop}
	\nitem 
	If you steal from your employer, you should only steal a little.
	\nitem \label{cant} 
	If he is in the pub, he can't be at home.
	\nitem 
	If he is in the pub, he might be at home.
\end{prop}
These sentences could perhaps be assimilated within the two-proposition model, although there is pressure to think that the consequent is something rather different from the proposition that would be naturally be expressed by a standalone occurrence of the main clause. There is also a tradition in which certain sentences like these are reconstructed in such a way that there is a mismatch between the real and apparent scopes of the modals which on the surface occur in the main clause, so that the content of \ref{cant}, for example, would be more perspicuously represented by ‘It can't be that (if he is in the pub, he is at home)’. And some (e.g.~Kratzer ***) reject the application of the two-proposition model to these sentences in more radical ways.%
\footnote{Note that linguistically ‘will’ and ‘would’ are also modals\ldots{}}

One other kind of problem case for the two-proposition model is exemplified by
\begin{prop}
	\nitem \label{any} 
	If Fred had eaten any kind of shellfish, he would have been sick
\end{prop}
Some will be comfortable assimilating \ref{any} within the two-proposition model, claiming that its antecedent is the existential proposition that Fred eats some kind of shellfish (during the relevant time period). But we are not sure this is right: thinking about the difference between the natural reading of \ref{any} and the natural reading of \ref{any2},
\begin{prop}
	\nitem \label{any2} 
	If Fred had eaten any kind of shellfish, he would have eaten lobster
\end{prop}
one feels that the subordinate clause in \ref{any2} is much more apt than the one in \ref{any} for a mundane existential treatment. (The ‘any’ in \ref{any} strikes us as somewhat reminiscent of the indefinite article in \ref{llamas}, ‘If a farmer owns llamas, he is rich’.) Moreover, the reasons for being cautious in the case of \ref{any} arguably carry over to \ref{or}, whose contrast with \ref{or2} is similar to that between \ref{any} and \ref{any2}:
\begin{prop}
	\nitem \label{or} 
	If Fred had eaten either lobster, crab, or prawns, he would have been sick.
	\nitem \label{or2} 
	If Fred had eaten either lobster, crab, or prawns, he would have eaten lobster.
\end{prop}

We think it is important for an adequate theory of conditionals to account for all these sentences. Nevertheless, in the first three chapters of this book, we will focus entirely on `if' sentences like \ref{indicative} and \ref{counterfactual}, where it is especially plausible that the meaning somehow involves two propositions and where there is likely to be little dispute about which propositions they are. We will defend a view on which these sentences express propositions that are evaluable for truth and falsity, and we will attempt to say something systematic about the kind of relation that needs to hold between the antecedent and the consequent for the proposition expressed by the whole conditional to be true. Note that in developing a theory of this sort, we will not need to make any particular assumptions about the syntactic structure of the relevant `if' sentences, or about the semantic contribution of the word ‘if’. It is compatible with our view, for example, that ‘if’ doesn't have a semantic value at all, so that the semantic value of the ‘if’-clause ‘if Jack is in the park now’ is the same as the semantic value of ‘Jack is in the park now’ (just as, e.g., ‘snow is white’ and ‘that snow is white’ arguably have the same semantic value in ‘I believe that snow is white’).  On this picture, that the determination of the semantic value of the whole conditional sentence will be governed entirely by compositional rules and/or the semantic values of unpronounced constituents.%
\footnote{Examples like ‘If he came to the party and if he behaved himself, he was invited back’ are especially problematic for the idea that the semantic value of ‘if’ is a relation between propositions.}
Still less are we making any assumption about the semantic value of ‘then’ in ‘If p, then q’, or committing ourselves to the unlikely hypothesis that ‘if’ and ‘then’ comprise a single scattered semantic unit. All that matters for our positive view is that the contributions of the subordinate clause and main clauses in \ref{indicative} and \ref{counterfactual} are propositions (or things that in some natural way determine propositions), and the very propositions that the philosophical tradition takes them to be.%
\footnote{Examples of what we mean by ‘things that naturally determine propositions’: sets of worlds, properties of worlds, context-change potentials that map each set of worlds to the result of intersecting it with some given set, \ldots{} For now we will not need to make any assumptions about how fine-grained propositions are, although we will have more to say about this in chapter 3.}

Having narrowed our field in this way, we are in a position to state the theory we will be defending in a schematic way, as follows:
\begin{prop} 
	\item
	A conditional with antecedent $p$ and consequent $q$ is true iff either there is no accessible world at which $p$ is true, or the closest accessible world at which $p$ is true is a world at which $q$ is true. 
\end{prop}
The schema is not original to us: it has been around since the classic works of Stalnaker and Lewis. Our contribution will be in what we have to say by way of fleshing out the key technical terms of art: ‘accessible’, ‘closest’, and ‘world’. Quite obviously, different ways of cashing out this ideology will yield wildly different theories. Suppose for example that we interpret ‘accessible’ in such a way that at any given world, that world is the only accessible world. Then the schema will yield the same truth conditions as the material conditional: when the antecedent is false, it is false in all accessible worlds, and so the conditional is vacuously true; when the antecedent is true, the actualised world must be the closest accessible world where the antecedent is true (since it is the only accessible world), and so in that case the conditional is true if and only if the consequent is true. Our preferred take on the relevant ideology, which works very differently from this, will emerge over the course of the next three chapters, which consider ‘accessible’, ‘closest’, and ‘world’ respectively.

\begin{itemize} 
	\item Be clearer here about: (a) the way in which we are treating words like ‘accessible’ in the light of our contextualism, and (b) the idea that we are identifying (at least up to necessary equivalence) the propositions expressed by conditional utterances / sentences in contexts. 
	\item See if we are being consistent in using “conditional” to mean a kind of sentence not a kind of proposition. If necessary say that we are going to leave it up to context to decide. 
	\item Add: a big table of logical principles validated by the view, plus another table listing some that aren't validated by the view. Report completeness results due to Stalnaker or whoever. 
	\item Be more wry about the oldie-but-goodie character of the project. 
	\item
	Add a chapter-by-chapter summary.  In the part for the ‘worlds’ chapter include warnings about world.
	\item
	The strangeness of ‘does-will’ (and ‘does-does’) indicatives---one aspect is that ‘He won't go because if he does it'll be bad’ is fine in a way that, say, ‘He won't go because if he will go he'll have a bad time’ is not.  (And it doesn't change if you mention a particular time.)
\end{itemize}

\end{document}