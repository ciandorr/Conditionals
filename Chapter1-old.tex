Chapter 1: Accessibility

\subsection{The context-sensitivity of
conditionals}\label{the-context-sensitivity-of-conditionals}

Even holding fixed which propositions are the antecedent and consequent,
there are different propositions that could be expressed by uttering a
conditional on different occasions. For one thing, it is well known that
the difference between `indicative morphology' and `counterfactual
morphology' is semantically significant, as exemplified by Ernest
Adams's famous minimal pair:

\begin{enumerate}
\def\labelenumi{(\arabic{enumi})}
\item
  If Oswald didn't kill Kennedy, someone else did
\item
  If Oswald hadn't killed Kennedy, someone else would have
\end{enumerate}

This difference seems to be driven by something about the pattern of
tense markings that appear on the surface within the main and
subordinate clauses; however, the way in which these morphological
differences actually contribute to meaning is quite controversial. For
the moment, we will take a capacity to sort ``indicative'' from
``counterfactual'' conditionals for granted, and we will try only to use
examples whose classification is uncontroversial.

For another thing, the proposition expressed by a conditional can vary
even when the wording remains exactly the same. In the case of
counterfactuals (i.e.~sentences syntactically like (2)), this
variability can be illustrated by an example of Quine's:

\begin{enumerate}
\def\labelenumi{(\arabic{enumi})}
\setcounter{enumi}{2}
\item
  If Caesar had fought in Korea, he would have used nuclear weapons
\item
  If Caesar had fought in Korea, he would have used catapults
\end{enumerate}

Given an appropriate conversational background, it seems that one could
speak the truth by uttering each of these sentences, though in no normal
background could one speak the truth by uttering their conjunction.
Similarly, each of (5) and (6) seems like something one could use to
assert a truth, although their conjunction seems absurd:

\begin{enumerate}
\def\labelenumi{(\arabic{enumi})}
\setcounter{enumi}{4}
\item
  If I had jumped out the window right now, I would have been killed
\item
  If I had jumped out the window right now, I would have done so only
  because there was a soft landing laid out for me
\end{enumerate}

In the case of indicatives (i.e.~sentences syntactically like (1)), the
case for context-sensitivity is more controversial, but still strong.
The main observations here are due to Gibbard (****), although the
following case is due to Bennett. The setting is one where there are two
channels by which water can flow from an upper reservoir to a lower
reservoir; each is closed off by a gate. Speaker A discovers that the
east gate is closed and asserts

\begin{enumerate}
\def\labelenumi{(\arabic{enumi})}
\setcounter{enumi}{6}
\itemsep1pt\parskip0pt\parsep0pt
\item
  If the water got to the bottom, it got there via the west channel and
  not the east channel
\end{enumerate}

Speaker B discovers that the west gate is closed and asserts

\begin{enumerate}
\def\labelenumi{(\arabic{enumi})}
\setcounter{enumi}{7}
\itemsep1pt\parskip0pt\parsep0pt
\item
  If the water got to the bottom, it got there via the east channel and
  not the west channel
\end{enumerate}

Intuitively, both speeches seem true; but the conjunction is extremely
odd. It is thus very natural to explain the acceptability of (7) and (8)
by appeal to context-dependence.

Our view is that all of the differences we have just been talking about
are to be explained by differences in what it takes for a world to be
``accessible'' in the sense relevant to the context. For example, in the
context where (3) seems true, accessibility requires match with respect
to what kinds of military technology is in use at a given historical
period, whereas the context where (4) seems true, accessibility requires
match with respect to what kinds of military technology were actually
available to particular people.\footnote{Note that if, unbeknownst to
  us, there are no such things as nuclear weapons and catapults are
  still in wide use even now, the true-seeming utterances of (3) are in
  fact false; similarly, if unbeknownst to us, Martians actually
  provided Caesar with nuclear weapons which he only didn't use because
  he had no need to resort to them, the true-seeming utterances of (4)
  are in fact false. \textbf{Perhaps put this later and in the main
  text. }} Similarly, in the context where (5) seems true, accessibility
requires match with respect to the laws of nature and the (approximate)
course of history up until very shortly before the time of utterance,
whereas in the context where (6) seems true, the constraints on matching
history are much weaker.\footnote{Explain and reference CD's paper.} In
the contexts where (7) and (8) seem true, meanwhile, accessibility
typically involves something like compatibility with the knowledge of
the speaker at the time of utterance, and the contextual variability is
primarily due to variation in who is speaking and when.

Notice that according to these suggestions about what plays the role of
accessibility in the natural interpretations of (3--8), there is in each
case at least one ``accessible'' world at which the antecedent of the
conditional is true, and moreover the fact that there is such a world is
something that the speaker is in a position to know. As we are thinking
about things, this is characteristic of conditional utterances (both
indicative and counterfactual): the intended interpretation almost
always sets accessibility in such a way that it can be taken for granted
that the conditional is not vacuously true: i.e.~it is not the case that
both it and the conditional with the same antecedent and the negated
consequent are true. Label this phenomenon `the presupposition of
non-vacuity'. We will defend and explore the claim that this
presupposition exists in section ***, but throughout the discussion we
shall be helping ourselves to the idea that this is one of the guiding
constraints on the interpretation of conditionals.\footnote{*** Some use
  `presupposition' in such a way that the claim that a sentence
  presupposes a certain proposition entails that that sentence has no
  truth-value at all if the proposition in question is false. As our
  ideology of ``vacuous truth'' shows, we are emphatically not using
  `presupposition' in that way.}

--- be more explicit that `accessibility' is a property of worlds not a
set of worlds

\subsection{The difference between indicatives and
counterfactuals}\label{the-difference-between-indicatives-and-counterfactuals}

Given our framework, it is clear that the grammatical difference between
indicative and counterfactual conditionals goes along with some
systematic difference in the resolution of the context-sensitivity of
`accessibility'. We propose that the key difference lies in a general
constraint on indicatives which does not apply to counterfactuals,
namely, that for an indicative conditional, \emph{accessibility must
entail epistemic possibility}. Roughly, to say that a world is
epistemically possible is to say that it is a live candidate for being
how things actually are, from the perspective of some contextually
relevant individual or group. This explains why we are forced to use
counterfactual morphology to get across the plausible thought we
naturally would try to get across by uttering (3). Given the
presupposition of non-vacuity, this thought involves some accessibility
property that is possessed by some worlds where Caesar fought in Korea.
It is very hard to get oneself into a mindset where such worlds are
treated as ``live''. (And it is harder still to get oneself into a
mindset where there are ``live'' worlds in which Caesar was present in
Korea and where there is no special sort of deception involved in the
transmission of information about his activities.) Similarly,
conditionals beginning with `If I didn't exist right now\ldots{}' are
generally much more intelligible than those beginning with `If I don't
exist right now\ldots{}': processing the latter requires entering into a
most unusual sense of openness to nihilistic metaphysics.\footnote{Cite
  and discuss von Fintel, `The presuppositions of subjunctive
  conditionals'; be neutral as regards which is the `default'. Flag that
  you could get it for one and then derive it for the other via
  `maximise presupposition'.}

This characterisation of how accessibility works for indicative
conditionals is, for us, intimately related to a certain natural account
of epistemic modals: `It must be that Φ'; `It might be that Φ'; `It is
possible that Φ', etc. In orthodox fashion, we take `It must be that Φ'
to be true iff Φ is true at all accessible worlds, and `It might be that
Φ'/`It is possible that Φ' to be true iff Φ is true in some accessible
world. What it takes for a world to be accessible is a context sensitive
matter: but the ``epistemic'' character of these modals consists in the
fact that accessibility always has to entail epistemic liveness (from
the standpoint of the contextually relevant individual or group---in
simple cases, typically the speaker or a group containing the speaker).
When one uses both epistemic modals and indicative conditionals, the
default is for both to be interpreted using the same notion of
accessibility. Thus, when this parameter of interpretation is held
fixed, the proposition expressed by `It must be that Φ or Ψ' entails the
one expressed by the indicative conditional `If not Φ, then Ψ'. Also,
the propositions expressed by `It might be that Φ' and `If Φ, then Ψ'
jointly entail the proposition expressed by `It might be that Ψ'.

Given the presupposition of non-vacuity, an utterer of (1) (`If Oswald
didn't kill Kennedy, someone else did') is taking for granted that there
are accessible worlds in which Oswald didn't kill Kennedy, which on the
liveness-theoretic gloss requires that its is a live epistemic
possibility that he did not do so. One might think this a problematic
feature of the view: after all, most of us know perfectly well that
Oswald did kill Kennedy and yet are happy to utter (1). There are a few
different routes one might follow in responding to this objection. On a
radical view, the proposition expressed by utterances of (1) by us
knowledgeable folk is in fact vacuously true, but when we utter it we
take for granted something we know to be false, either out of a kind of
pretended epistemic modesty, or else a false belief that we don't really
know who shot Kennedy. Similarly, the radical view will say that while
the sentence `I might be dreaming' is natural, utterances of it will
express a false proposition that is treated as acceptable either because
we are pretending that it is true, or because we falsely believe that it
is true (since at the time of utterance we are taken in by the thought
that we don't know whether we are dreaming). On a less error-theoretic
view, the presupposition of nonvacuity introduced by (1) is indeed
satisfied: in the relevant sense of accessibility, there is an
accessible world in which Oswald didn't kill Kennedy. This could be
reconciled with the thesis that accessibility for indicative
conditionals requires liveness in a few different ways. One option is to
appeal to a contextualism about the verb `know': in the context invoked
by (1), `We don't know whether Oswald killed Kennedy' expresses a truth,
although it expresses a falsehood in many other contexts. Another option
is to allow our notion of liveness to have a kind of context-sensitivity
that does not always march in lock-step with `know': for example,
liveness could in some cases be a matter of consistency with some more
attenuated body of knowledge, or liveness might be a matter of
consistency with a body of propositions of which one has an especially
secure kind of knowledge.

\begin{itemize}
\itemsep1pt\parskip0pt\parsep0pt
\item
  Note to selves: why it's so much easier to say `If P then Q' than `It
  might be that P' (for obviously false P). Diagnosis: presupposition
  leads to quiet accommodation.
\end{itemize}

Epistemic modals need not always be anchored to the facts about what is
live at the time of speech. In a sentence like `It was possible that it
had rained, since the ground seemed a bit damp', the operative epistemic
perspective is in the past: what we are saying, roughly, is that what
the relevant people knew at the relevant past time was consistent with
the proposition that it had rained earlier than that time. There is no
suggestion that the speaker is ignorant or in any way open at the time
of speech to the possibility that it had rained: what matters is just
the epistemic standpoint of the salient person or group (or more
abstract `perspective') as of the target time.\footnote{In some cases
  one needs to work with such notions as `the evidence that was
  available at the time' even when there is no relevant person around to
  gather it:} `It was possible that it was raining' and `It was possible
that it was going to rain' are similar. `Might have' and `could have'
claims also have a reading that works like this: `It might have been
raining' can mean `It was possible that it was raining', although it can
also mean `It is possible that it has been raining'.\footnote{In some
  other natural languages where the analogues of `might' are normal
  tensed verbs, this ambiguity is lexically resolved. In English `have
  to' works like this: we distinguish `It has to have been raining' from
  `It had to be raining'.}

Given the intimate relation we posit between epistemic modals and
indicative conditionals, we would thus expect that in some cases, the
accessibility parameter of an indicative conditional is tied to an
earlier epistemic perspective. This does indeed seem to be going on in
examples like the following:

\begin{enumerate}
\def\labelenumi{(\arabic{enumi})}
\setcounter{enumi}{8}
\item
  If Oswald hadn't killed Kennedy, someone else had
\item
  If Oswald hadn't killed Kennedy, someone else would
\item
  If Oswald didn't kill Kennedy, someone else was going to succeed in
  doing so
\item
  If it wasn't Oswald killing Kennedy, it was someone else
\end{enumerate}

Just like `It was raining' and `Einstein was going to win a Nobel
Prize', these are sentences that could not be felicitously asserted out
of the blue: some particular past time has to be salient, and the
proposition asserted involves that time somehow or other. (The required
salience could be achieved either by earlier discourse or by some
nonlinguistic clues.) In the case of (9)--(12), the most natural reading
is one on which one of the roles of this salient past time is that of
providing the relevant epistemic perspective. None of these sentences
commits the speaker to accepting `It \emph{is} possible that Oswald
didn't kill Kennedy', or to treating this as a live possibility: but
they do seem to commit us to a claim about what \emph{was} epistemically
possible at the time in question. Note that one natural use for
sentences like (9--12) is in indirect speech reports of present-tense
speeches made at the relevant time, namely

(9') If Oswald hasn't killed Kennedy, someone else has

(10') If Oswald hasn't killed Kennedy, someone else will

(11′) If Oswald doesn't kill Kennedy, someone else is going to succeed
in doing so

(12′) If it isn't Oswald killing Kennedy, it is someone else

On our account, the very proposition that would be asserted by
(9′)--(12′) can in fact later be asserted or reported by uttering
(9)--(12).

Although there are some syntactic similarities between the likes of
(9--12) and (2) (`If Oswald hadn't killed Kennedy, someone else would
have'), our view is that there is a significant semantic gulf between
them. On its natural interpretation, (2) does not commit us to its being
a live possibility that Oswald didn't kill Kennedy, either from our own
perspective or from any other perspective---intuitively, its meaning
feels altogether more ``worldly'' by comparison with the
``perspectival'' feel of (9--12). This is what we have been getting at
in our use of the labels `indicative' and `counterfactual', roughly in
line with the philosophical tradition. Note however that examples like
(9--12) are rather different from the paradigmatic examples of
indicative conditional sentences, and would provide counterexamples to
many generalisations that philosophers have been wont to make about
indicative conditionals (e.g.~generalisations about how the probability
or ``assertability'' of an indicative conditional relates to the
corresponding conditional probability), or about the superficial
linguistic form distinctive of indicatives and/or counterfactuals (for
example, (10) shows that a `would' in the main clause does not suffice
for counterfactuality).

Indeed, as we see it, there is no failsafe way to distinguish
counterfactuals from indicatives on the basis of superficial syntax,
since many conditional sentences admit of both kinds of interpretation.
For example even though the dominant reading of (2) is counterfactual,
it does have a ``past-perspective'' indicative meaning, which comes to
the fore when we imagine using (2) in reporting a past utterance of the
following somewhat unusual but perfectly intelligible sentence:

\begin{enumerate}
\def\labelenumi{(\arabic{enumi})}
\setcounter{enumi}{12}
\itemsep1pt\parskip0pt\parsep0pt
\item
  (2′) If Oswald hasn't killed Kennedy, someone else will have
\end{enumerate}

And once this epistemic interpretation has been noticed, one can imagine
contexts in which it is the intended interpretation even outside
indirect speech reports.

There are, however, some conditionals for which a counterfactual meaning
is grammatically required. The most prominent examples are ones in which
the main verb in the subordinate clause takes a subjunctive form:

\begin{enumerate}
\def\labelenumi{(\arabic{enumi})}
\setcounter{enumi}{13}
\item
  If Gore were president, he would deal with this problem
\item
  If I were to resign tomorrow, I would be hired by Google the day after
  tomorrow.
\end{enumerate}

Putting certain present-tense verbs in the main clause also seems to
close off the possibility of indicative readings:

\begin{enumerate}
\def\labelenumi{(\arabic{enumi})}
\setcounter{enumi}{15}
\item
  If I resigned tomorrow, I reckon I would be hired by Google the day
  after tomorrow.
\item
  If I got a tattoo, it is unlikely that anyone would notice.
\end{enumerate}

We won't try to give a theory of what's going on in these sentences yet,
but it is plausible that the unavailability of indicative readings can
be traced to the same source as the unacceptability of sentences like
`It was possible that I am happy'. There are also cases where a
counterfactual reading seems to be required on semantic grounds:

\begin{enumerate}
\def\labelenumi{(\arabic{enumi})}
\setcounter{enumi}{17}
\item
  If I had married someone other than the person I did marry, I would
  not be happy.
\item
  If giraffes were any taller than they actually are, they wouldn't be
  able to pump blood up to their brains
\end{enumerate}

In these cases, it is plausible that what forces the counterfactual
reading is the fact that the antecedent isn't live from any reasonable
perspective.\footnote{When (15) and (16) are embedded they need not
  always be counterfactuals: for example, `He said that if I had married
  someone other than the person I did marry, I would not be happy' could
  be a felicitous report of a past utterance of the `If he has married
  someone other than Jessica, he will not be happy'.}

An idea we like is that counterfactual uses of conditionals are marked
by the occurrence of what Iatridou (***) calls `fake past tense': an
extra layer of past tense morphology is used in a way that does not
carry the usual significance of temporal pastness. This apparently
non-temporal use of the past is not confined to conditionals: it also
occurs with certain modals and attitude verbs, as in the following
examples.

\begin{enumerate}
\def\labelenumi{(\arabic{enumi})}
\setcounter{enumi}{19}
\item
  I wish they were here right now
\item
  Oh, that they had been here right now!
\item
  They might have been here right now.
\item
  They couldn't have been here right now.
\end{enumerate}

Note that in general, `might have been' also has uses where the past
tense is genuinely tied to the temporal past---as noted earlier, 'It
might have been raining' can mean `It is possible that it has been
raining' and `It was possible (then) that it was raining (then)'. But
the use of `right now' in (21) seems to preclude these readings:
plausibly for reasons similar to those that make for the badness of `It
is possible that they have been here right now' and `It was possible
that they are here right now'.

\begin{itemize}
\itemsep1pt\parskip0pt\parsep0pt
\item
  Find some cross-linguistic examples, and discuss Iatridou's
  cross-linguistic evidence.
\end{itemize}

In many paradigmatically counterfactual conditionals, the subordinate
clause involves pluperfect morphology, which can be thought of as
involving two ``layers'' of past tense. In an ordinary use of the
pluperfect both layers play a temporal role: `I had eaten breakfast'
takes us to a past ``reference time'' and then places an eating event
earlier than that time. However, as Iatridou notices, there is no
similar sense of double pastness in the natural counterfactual use of
`If I had eaten breakfast this morning, I would have skipped lunch'. Her
view of these cases---which we like---is that at least one layer of past
tense is fake past, while at most one is the usual temporal past. (In
`If it had been raining right now, the ground would have been wet', both
layers seem to be fake.)

Why should the past tense be ambiguous in this way? An intriguing but
elusive idea of Iatridou's is that the past tense has a more skeletal
core meaning of ``distance'', which can be cashed out either temporally
or modally---in our framework, the relevant thought in the modal case
would presumably be one to the effect that non-live possibilities are
``distant'', so that invoking a notion of accessibility that extends to
the non-live requires reaching out to the distant. Obviously this
picture raises many questions---for example, why do actual future events
not count as ``distant'' in the more abstract sense and thus apt to be
described by verbs in the past tense? But we won't try to address such
questions here: as with many other facts about linguistic structure,
recognising that the phenomenon of fake past tense exists does not
require having an explanation of why language is so configured.

We have posited a interpretative link between indicative conditionals
and epistemic modals: barring context-shift, they force the same
interpretation of accessibility. Given that fake past tense can occur
with modals, as in (21) and (22), it is quite plausible to posit that
such modals bear an analogous relation to counterfactual conditionals.
If so, the non-epistemic reading of `It couldn't have been that P and Q'
will (holding context fixed) entail `If it had been that P, it would
have been that Q'. Also, `It could have been that P' and `If it had been
that P, it would have been be that Q' jointly entail `It could have been
that Q'. Also, the presupposition of non-vacuity will mean that `If it
had been that P, it would have been that Q' will be presuppositionally
bad unless `It could have been that P' is true. These claims seem no
less plausible than the corresponding claims about indicatives.

There is a competing picture of the role of the past-tense morphology in
counterfactual conditionals which rejects the idea of `fake past', and
instead regards the relevant uses of the past tense as introducing
reference to genuine past times in much the same way that we see in
past-perspective indicative conditionals like (9--12). As developed,
e.g., by Condoravdi (****) and Khoo (****), the idea is that just we can
talk about a possibility being epistemically live or not at a given
time, we can talk about a possibility being ``metaphysically live'' or
not at a given time, where the metaphysically live possibilities at a
time are (roughly) those which share their history up to that time. The
picture is that when we utter a counterfactual, there is a particular
time earlier than the time of utterance such that the accessible worlds
are all and only those that are live as of that time, and at least one
of the past-tense morphemes in the conditional refers to that past time.

There are several problems with this view. First, it will have trouble
explaining why standard counterfactuals can be acceptable even when no
particular past time has been raised to salience as a target to be
referred to by the relevant morphemes---by contrast, `It was raining'
clearly does require such salience to be in place, and so do
past-perspective indicatives like (9--12). Second, consider
counterfactuals about the future like `If I resigned next week I would
be hired by Google the week after'. Even granting that the set of
accessible worlds consists in all those whose history matches that of
the actual world up to some given time, there is little reason to think
that the time in question is in the past: it is much more plausible that
the time of divergence is identical to or after the time of utterance,
given that in evaluating such counterfactuals we freely draw on known
truths about the history of the actual world right up to the time of
utterance. Third, it is just not plausible the notion of accessibility
relevant to the evaluation of counterfactuals always requires matching
any part of the history of the actual world: consider `If gravity had
obeyed an inverse cube law, stars would have been unstable', `If there
had always been infinitely many stars, then there would always have been
infinitely many planets', etc. Fourth, it is very hard to see what
reference to a past time could be going on in `I wish he were here right
now'; but once we admit that the past tense is fake here, what reason is
there to think it is genuinely temporal in `If he were here right now,
then he would be happy', especially considering that the word `were' can
do double duty in `If, as I wish, he were here right now, he would be
happy'.\footnote{***Condoravdi and Khoo argue for their view on the
  grounds that it explains why a present-present conditional like `If
  it's raining the ground is wet' can only have epistemic readings. Why
  we are not impressed by this.

  *** Note that in (10) and the past-perspective indicative reading of
  (2), the time of the epistemic perspective has to be to a time AFTER
  the (actual-world) killing of Kennedy. This makes it implausible that
  the counterfactual reading of (2) is just the same thing with
  ``metaphysical'' rather than epistemic accessibility. You might think
  that the view was getting you some kind of economy by letting you
  treat `had been' as functioning in the same way in conditionals as
  elsewhere, but it's not at all.}

\begin{itemize}
\item
  Present tense perspective-shifted uses
\item
  Note somewhere or other: you can quantify into the accessibility
  parameter
\end{itemize}

\subsection{Quasi-validity and the or-to-if
inference}\label{quasi-validity-and-the-or-to-if-inference}

Considering `arguments' as sequences of sentences, let's say that an
argument is valid iff the propositions expressed by the premises jointly
entail the proposition expressed by the conclusion, on any uniform
resolution of context-sensitivity. And let's say that an argument is
`quasi-valid' iff the argument that results from it when `It must be
that\ldots{}' is affixed to each of the premises is valid. Many
arguments that are quasi-valid but not valid feel intuitively like
excellent deductive arguments. For example:

\begin{enumerate}
\def\labelenumi{(\arabic{enumi})}
\setcounter{enumi}{23}
\itemsep1pt\parskip0pt\parsep0pt
\item
  Either this is a horse or it's a donkey
\end{enumerate}

\begin{enumerate}
\def\labelenumi{(\arabic{enumi})}
\setcounter{enumi}{24}
\item
  It's not a horse.
\item
  So it must be a donkey.
\end{enumerate}

Even someone with an excellent logical training might naturally and
unreflectively answer `yes' when asked whether this is valid.
Nevertheless on reflection it is clearly not valid in our sense, or on
any of the standard ways of defining validity. \textbf{***}(taking it for
granted it expresses a proposition. It's still a great argument though,
and any full account of `must' needs to explain (i) why this is so, and
(ii) why this kind of awesomeness is so readily mistaken for validity
(contrast? the result of replacing `must' with `we ought to believe
that')***}

One prominent argument for the `materialist' view of indicative
conditionals, according to which `If P, Q' has the same truth-conditions
as `Either not P or Q', appeals to the seeming excellence of `or-to-if'
arguments: i.e.~arguments of the form `P or Q; therefore if not P, Q'.
If we took this excellence as a sign of validity proper, it would be a
short step to materialism.\footnote{***Fill this in from notes. `The
  only further resources needed to derive the logical equivalence of `If
  P, Q' and `Not-P or Q' are the validity of double-negation
  elimination, the law of excluded middle, modus ponens and proof by
  cases.' To get from `If P, Q' to `not-P or Q', use LEM to get `not-P
  or P', and then proof by cases + MP to get not-P or Q'.} However,
given that quasi-valid arguments as well as valid ones can be expected
to strike us as excellent, our view provides a ready rejoinder to this
argument. For recall that, given the connection we posit between
epistemic modals and indicative conditionals, `It must be that (P or Q),
therefore if not P, Q' is valid on our view. Hence arguments of the form
`P or Q, therefore if not P, Q' are quasi-valid, and just as excellent
as the intuitively excellent argument (23) above. We don't see any
pre-theoretic pressure to think that the or-to-if argument is even
better than the likes of (23).

Note that this feature of our view depends on our decision to let the
conditional be true when there is no accessible world in which the
antecedent is true. If we had instead gone for a view on which
conditionals are false in this case, we would need to say something more
complicated, and perhaps less satisfying, about the or-to-if
inference.\footnote{This variant of our view is also unappealing in
  other ways: most importantly, it entails that sentences of the form
  `If P, P' are false in many contexts.}

(A related argument for materialism appeals not to intuitions about the
validity of arguments, but to intuitions about the logical truth of
single sentences of the form `If P or Q, then if not-P, Q'. The notion
of quasi-validity does not provide a response to this form of argument,
since there is no corresponding notion of `quasi-logical-truth': if we
are willing to speak of arguments with zero premises, then
quasi-validity will coincide with validity for such arguments. We think
there is a good answer to this objection too, but we will not have the
resources to explain it until we get to consider conditionals which
embed other conditionals in chapter ***. But for the present, note that
it would be difficult to endorse this as an argument for materialism
unless one also endorsed the apparent tautologousness of `If this is
either a horse or a donkey and it's not a horse, then it must be a
donkey' as an argument for the vacuity of `must'.)

It is worth comparing our response to the `or-to-if' argument for
materialism with a somewhat similar response offered by Robert
Stalnaker. On Stalnaker's view, the distinctive role of epistemic
possibility in the semantics for indicative conditionals consists not in
the exclusion of epistemically impossible worlds from the domain of
accessibility (as on our view), but in the fact that indicative
conditionals impose a distinctive kind of closeness ordering, in which
any epistemically possible world must count as closer than every
epistemically impossible world. On his view, asserting an indicative
conditional also carries the presupposition that its antecedent is
epistemically possible. These commitments lead Stalnaker to somewhat
different diagnoses of the status of the argument-forms we have been
discussing. For him, the or-to-if inference is not quasi-valid in our
sense, since `Must (P or Q); therefore if not P, Q' is not valid. If it
is epistemically necessary that P and the closest not-P-world is a not-Q
world, then `Must (P or Q)' will be true even though `If not-P, Q' is
false. But this argument-form does have another status for Stalnaker: if
the premise is true and the presupposition of the conclusion is
satisfied, the conclusion is true. Stalnaker thinks that the fact that
his view secures this good status is enough to undermine the case for
materialism based on the seeming validity of the or-to-if argument. But
notice that the status in question is also possessed by some arguments
which don't intuitively seem that great, for example

\begin{enumerate}
\def\labelenumi{(\arabic{enumi})}
\setcounter{enumi}{26}
\item
  No-one will be smoking in the year 2050; therefore Fred will have
  stopped smoking by 2050
\item
  John regrets everything he has ever done; therefore John regrets
  killing his mother
\item
  Every creature in the room is purring; therefore, the elephant in the
  room is purring
\end{enumerate}

So \emph{prima facie}, our view does a better job than Stalnaker's at
explaining what is good about the or-to-if inference. It is worth
comparing our response to the `or-to-if' argument for materialism with a
somewhat similar response offered by Robert Stalnaker. On Stalnaker's
view, the distinctive role of epistemic possibility in the semantics for
indicative conditionals consists not in the exclusion of epistemically
impossible worlds from the domain of accessibility (as on our view), but
in the fact that indicative conditionals impose a distinctive kind of
closeness ordering, in which any epistemically possible world must count
as closer than every epistemically impossible world. On his view,
asserting an indicative conditional also carries the presupposition that
its antecedent is epistemically possible. These commitments lead
Stalnaker to somewhat different diagnoses of the status of the
argument-forms we have been discussing. For him, the or-to-if inference
is not quasi-valid in our sense, since `Must (P or Q); therefore if not
P, Q' is not valid. If it is epistemically necessary that P and the
closest not-P-world is a not-Q world, then `Must (P or Q)' will be true
even though `If not-P, Q' is false. For the same reason, the inference
`Must Q; therefore if P, P and Q' is not valid for Stalnaker. These
argument-forms do however have another status for Stalnaker: whenever
the premise is true \emph{and the presupposition of the conclusion is
satisfied}, the conclusion is true. Stalnaker thinks that the fact that
his view secures this good status is enough to undermine the case for
materialism based on the seeming validity of the argument forms in
question. But notice that the presupposition-theoretic status is also
possessed by some arguments which don't intuitively seem that great, for
example

\begin{enumerate}
\def\labelenumi{(\arabic{enumi})}
\setcounter{enumi}{29}
\item
  No-one will be smoking in the year 2050; therefore Fred will have
  stopped smoking by 2050
\item
  John regrets everything he has ever done; therefore John regrets
  killing his mother
\item
  Every creature in the room is purring; therefore, the elephant in the
  room is purring
\end{enumerate}

So \emph{prima facie}, our view does a better job than Stalnaker's at
explaining what is good about the or-to-if inference.\footnote{Another
  possible resource for Stalnaker is to appeal to the idea that
  utterances of `P or Q' or `Must (P or Q)' carry the implicature that
  both P and Q hold in live possibilities. (Such an implicature can
  plausibly be predicted on Gricean grounds.) Given this, any case where
  both the content and the implicature of `Must (P or Q)' are true will
  be one where `If P, Q' is true. However this point does not carry over
  to `Q; therefore if P, P and Q' and the like.}

\textbf{***Note: check on Stalnaker's paper, also Moritz Shultz \&
Robbie. }

One might worry that in going as far as we are going to accommodate the
validity-judgments that motivate materialism, we will also be saddling
ourselves with some of the problems of materialism---specifically, the
so-called ``paradoxes of material implication'', arguments that are
valid if materialism is true but don't seem intuitively good at all. One
of these ``paradoxes'' is the inference form `Q; therefore if P, Q'. On
our view, this is quasi-valid.\footnote{Similarly, Stalnaker's view
  assigns this argument has the same presupposition-theoretic status as
  the or-to-if inference.} We admit that it would be odd and suspicious
if someone actually produced such an argument in the course of
reasoning. But we don't think this is a good reason to deny that such
arguments are quasi-valid: for `This is a horse; therefore this must be
a horse' is uncontroversially quasi-valid but also seems quite strange.
If someone were to utter something like this, we would feel pragmatic
pressure to interpret the `must' in some way that would give the
conclusion some conversational point, and this will require some
interpretation that doesn't tie the `must' in a flat-footed way to what
has been established at that point in the conversation. In general, when
we are dealing with super-simple arguments, intuitions of excellence
will be tempered by the fact that such arguments have so little
conversational point that they send us looking for non-obvious
interpretations under which they are more tendentious or informative.
Notice once we turn to slightly more complex arguments in the same
family as `Q; therefore if P, Q', the intuitions of excellence start to
fall into place: for example, `P; therefore if Q, P and Q' sounds
great--- cf. `John is in the room; so if Jill is in the room, two people
are in the room'---and is quasi-valid on our account.

The second ``paradox of material implication'' is the argument form
`not-P; therefore if P, Q'. This is also valid on the materialist view
and quasi-valid on our view, and it sounds like a truly awful template
for argumentation. But note that given the presupposition of
non-vacuity, the argument `It must be that not-P, therefore if P, Q' has
the bad-making feature that if the premise is true, the presupposition
of the conclusion is violated. In this respect it is similar to
arguments like `She hasn't stolen from anyone; therefore she doesn't
regret stealing from her employer', or `Everyone is having a good time;
therefore everyone who isn't having a good time is a terrorist'. The
latter is especially pertinent. If the domain was constant throughout
the argument, then the truth of the premise would mean that the
presupposition of the conclusion was violated. If someone actually
produced such an argument, it would be natural to take the domain of the
conclusion to be different from that of the premise, precisely so as to
avoid convicting them of presupposition-failure. That is how
presupposition typically operates as a way of allowing an audience to
recognise lack of uniformity across time in the interpretation of
context-sensitive elements. But insofar as there is such a shift, the
argument is terrible.

The idea that sometimes excellent arguments are quasi-valid rather than
valid can also be used to undercut a certain case for the `strict
conditional' theory, according to which `If P, Q' has the same truth
conditions as `It must be that (either not-P or Q)'. Someone might argue
for ``strictism'' on the grounds that the excellent-seeming argument `If
P, Q; therefore it can't be that P and not-Q' should be regarded as
valid, in which case the argument `If P, Q; therefore it must be that
(not-P or Q)' is surely also valid. On our view, these arguments are
merely quasi-valid: since `If P, Q' logically entails `Not-P or Q', `It
must be that (if P, Q)' logically entails `It must be that (not-P or
Q)'.

Some other controversial argument-forms that are valid according to
materialists and strictists, but not valid on our account include
contraposition (`if P, Q; therefore if not-Q, not-P'), transitivity (`If
P, Q; if Q, R; therefore if P, R'), and antecedent strengthening (`If P,
R; therefore if P and Q, R').

\footnote{Make sure not to forget to talk about Stalnaker's distinctive
  claims about `actually'.}

\subsection{The presupposition of
non-vacuity}\label{the-presupposition-of-non-vacuity}

Up to now we have been helping ourselves to the idea that conditionals
carry a presupposition of non-vacuity; it is time we said something in
defence of this claim. The most obvious thing to be said in favour of
the thesis is that it explains what is wrong with certain kinds of
sentences that seem to almost always unacceptable. For example,
conjunctions of the form `If P, Q and if P, not Q' are almost always
unacceptable, unless P is something truly bizarre like ``All
contradictions are true''. This is true whether the conditionals are
indicative or counterfactual, and seems like the kind of phenomenon that
deserves a unified explanation. Given our decision to count `If P, Q' as
true when there are no accessible P-worlds---a decision we motivated in
the previous section---we can't explain this infelicity as semantic
falsehood in all contexts: 'If P, Q and if P, not Q' will be true in any
context where under which no P-worlds count as accessible. So we need
something pragmatic to say about these sentences, and the presupposition
of non-vacuity fits the bill quite nicely.

The presupposition of nonvacuity also helps us explain why certain
inferences which one might expect to be good given just the semantic
part of our story in fact seem problematic. We already considered one
such example in the previous section, namely the inference from `It
can't be that P' (or just `not P') to the indicative `If P, Q'.
Similarly in the counterfactual case, `It couldn't have been that P'
will entail `If P it would be that Q' for any Q, but such inferences are
intuitively highly problematic: we explain this by saying that if
accessibility relation for the conditional is interpreted as the same
one that matters for the modal, the premise entails that the
presupposition of the conditional is violated; so the discourse puts
pressure on us to invoke two different accessibility relations, in which
case the inference will be simply invalid.

However, there is a competing explanation which might be given for the
phenomena just described. The competing explanation invokes a weaker
presupposition, namely that the conditional being asserted is not
\emph{known} to be vacuously true. The weaker presupposition is, for
example, enough to explain why `If P, Q and if P, not Q' should sound
bad: after all, in uttering this conjunction one represents oneself as
knowing it; but since it could be true if both conjuncts are vacuously
true, such knowledge will easily yield knowledge of vacuity. (One could
try to make trouble for this explanation by appealing to the fact that
ordinary people may not make the relevant inference and hence may not
speaking speaking have any view at all about whether vacuity holds; but
we don't think this gets to the heart of the matters, since in semantics
it is often to helpful to use somewhat technical vocabulary in one's
representation of what language users tacitly know.) Similarly, if one
knows that it couldn't be that P, one will be in a position to know that
`if P it would be that Q' is vacuous (holding fixed the accessibility
relation), and hence unacceptable even given the weaker presupposition.

We noted earlier that our schematic analysis of `If P, Q' can yield the
truth conditions of a material conditional in the limiting case where
accessibility is understood in such a way that each world is accessible
only at itself, or more broadly when the accessible worlds all have
match actuality with respect to the truth value of the antecedent. This
kind of interpretation of an unembedded conditional will always violate
the presupposition of non-vacuity, but need not violate the weaker
presupposition of lack of known vacuity. So if one thought that many
conditionals are truth-conditionally tantamount to material conditionals
that would be evidence against the stronger presupposition, while the
view that such truth conditions are rare would support it.

While adopting this alternative presupposition wouldn't be damning for
our overall purposes, we do think there are reasons to favour the
stronger presupposition over its weaker rival. For example, (24) is
clearly a terrible thing to assert unless one believes that the number
of full-time people in the department is even:

\begin{enumerate}
\def\labelenumi{(\arabic{enumi})}
\setcounter{enumi}{32}
\itemsep1pt\parskip0pt\parsep0pt
\item
  If there had been exactly half as many full-time people in the
  department as there actually are, then we would have had to hire at
  least three adjuncts.
\end{enumerate}

This is readily explained by the presupposition of nonvacuity, since the
antecedent is metaphysically impossible to satisfy if the relevant
number is odd. But so long as you don't know whether the number is odd
or even, you don't know whether the conditional is vacuous, so the
weaker epistemic presupposition does not account for the infelicity of
(24). Along similar lines: suppose that I know one of Jack and Jill had
three jacks and a queen and the other had a worthless hand with no
queens. Both fold before the final draw; I look at the top of the deck
and see that each would have been dealt a queen. If the only relevant
barrier were the epistemic presupposition, it's hard to see why the
following wouldn't be a reasonable thought to think:

\begin{enumerate}
\def\labelenumi{(\arabic{enumi})}
\setcounter{enumi}{33}
\itemsep1pt\parskip0pt\parsep0pt
\item
  If Jill had kept playing with those four cards and had ended up with
  two queens, she would have won.
\end{enumerate}

But in fact (25) is bizarre. The presupposition of non-vacuity explains
this: an interpretation on which you knew that there was an accessible
world where the antecedent was true would require not holding fixed such
facts as that a poker hand only contains five cards, and once we let in
worlds where the rules of poker allow six-card hands, we don't know that
(25) is true. (If she in fact had the worthless hand, then the closest
six-card poker world where she has two queens as well is probably a
losing world.)

*** two more examples: (Setting: I'm not sure whether today or yesterday
is the first day of the semester. I know Ted was in town until
yesterday.) I say `If Ted had left on the first day of the semester, he
would have been in the airport this morning'. `If he had won for the
first time yesterday, he would barely have noticed the increase in his
wealth'.

The foregoing examples all involved counterfactual conditionals. With
indicative conditionals, it is much more delicate to tease the effects
of the two presuppositions apart. If we know which worlds are and are
not epistemically possible, then whenever it is consistent with our
knowledge that there is an epistemically possible world where the
antecedent of a given conditional is true, we will know that there is
such a world, so we will know what we need to know to satisfy the
stronger presupposition. Thus to distinguish the proposals one must
consider either cases where accessibility requires the application of
some further, epistemically non-transparent constraint, or else cases
where what's epistemically possible is not transparent. We will be
considering the former phenemenon in chapter 2; the most straightforward
examples of the latter phenomenon involve conditionals like (9--12) from
section ***, where the relevant epistemic perspective is in the past.
Consider for example

\begin{enumerate}
\def\labelenumi{(\arabic{enumi})}
\setcounter{enumi}{8}
\itemsep1pt\parskip0pt\parsep0pt
\item
  If Oswald hadn't killed Kennedy, someone else had
\end{enumerate}

Consider a scenario where we aren't sure which of two scenarios obtain.
In scenario A, the relevant past people had evidence that someone or
other had killed Kennedy, and were unsure whether or not it was Oswald.
In scenario B, they knew Oswald had killed Kennedy, and this was their
entire basis for thinking that anyone had. (9) is not intuitively
felicitous if this is all we now know. But if the only relevant
presuppositional requirement were one of lack of \emph{known} vacuity,
we would presumably be able to interpret (9) in such a way that the
accessible worlds were just those consistent with what was known at the
past time, so that (9) is non-vacuously true in scenario A and vacuously
true in scenario B, and hence known to be true either way. By contrast,
the stronger presupposition predicts infelicity in this case, since we
do not know the presupposition to obtain---given our ignorance about
what was \emph{known} at the past time, the presupposition would send us
looking for a wider interpretation of accessibility on which we now know
that some worlds where Oswald didn't kill Kennedy are
accessible.\footnote{*** A view where we are somehow allowed to
  substitute lack of known vacuity at the past time for lack of vacuity
  known now would be even worse, allowing lots of
  material-conditional-like interpretations.} \footnote{??? Saying `we
  are playing a game where you are only allowed to say something if it
  was known at the time' is no good, obviously.}

So far we have been talking about the presuppositions of standalone
utterances of conditionals. When conditionals are embedded inside more
complex sentences, there is a standard lore about ``presupposition
projection'' which can be used to generate predictions for what those
more complex sentences will presuppose. For example, the standard lore
says that when φ presupposes that P, `x thinks that φ' will typically
presuppose that x thinks that P, and not presuppose that P: for example,
in uttering `He thinks that the ghost haunting his house is benevolent'
we do not normally take it for granted that there is a ghost, but we do
seem to be taking it for granted that he thinks that there is a ghost
haunting his house. For conjunctive sentences, the lore says that when
relative a certain context φ asserts P and presupposes P′ and ψ asserts
Q and presupposes Q′, then relative to that context, 'φ and ψ'
presupposes that P′ and if P and P′, Q′.\footnote{The lore is rather
  unspecific as regards how that conditional should be
  interpreted---perhaps it is just a material conditional.} The
presupposition of nonvacuity can be slotted comfortably into this
framework. For example, when we say `He thinks that if Clark Kent lived
far away from Superman he would be happier' we are not committed to the
antecedent in fact being metaphysically possible, but we do seem to be
taking it for granted that the subject thinks it is.\footnote{*** The
  lore also says that presuppositions carry over to questions, which
  also fits.} Likewise, when we say `It might be raining, and if it is
the ground is slippery' we are clearly not representing ourselves as
\emph{taking it for granted} that it might be raining. The standard lore
for conjunctions just described explains this, since it predicts that
the second conjunct's presupposition gets conditionalised into the
trivial `If it might be raining, it might be raining'.

The label `presupposition' is used to cover a rather disparate array of
phemomena. The common thread is that in uttering a certain sentence, one
presents oneself as taking certain things for granted, as opposed to
presenting them as answers to the main questions at hand. But there are
significant differences in how tight the association is between uttering
a self-standing sentence and presenting oneself as taking the relevant
propositions for granted. At one extreme lie parentheticals, where the
connection is very tight indeed: one can scarcely imagine a context
where one utters `The president (who is a Democrat) comes from Chicago'
unembedded without presenting oneself as taking for granted that the
president is a Democrat. (Similarly for when it's the first conjunct of
a conjunction.) By contrast, while uttering `He doesn't know that it's
raining' normally suggests that one takes it for granted that it is
raining, there are exceptional cases where this suggestion is
absent---for example in `He doesn't \emph{know} it's raining! Remember
that he's relying on those notoriously unreliable instruments'.

With presuppositional sentences that lie towards the weaker end of this
spectrum, there is a broad and only partially explored array of
pragmatic factors that help to determine whether a suggestion that the
speaker takes a certain thing for granted will be heard in a given case.
One such factor that is especially interesting to us is that when
sentences would standardly have contradictory or obviously false
presuppositions, in some cases this leads the presupposition to be
suspended. For example

\begin{enumerate}
\def\labelenumi{(\arabic{enumi})}
\setcounter{enumi}{34}
\itemsep1pt\parskip0pt\parsep0pt
\item
  He doesn't know she is guilty, and he doesn't know she is innocent
\end{enumerate}

would be standardly predicted to presuppose something bizarre such as
`She is guilty, and if she's guilty and he doesn't know it, she is
innocent'; but in fact it carries no strong suggestion that anything in
particular is being taken for granted. Similarly

\begin{enumerate}
\def\labelenumi{(\arabic{enumi})}
\setcounter{enumi}{35}
\itemsep1pt\parskip0pt\parsep0pt
\item
  I won't salute the king and I won't salute the regent
\end{enumerate}

would be standardly predicted to presuppose that there is a king; but if
it's taken for granted that if there's a king there isn't a regent and
vice versa, an utterer of (27) will probably not be taken to be
committed to the presupposition predicted by standard lore.

We think we can see cases where the presupposition of nonvacuity behaves
in similar ways. Let's look at another poker example. Suppose I know
that Sally's four cards are either the 10, Jack, Queen, and King of
Spades, or two nines and two eights. I have two kings and three aces.
Sally folds; I look at the top card and see that it is the 9 of Spades.
I say

\begin{enumerate}
\def\labelenumi{(\arabic{enumi})}
\setcounter{enumi}{36}
\itemsep1pt\parskip0pt\parsep0pt
\item
  If she had drawn that card and had a straight she would have won, but
  if she had drawn that card and had a full house she would have lost.
\end{enumerate}

This seems acceptable given the facts. But to make it acceptable it
seems that we need to be holding fixed the actual makeup of Sally's
first four cards and my five---after all, if we allow accessible worlds
where the four cards in her hand are different, then the first conjunct
is quite dubious since not every straight beats a full house. However,
if all accessible worlds match actuality in these respects, then
inevitably one of the two counterfactuals will be vacuously true.

\begin{itemize}
\item
  Other examples that are prima facie problematic for our presupposition
  but where we think that a proper delicacy about the strength of the
  thing is all you need:

  (He knows where the keys are and is going to get them.) `Wherever he
  goes he'll open a drawer, because he knows that if it's in the kitchen
  it's in the kitchen drawer and if it's in the living room it's in the
  living room drawer'.
\item
  Problematic anti-materialist arguments: truth-value judgments;
  validity judgments (paradoxes of material implication).
\item
  Confidence-theoretic arguments against the commonness of material
  conditional readings.
\item
  Tell about the McDermott cases --- suggest that here at least the
  presupposition is weakened. (Speculate about why?)
\item
  mention monkey's uncle conditionals. {[}Puzzle - why don't we get them
  with counterfactuals? How do they relate to crazy conditional promises
  like `If \ldots{} I'll eat my hat' / `If I'm wrong about this just
  shoot me'.
\item
  Status of the presupposition - is it derivable by general (Gricean?)
  pragmatic considerations? We are not optimistic about this, though it
  wouldn't be a disaster. (It's easier to see how it would work if it
  were not-known-vacuity.)
\item
  Hard to cancel.
\item
  Maybe come back later to other common presuppositions/implicatures:
  inhomogeneity, falsity of antecedent for counterfactuals.
\end{itemize}

\subsection{Accessibility for
counterfactuals}\label{accessibility-for-counterfactuals}

As many authors have observed, our standard for evaluating a
counterfactual whose antecedent concerns a particular period of time
involves helping ourselves to all kinds of facts about earlier times.
For example, if John has had breakfast every day for the last year, we
will unhesitatingly endorse

\begin{enumerate}
\def\labelenumi{(\arabic{enumi})}
\setcounter{enumi}{37}
\itemsep1pt\parskip0pt\parsep0pt
\item
  If John had forgotten to have breakfast on Tuesday morning, that would
  have been the first time this year.
\end{enumerate}

The phenomenon is pervasive: even when the consequent of a
counterfactual isn't about the past, in deciding what to make of it we
will typically be tacitly holding the past fixed. For example, in
thinking about

\begin{enumerate}
\def\labelenumi{(\arabic{enumi})}
\setcounter{enumi}{38}
\itemsep1pt\parskip0pt\parsep0pt
\item
  If I had run a four minute mile this morning, I would have been
  extremely surprised.
\end{enumerate}

we seem to be holding fixed the speaker's previous track record.

Our favoured account of this phenomenon appeals to fine-grained
context-sensitivity in the accessibility parameter. When considering a
counterfactual whose antecedent is about a certain period, it is natural
to resolve its context-sensitivity in such a way that the accessible
worlds are all required to match with respect to earlier
times.\footnote{The match may not be perfect (cite), and it may need to
  be restricted to allow for a smooth transition into the kind of event
  described by the antecedent. Indeed for certain antecedents ---
  consider `If a giant comet had struck Washington DC yesterday
  afternoon\ldots{}.' --- we may need quite a long stretch of preceding
  time where there is nothing like exact match in order to achieve the
  smooth transition.} Notice that the closeness relation plays no role
in this account of why the relevant counterfactuals are true. So for
example, while our account suggests that (29*) is true in the context
\emph{it} naturally evokes for the same reason as (29), it provides no
reason to expect (29*) to be true relative to context naturally evoked
by (29):

(29*) If John had forgotten to have breakfast on Wednesday morning, that
would have been the first time this year.

After all, the context evoked by (29) is one where the accessible worlds
are merely required to match with respect to history up to Tuesday;
since some of them involve forgetting to have breakfast on Tuesday and
on Wednesday, the accessibility facts provide no guarantee for the truth
of (29*) in this context. And indeed, the account of closeness that we
will develop in chapter 2 will provide no grounds for confidence that
the closest worlds where John forgets breakfast on Wednesday are not
worlds where he forgets breakfast on Tuesday as well. Meanwhile, while
(29**) is non-vacuously true in the context it naturally evokes, it is
vacuously true in the context evoked by (29).

(29**) If John had forgotten to have breakfast on Monday morning, that
would have been the first time this year.

By contrast, Lewis's influential treatment of counterfactuals tells a
rather different story about why holding the past fixed is a reliable
way of evaluating counterfactuals. For Lewis, what does the work is
closeness, not accessibility---indeed, his account would work even on
the assumption that all metaphysically possible worlds are accessible in
all contexts.\footnote{Check if Lewis says this. Mention others who do.}
Moreover, Lewis's account of the phenomenon does not rely in any
essential way on context-sensitivity. For example, he thinks that there
is a single context on which we can knowledgeably utter (29), (29*), and
(29**): given that we know that John in fact had breakfast every day
this year, we know that worlds where John skips breakfast for the first
time on Wednesday are closer than worlds where he skips breakfast for
the first time on Tuesday, which are in turn closer than worlds where he
skipped breakfast on Monday. More generally, the closeness relation that
Lewis thinks is standard is one on which worlds whose history diverges
from that of the actual world at later times are ipso facto closer than
worlds where history diverges earlier (Lewis 1979).

This feature of Lewis's view leads to some well-known oddities that our
view avoids. Pollock (***) noticed that Lewis's view seems to underwrite
counterfactuals such as the following:

\begin{enumerate}
\def\labelenumi{(\arabic{enumi})}
\setcounter{enumi}{39}
\itemsep1pt\parskip0pt\parsep0pt
\item
  If my coat had been stolen last year it would have been stolen on
  December 31st.
\end{enumerate}

Given that my coat was not in fact stolen, on Lewis's view worlds where
it is stolen are closer to the extent that they diverge later from the
actual world.\footnote{This is a slight oversimplification: if there are
  certain days on which it would have taken a large miracle for my coat
  to be stolen they will count as further away from actuality on Lewis's
  account. But this doesn't make the problem go away. Let's just make
  the plausible assumption that on any day last year, a small miracle
  (perhaps taking place in the brain of some errant youth) on that day
  would have been enough to take the world onto a trajectory where my
  coat gets stolen on that day.} But intuitively, unless I have some
special reason to think that my coat was unusually vulnerable to theft
on December 31st, I have no right to assert (44).

It would really be quite bizarre if we had to start computing
counterfactuals in the way seemingly recommended by Lewis. Just to take
one more example: suppose that, sadly, someone fell from a ship and
drowned while people stood and watched without anyone diving in to help.
On Lewis's approach, there seems to be a quite strong reason to accept

\begin{enumerate}
\def\labelenumi{(\arabic{enumi})}
\setcounter{enumi}{40}
\itemsep1pt\parskip0pt\parsep0pt
\item
  If someone had dived in to try to help her, she would still have
  drowned.
\end{enumerate}

After all, if the later the divergence the closer, then the closest
worlds will be ones where someone dives in too late, even if there are
plenty of worlds where a more timely rescue attempt would have saved
her.

The examples cry out for a treatment on which the only period of history
that is completely held fixed is one that predates the period in
question---last year in the case of (44), the period during which the
drowning victim was in the water in the case of (45). A contextualist
approach like ours can readily accommodate this. Given the antecedent of
(44), it would be completely unmotivated to select some time after the
beginning of last year as the basis for resolving accessibility: the
natural constraint on accessibility will allow more or less
indiscriminate importation of facts about the actual world only for
times prior to last year.

One possible response to the problems for Lewis's account would be to
somehow appeal to the idea that for some reason (44) is computed as
equivalent to

(44′) On any day last year, if my coat had been stolen on that day it
would have been stolen on December 31st.

It's not obvious what would motivate this seemingly \emph{ad hoc}
mechanism beyond a desire to avoid the counterexamples. But in any case,
the proposal is clearly inadequate. Suppose that halfway through the
year I moved from a dangerous neighbourhood to a much safer one; then I
could naturally assert

\begin{enumerate}
\def\labelenumi{(\arabic{enumi})}
\setcounter{enumi}{41}
\itemsep1pt\parskip0pt\parsep0pt
\item
  Probably if my coat had been stolen last year, it would have been
  stolen during the first half of the year.
\end{enumerate}

By contrast, (46′) is obviously unassertable:

(46′) Probably on any day last year, if my coat had been stolen on that
year it would have been stolen during the first half of the year.

The problems raised by the coat and drowning examples are not peculiar
to the details of Lewis's approach, or even to the possible worlds
framework. Given certain minimal logical assumptions, problems of this
sort will arise for any view that attempts to secure the reliability of
`holding the past fixed' for antecedents about different times without
appealing to context-shift. Suppose that there was a thunderstorm all
weekend in the area that destroyed many trees, but my tree luckily
survived. Now consider

\begin{enumerate}
\def\labelenumi{(\arabic{enumi})}
\setcounter{enumi}{42}
\itemsep1pt\parskip0pt\parsep0pt
\item
  If this tree had been in pieces on Sunday morning, Sunday would have
  been an especially sad day for me.
\end{enumerate}

This is a typical example of the kind of sentence that is evaluated by
holding the past fixed; so it's plausible that in the context naturally
evoked by (), () is true:

\begin{enumerate}
\def\labelenumi{(\arabic{enumi})}
\setcounter{enumi}{43}
\itemsep1pt\parskip0pt\parsep0pt
\item
  If the tree had been dead on Sunday, it would have been alive on
  Saturday afternoon.
\end{enumerate}

But surely (45) and (46) are true as well (in the relevant context):

\begin{enumerate}
\def\labelenumi{(\arabic{enumi})}
\setcounter{enumi}{44}
\item
  If the tree had been dead on Sunday, it would have died during the
  weekend.
\item
  If the tree had died during the weekend, it would have been dead on
  Sunday.
\end{enumerate}

After all, Sunday is part of the weekend, so denying (45) looks
completely unpromising; meanwhile given that trees don't reform once
they have split into pieces, (46) seems safe too. But given (45), (46),
and (44), we are in a position to apply the inference-schema sometimes
called `CSO', according to which arguments of the following form are
valid:

\begin{description}
\itemsep1pt\parskip0pt\parsep0pt
\item[CSO]
If A then B; If B then A; If A then C ⊦ If B then C
\end{description}

Plugging in `The tree was dead on Sunday', `The tree died during the
weekend', and `The tree was alive on Saturday afternoon' for A, B, and
C, we can derive

\begin{enumerate}
\def\labelenumi{(\arabic{enumi})}
\setcounter{enumi}{46}
\itemsep1pt\parskip0pt\parsep0pt
\item
  If the tree had died during the weekend, it would have been alive on
  Saturday afternoon.
\end{enumerate}

But () seems very problematic in the same way as the coat and drowning
examples considered earlier.

CSO is valid according the best-known logics for counterfactual
conditionals, namely those of Stalnaker and Lewis. And to our my minds
its popularity is well-deserved. While it may not be immediately
compelling when first encountered, it follows from two other principles
whose appeal is more immediate, namely:

\begin{description}
\itemsep1pt\parskip0pt\parsep0pt
\item[Cumulative Transitivity (CT)]
If A then B; if A and B then C ⊦ If A then C
\item[Very Limited Antecedent Strengthening (VLAS)]
If A then B; If A then C ⊦ If A and B then C
\end{description}

For suppose the premises of CSO hold: if A, B; if B, A; and if A, C.
Given the first and third, by VLAS it follows that if A and B then C;
and from this and the second premise (If B, A) CT yields that if B,
C.\footnote{Also note how CSO entails CT and VLAS.}

Reliance on CT seems to pervade a great deal of our ordinary
counterfactual reasoning. Consider for example just how good the
following piece of reasoning looks: `If they attacks they will use their
cavalry; if they attacks using their cavalry, they will win; so if they
attack, they will win'. Clearly there is some general principle
underlying such reasoning, and it is hard to see what it could be if CT
is invalid. Regarding VLAS, similar remarks apply to the following
excellent argument (which involves a contraposed application of VLAS):
`If they attack they will use their mercenaries; but it's not the case
that if they attack using their mercenaries they will win; so it's not
the case that if they attack they will win'.

Given the general considerations for thinking that context-sensitivity
is pervasive and the costs of giving up CSO, it seems to us much more
appealing to explain the relevant data by appeal to context sensitivity
rather than jettisoning CSO. On the picture we favour, () is indeed
straightforwardly true in the context naturally evoked by (), but would
be a very risky assertion in the context it naturally evokes. Those who
endorse CSO but think that there is no relevant context-sensitivity in
play will, like Lewis, be forced to accept the problematic ().

The diagnosis of context-sensitivity enjoys substantial independent
support, since there are many cases where counterfactuals with the same
antecedent suggest different ways of holding the past fixed. For
example, each of () and () seems to be true in the natural context they
suggest:

\begin{enumerate}
\def\labelenumi{(\arabic{enumi})}
\setcounter{enumi}{47}
\item
  If I had been in the Caribbean this morning I would have been feeling
  refreshed
\item
  If I had been in the Caribbean this morning I would have been
  exhausted from travelling
\end{enumerate}

In these cases it is the consequent rather than the antecedent that
provides the linguistic cues as regards which portions of the past to
hold fixed.\footnote{Counterfactuals with stative antecedents especially
  prone to this.} (Of course there may be non-linguistic cues as well.)
On a view where (48) and (49) are both true in the same context, one
would have to either live with bizarre counterfactuals like ()

\begin{enumerate}
\def\labelenumi{(\arabic{enumi})}
\setcounter{enumi}{49}
\itemsep1pt\parskip0pt\parsep0pt
\item
  If I had been in the Caribbean this morning I would have been feeling
  refreshed and exhausted from travelling
\end{enumerate}

or else postulate widespread violations of an even more obviously
compelling inference rule, namely Conjunction:

Conjunction: If A, B; If A, C ⊦ If A, B and C

The appeal to context-sensitivity is thus pretty much inevitable when it
comes to sentences like () and (). We see no reason to deny that the
phenomenon manifested by the previously examples is any different: in
both cases, contextual triggers determine how much of the past is held
fixed in all accessible worlds.

Lewis, for his part, is willing to play the context-sensitivity card in
a much more limited way. His picture is that there is a ``standard''
resolution of context-sensitivity under which worlds that diverge later
are closer, and a separate category of ``backtracking'' resolutions of
context-sensitivity where a different standard of closeness is in play.
Lewis gives various examples where he thinks a backtracking resolution
of context-sensitivity is natural, such as `If Jim were to ask Jack for
help today, there would have to have been no quarrel yesterday' (Lewis
1979, p.~33). He might say that one or both of our sentences () and ()
fall into this category as well.\footnote{While the examples Lewis
  focuses on are sentences where the consequent directly concerns times
  earlier than those mentioned in the antecedent, he is careful to leave
  open the possibility that the non-standard resolutions of
  context-sensitivity required to handle those are also natural for a
  range of other counterfactual utterances that don't have this
  structure.} Lewis never develops a theory about what plays the role of
the closeness relation in backtracking contexts. But we think there is
very little prospect for a theory that posits a \emph{single} closeness
relation for backtracking contexts. Typically even evaluating a
counterfactual in a ``backtracking'' way we still draw freely on quite a
lot of facts about the past---for example, in evaluating () we probably
draw on the fact that in the not-so-distant past the speaker was not
feeling refreshed. There is no prospect of a general rule saying how
much of the past to hold fixed for evaluating backtrackers. Given that
fine-grained context-sensitivity is thus hard to avoid when dealing with
the counterfactuals Lewis categorises as backtrackers, it would not at
all surprising if, as we contend, it also extends to those he
categorises as non-backtrackers.

------

As well as requiring some bullet-biting about cases like Pollock's coat,
Lewis's approach also requires him to give up a prima facie compelling
principle about the logic of counterfactuals. If time extends infinitely
into the future, and worlds are closer the later they diverge from
actuality, then for every non-actual world, there is a closer one. The
closeness relation thus fails to obey the `Limit Assumption', which says
that every set of worlds contains some worlds that are as close to
actuality as any world in the set. Lewis is well aware of this
structural feature, and crafts his truth conditions accordingly: a
counterfactual is true iff either there is no (accessible) world where
the antecedent is true, or at least one world where antecedent is true
is such that every world where the antecedent is true that is at least
as close as it is one where the consequent is true. However, the upshot
of this semantics is that counterfactuals do not generally obey the
following principle:

AGGLOMERATION: if some propositions each would have been true if a
certain condition had obtained, then if that condition had obtained, all
of them would have been true

On Lewis's account AGGLOMERATION is not true in general, although it is
true for finite collections of propositions. To see why it can fail in
the infinite case, consider the collection of propositions which
contains, for each time \emph{t}, the proposition that history proceeds
just as it actually does until \emph{t}. Each conditional of the form

\begin{enumerate}
\def\labelenumi{(\arabic{enumi})}
\setcounter{enumi}{50}
\itemsep1pt\parskip0pt\parsep0pt
\item
  If history had diverged from actuality at some point, then history
  would have proceeded just as it actually does until \emph{t}
\end{enumerate}

will be true, since for any \emph{t} worlds which diverge after \emph{t}
are closer than worlds which diverge earlier. But clearly the following
is false:

\begin{enumerate}
\def\labelenumi{(\arabic{enumi})}
\setcounter{enumi}{51}
\itemsep1pt\parskip0pt\parsep0pt
\item
  If history had diverged from actuality at some point, then each
  \emph{t} is such that history would have proceed just as it actually
  does until \emph{t.}
\end{enumerate}

As many authors have noted---starting with Pollock ***---the intuitive
force of AGGLOMERATION carries over very smoothly from the finite to the
infinite case. It is thus a uncomfortable feature of Lewis's view that
it requires driving a wedge between the two kinds of cases.

This feature is very hard to avoid on any account that attempts to
accommodate the phenomena without making essential appeal to
context-sensitivity. For on such an account, all sentences of the form

\begin{enumerate}
\def\labelenumi{(\arabic{enumi})}
\setcounter{enumi}{52}
\itemsep1pt\parskip0pt\parsep0pt
\item
  If things had gone differently on day \emph{n} or later, things would
  have been the same up to day \emph{n}
\end{enumerate}

will be true in a single context. Moreover, given the abundance of
causal influence in the futurewards direction, there is strong pressure
to accept all conditionals of the form

\begin{enumerate}
\def\labelenumi{(\arabic{enumi})}
\setcounter{enumi}{53}
\itemsep1pt\parskip0pt\parsep0pt
\item
  If things had gone differently on some day, things would have gone
  differently on some day later than day \emph{n}.
\end{enumerate}

After all, a scenario where the only differences occur before day
\emph{n} and then history then re-converges to that of the actual world
seems quite far-fetched. But obviously (3) is also true irrespective of
context:

\begin{enumerate}
\def\labelenumi{(\arabic{enumi})}
\setcounter{enumi}{54}
\itemsep1pt\parskip0pt\parsep0pt
\item
  If things had gone differently on day \emph{n} or later, things would
  have gone differently on some day.
\end{enumerate}

So by CSO, we can conclude that all the conditionals of the following
form are true in the same context:

\begin{enumerate}
\def\labelenumi{(\arabic{enumi})}
\setcounter{enumi}{55}
\itemsep1pt\parskip0pt\parsep0pt
\item
  If things had either gone differently on some day, things would have
  been the same up to day \emph{n}.
\end{enumerate}

This is obviously inconsistent with AGGLOMERATION. The case for (2) is
perhaps not quite watertight: one might think that for all we know, if
there had been any divergence at all it would have been a very slight
and temporary divergence confined to some initial finite period. But we
can forestall this response by substituting the property of being a day
such that things go differently from how they actually go on it and all
subsequent days for the property of being a day on which things go
differently from how the actually go throughout the argument: then (2)
and (3) both become trivial, while the case for (1) is in no way
weakened.

Our view preserves AGGLOMERATION: if each of a certain class of
propositions is true at the closest accessible world where P is true,
then at that world, all of those propositions are true. We think that
the above argument against AGGLOMERATION fails because there is no
single context in which all sentences of the form (1) are true. Each,
nevertheless, is true in some contexts, and indeed has some tendency to
evoke the kind of context in which it is true.\footnote{***Note that the
  relevant tendency seems considerably stronger for counterfactuals with
  eventive antecedents. {[}Is this a good place to show an awareness of
  the H? paper about `event' conditionals?{]}}

Given the centrality of CSO to the arguments we have been considering,
it is worth addressing a particular kind of putative counterexample to
CSO that has been presented in some recent work by Kit Fine. Fine is
concerned in the first instance not with CSO but with the principle that
logically equivalent sentences can be substituted salva veritate in
conditionals. On Fine's view, for example, () is true whereas () is
false, despite the fact that its antecedent is logically equivalent to
that of ():

\begin{enumerate}
\def\labelenumi{(\arabic{enumi})}
\setcounter{enumi}{56}
\item
  If the tree had been dead on Sunday, it would have been alive on
  Saturday afternoon.
\item
  If the tree had been either dead on Sunday and not alive on Saturday
  afternoon, or dead on Sunday and alive on Saturday afternoon, it would
  have been alive on Saturday afternoon.
\end{enumerate}

These sentences also generate counterexample to CSO, since Fine accepts
that the logically equivalent antecedents of () and () are also
``counterfactually equivalent'' in the sense that () and () are both
true:

\begin{enumerate}
\def\labelenumi{(\arabic{enumi})}
\setcounter{enumi}{58}
\item
  If the tree had been dead on Sunday, it would have been either dead on
  Sunday and not alive on Saturday afternoon or dead on Sunday and alive
  on Saturday afternoon
\item
  If the tree had been either dead on Sunday and not alive on Saturday
  afternoon or dead on Sunday and alive on Saturday afternoon, it would
  have been dead on Sunday.
\end{enumerate}

Fine develops a new kind of model-theoretic semantics for
counterfactuals which is capable of validating these judgments. While we
don't propose to work through the details of this semantics here, we do
wish to draw attention to two of its key features. The first feature is
that `If A or B, C' turn out to be logically equivalent to the
conjunction of `If A, C' and `If B, C'. Thus according to Fine, ()
entails the obviously false (), whereas () does not:

\begin{enumerate}
\def\labelenumi{(\arabic{enumi})}
\setcounter{enumi}{60}
\itemsep1pt\parskip0pt\parsep0pt
\item
  If the tree had been dead on Sunday and not alive on Saturday
  afternoon, it would have been alive on Saturday afternoon.
\end{enumerate}

Now, as we mentioned in the introduction (*PAGEREF*), we are in
agreement with Fine to this extent: sentences of the form `If A or B, C'
really do admit quite natural readings on which they are equivalent to
`If A, C and if B, C'. (We be returning to this phenomenon in chapter
***: our treatment of it will involve giving up the idea that the
relevant conditionals can be characterised as predicating a relation of
two propositions, and will appeal to context-sensitivity in a way that
leaves CSO intact.)

The second feature of Fine's view that we want to note is the fact that
the semantics allows us to classify each sentence (relative to a
context) as either disjunctive or non-disjunctive. Disjunctions are
always disjunctive; atomic sentences may or may not be disjunctive.
(Fine is amenable to a dose of context-dependence here.) Given that he
has this ideology at his disposal, it would thus be natural for Fine to
accommodate the seeming goodness of the inferences whose validity we
wanted to explain by appeal to CSO (or by appeal to CT and/or VLAS) by
maintaining that these inference-rules are valid when restricted to
conditionals with non-disjunctive antecedents and consequents. In any
case, Fine does not present any cases that make trouble for this
fallback position. Insofar as Fine's new machinery provides a diagnosis
of what goes wrong with the CSO-based arguments we have been
considering, the diagnosis will thus have to turn that some of the
relevant sentences are disjunctive (in the relevant contexts). For
example, it would be natural for Fine to say that the antecedent of ()
is disjunctive:

\begin{enumerate}
\def\labelenumi{(\arabic{enumi})}
\setcounter{enumi}{61}
\itemsep1pt\parskip0pt\parsep0pt
\item
  If the tree had died during the weekend, it would have been alive on
  Saturday afternoon
\end{enumerate}

Disjunctive sentences behave in the semantics like explicit
disjunctions. Thus if, for example, Fine said that the antecedent of ()
was tantamount to the disjunction `Either the tree died on Saturday or
the tree died on Sunday', () would turn out to be equivalent to ():

\begin{enumerate}
\def\labelenumi{(\arabic{enumi})}
\setcounter{enumi}{62}
\itemsep1pt\parskip0pt\parsep0pt
\item
  If the tree had died on Saturday, it would have been alive on Saturday
  afternoon, and if the tree had died on Sunday, it would have been
  alive on Saturday afternoon.
\end{enumerate}

which sounds straightforwardly false.

On big problem we see with this account of () is that, in the context it
naturally evokes, () doesn't seem to be straightforwardly true or
straightforwardly false. In that context, it would be natural to say

\begin{enumerate}
\def\labelenumi{(\arabic{enumi})}
\setcounter{enumi}{63}
\itemsep1pt\parskip0pt\parsep0pt
\item
  If the tree had been in pieces at some point during the weekend, it
  might or might not have been intact all day Saturday
\end{enumerate}

Moreover, there are other contexts where a sentence such as () seems
more or less straightforwardly true, even though Fine seems to predict
that it's false. For example if there was no storm on Saturday but there
was one on Sunday it would be quite natural to assert ():

\begin{enumerate}
\def\labelenumi{(\arabic{enumi})}
\setcounter{enumi}{64}
\itemsep1pt\parskip0pt\parsep0pt
\item
  If the tree had died during the weekend, it would have died on Sunday.
\end{enumerate}

The problem is not essentially due to the postulated semantic
equivalence of `The tree died during the weekend' to `The tree died on
Saturday or Sunday', since the same problematic issues arise even when
we substitute that explicit disjunction into () and ().

Fine does have one other resource that he invokes in somewhat related
contexts. He notices that sentences like

\begin{enumerate}
\def\labelenumi{(\arabic{enumi})}
\setcounter{enumi}{65}
\itemsep1pt\parskip0pt\parsep0pt
\item
  If I had moved to the West coast or the East coast, I would have moved
  to the West coast.
\end{enumerate}

can be uttered felicitously even though his semantics predicts that it
entails the absurd-sounding

\begin{enumerate}
\def\labelenumi{(\arabic{enumi})}
\setcounter{enumi}{66}
\itemsep1pt\parskip0pt\parsep0pt
\item
  If I had moved to the East coast, I would have moved to the West
  coast.
\end{enumerate}

What he says is this: while (68) is indeed an absurd thing to utter
because of the presupposition of non vacuity, relative to the context in
which (67) is uttered felicitously, (68) is in fact not absurd but
instead vacuously true. Fine could say something similar about the
problem sentence (),

\begin{enumerate}
\def\labelenumi{(\arabic{enumi})}
\setcounter{enumi}{67}
\itemsep1pt\parskip0pt\parsep0pt
\item
  If it had died on Saturday or on Sunday, it would have died on Sunday.
\end{enumerate}

But this does not look like a promising suggestion in settings where we
are inclined to make remarks like

\begin{enumerate}
\def\labelenumi{(\arabic{enumi})}
\setcounter{enumi}{68}
\itemsep1pt\parskip0pt\parsep0pt
\item
  If it had died on Saturday or on Sunday, it might or might not have
  died on Saturday
\end{enumerate}

Since the truth of () turns on whether (64) is vacuously true, the kind
of ignorance we are avowing in uttering () would have, for Fine, to turn
on on ignorance as regards whether () is vacuously true. We have trouble
seeing how a plausible theory of vacuous truth could fit the bill.

It is time to turn briefly to an objection that we anticipate being
directed against our view. Consider the sequence of conditionals

\begin{enumerate}
\def\labelenumi{(\arabic{enumi})}
\setcounter{enumi}{69}
\item
  \begin{enumerate}
  \def\labelenumii{\alph{enumii}.}
  \itemsep1pt\parskip0pt\parsep0pt
  \item
    If I had skipped breakfast on Monday, that would have been the first
    time I did so all year

    \begin{enumerate}
    \def\labelenumiii{\alph{enumiii}.}
    \setcounter{enumiii}{1}
    \itemsep1pt\parskip0pt\parsep0pt
    \item
      \ldots{}
    \item
      If I had skipped breakfast on Sunday, that would have been the
      first time I did so all year
    \end{enumerate}
  \end{enumerate}
\end{enumerate}

On our account, each is true in the context it naturally suggests, but
there is no context where we can be confident that all of them are
nonvacuously true. But note that there appears to be a reading of the
quantified sentence (71) on which it is true:

\begin{enumerate}
\def\labelenumi{(\arabic{enumi})}
\setcounter{enumi}{70}
\itemsep1pt\parskip0pt\parsep0pt
\item
  Each day last week is such that if I had skipped breakfast on that
  day, it would have been the first time I did so all year.
\end{enumerate}

It is hard to deny that this quantified claim is true. Relevantly
similar claims occur in many other settings.

\begin{enumerate}
\def\labelenumi{(\arabic{enumi})}
\setcounter{enumi}{71}
\item
  If any member of my family had voted Republican she would have been he
  only one to do so
\item
  I bought lottery tickets on several occasions last year. On each of
  those occasions, if I had won I would have more than quadrupled my
  wealth.
\end{enumerate}

It is easy for someone like Lewis to explain how such sentences could
get be true in a single context: a single closeness relation along the
lines he suggests can vindicate each witness to the quantified claim.
But such sentences pose a challenge to our account, on which the
relevant explanatory work is done by accessibility rather than
closeness. For example, if for each day last week the set of accessible
worlds contains a world where I skip breakfast on that day, it is hard
to see what would stop there from being an accessible world where I skip
breakfast on two days last week. And unless we impose special
Lewis-style constraints on the closeness relation (something which we
want to resist) there is no obvious ground for confidence that the
closest world where I skip on a certain day isn't one where I have
already skipped on some earlier day.

Our response to this objection draws on some quite pervasive facts about
the interaction of quantification with context-dependence. The paradigm
on which we think most context-sensitive words should be modelled is
`local'. Sometimes, the semantic contribution of `local' is a particular
property, say the property of being in Brooklyn, or perhaps the property
of being within a certain distance of Barack Obama. But on other
occasions, `local' behaves semantically as if it contained a variable
bound by some higher quantifier. For example

\begin{enumerate}
\def\labelenumi{(\arabic{enumi})}
\setcounter{enumi}{73}
\itemsep1pt\parskip0pt\parsep0pt
\item
  Each of my brothers went to a local tattoo parlour
\end{enumerate}

can mean that each of my brothers went to a tattoo parlour \emph{in the
neighbourhood he lives in}. Readings with a similar structure seem to be
available for just about any context-sensitive expression (with the
exception of `I' and possibly a few others). For example `Each of my
brothers is angry that we don't spend enough time together' can mean
`Each of my brothers is angry that \emph{he and I} don't spend enough
time together'. `Everyone in my family is tall' can mean `Everyone in my
family is tall \emph{for his or her age}'. `Everyone is scared because
there might be a spider in the closet' can mean, roughly, `Everyone is
scared because it is \emph{an epistemic possibility for them} that there
is a spider in the closet'. And so forth. We intend `accessible' to fit
this pattern. So () could, for example, mean something like

\begin{enumerate}
\def\labelenumi{(\arabic{enumi})}
\setcounter{enumi}{74}
\itemsep1pt\parskip0pt\parsep0pt
\item
  Each day last week is such that either there is no world that matches
  the history of the actualised world up to that day where I skip
  breakfast on that day, or the closest such world is one where that day
  is the first day on which I skip breakfast all year.
\end{enumerate}

This provides, in effect, a different accessibility relation for each
day.

Circumstantial modals which stand to counterfactuals as epistemics stand
to indicatives: `It couldn't have happened that\ldots{}'

It would be nice to say something about why holding history fixed is the
canonical way of doing things. Connection to deliberation? (Look at Luke
Glynn recent paper?)

\begin{itemize}
\item
  mention Morgenbesser's coin too. `If I had bet on heads I would have
  won'. Return to (46) once we have said this.
\item
  Play up the unity of our account.
\item
  Notice that we didn't say that it always just is epistemic possibility
  (for some relevant person/group/etc\ldots{}) We think it can be
  stronger.
\end{itemize}

\begin{itemize}
\item
  Examples of modals where it is stronger: constrained uses.
\item
  Examples of constrained conditionals --- maybe just announce some
  without giving the probability-theoretic argument\ldots{}
\item
  To do in next chapter: Talk about Matt Bird's example
\end{itemize}
